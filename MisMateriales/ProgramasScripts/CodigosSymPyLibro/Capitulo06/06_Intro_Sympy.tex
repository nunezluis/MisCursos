\documentclass[11pt]{article}

    \usepackage[breakable]{tcolorbox}
    \usepackage{parskip} % Stop auto-indenting (to mimic markdown behaviour)
    

    % Basic figure setup, for now with no caption control since it's done
    % automatically by Pandoc (which extracts ![](path) syntax from Markdown).
    \usepackage{graphicx}
    % Maintain compatibility with old templates. Remove in nbconvert 6.0
    \let\Oldincludegraphics\includegraphics
    % Ensure that by default, figures have no caption (until we provide a
    % proper Figure object with a Caption API and a way to capture that
    % in the conversion process - todo).
    \usepackage{caption}
    \DeclareCaptionFormat{nocaption}{}
    \captionsetup{format=nocaption,aboveskip=0pt,belowskip=0pt}

    \usepackage{float}
    \floatplacement{figure}{H} % forces figures to be placed at the correct location
    \usepackage{xcolor} % Allow colors to be defined
    \usepackage{enumerate} % Needed for markdown enumerations to work
    \usepackage{geometry} % Used to adjust the document margins
    \usepackage{amsmath} % Equations
    \usepackage{amssymb} % Equations
    \usepackage{textcomp} % defines textquotesingle
    % Hack from http://tex.stackexchange.com/a/47451/13684:
    \AtBeginDocument{%
        \def\PYZsq{\textquotesingle}% Upright quotes in Pygmentized code
    }
    \usepackage{upquote} % Upright quotes for verbatim code
    \usepackage{eurosym} % defines \euro

    \usepackage{iftex}
    \ifPDFTeX
        \usepackage[T1]{fontenc}
        \IfFileExists{alphabeta.sty}{
              \usepackage{alphabeta}
          }{
              \usepackage[mathletters]{ucs}
              \usepackage[utf8x]{inputenc}
          }
    \else
        \usepackage{fontspec}
        \usepackage{unicode-math}
    \fi

    \usepackage{fancyvrb} % verbatim replacement that allows latex
    \usepackage{grffile} % extends the file name processing of package graphics
                         % to support a larger range
    \makeatletter % fix for old versions of grffile with XeLaTeX
    \@ifpackagelater{grffile}{2019/11/01}
    {
      % Do nothing on new versions
    }
    {
      \def\Gread@@xetex#1{%
        \IfFileExists{"\Gin@base".bb}%
        {\Gread@eps{\Gin@base.bb}}%
        {\Gread@@xetex@aux#1}%
      }
    }
    \makeatother
    \usepackage[Export]{adjustbox} % Used to constrain images to a maximum size
    \adjustboxset{max size={0.9\linewidth}{0.9\paperheight}}

    % The hyperref package gives us a pdf with properly built
    % internal navigation ('pdf bookmarks' for the table of contents,
    % internal cross-reference links, web links for URLs, etc.)
    \usepackage{hyperref}
    % The default LaTeX title has an obnoxious amount of whitespace. By default,
    % titling removes some of it. It also provides customization options.
    \usepackage{titling}
    \usepackage{longtable} % longtable support required by pandoc >1.10
    \usepackage{booktabs}  % table support for pandoc > 1.12.2
    \usepackage{array}     % table support for pandoc >= 2.11.3
    \usepackage{calc}      % table minipage width calculation for pandoc >= 2.11.1
    \usepackage[inline]{enumitem} % IRkernel/repr support (it uses the enumerate* environment)
    \usepackage[normalem]{ulem} % ulem is needed to support strikethroughs (\sout)
                                % normalem makes italics be italics, not underlines
    \usepackage{soul}      % strikethrough (\st) support for pandoc >= 3.0.0
    \usepackage{mathrsfs}
    

    
    % Colors for the hyperref package
    \definecolor{urlcolor}{rgb}{0,.145,.698}
    \definecolor{linkcolor}{rgb}{.71,0.21,0.01}
    \definecolor{citecolor}{rgb}{.12,.54,.11}

    % ANSI colors
    \definecolor{ansi-black}{HTML}{3E424D}
    \definecolor{ansi-black-intense}{HTML}{282C36}
    \definecolor{ansi-red}{HTML}{E75C58}
    \definecolor{ansi-red-intense}{HTML}{B22B31}
    \definecolor{ansi-green}{HTML}{00A250}
    \definecolor{ansi-green-intense}{HTML}{007427}
    \definecolor{ansi-yellow}{HTML}{DDB62B}
    \definecolor{ansi-yellow-intense}{HTML}{B27D12}
    \definecolor{ansi-blue}{HTML}{208FFB}
    \definecolor{ansi-blue-intense}{HTML}{0065CA}
    \definecolor{ansi-magenta}{HTML}{D160C4}
    \definecolor{ansi-magenta-intense}{HTML}{A03196}
    \definecolor{ansi-cyan}{HTML}{60C6C8}
    \definecolor{ansi-cyan-intense}{HTML}{258F8F}
    \definecolor{ansi-white}{HTML}{C5C1B4}
    \definecolor{ansi-white-intense}{HTML}{A1A6B2}
    \definecolor{ansi-default-inverse-fg}{HTML}{FFFFFF}
    \definecolor{ansi-default-inverse-bg}{HTML}{000000}

    % common color for the border for error outputs.
    \definecolor{outerrorbackground}{HTML}{FFDFDF}

    % commands and environments needed by pandoc snippets
    % extracted from the output of `pandoc -s`
    \providecommand{\tightlist}{%
      \setlength{\itemsep}{0pt}\setlength{\parskip}{0pt}}
    \DefineVerbatimEnvironment{Highlighting}{Verbatim}{commandchars=\\\{\}}
    % Add ',fontsize=\small' for more characters per line
    \newenvironment{Shaded}{}{}
    \newcommand{\KeywordTok}[1]{\textcolor[rgb]{0.00,0.44,0.13}{\textbf{{#1}}}}
    \newcommand{\DataTypeTok}[1]{\textcolor[rgb]{0.56,0.13,0.00}{{#1}}}
    \newcommand{\DecValTok}[1]{\textcolor[rgb]{0.25,0.63,0.44}{{#1}}}
    \newcommand{\BaseNTok}[1]{\textcolor[rgb]{0.25,0.63,0.44}{{#1}}}
    \newcommand{\FloatTok}[1]{\textcolor[rgb]{0.25,0.63,0.44}{{#1}}}
    \newcommand{\CharTok}[1]{\textcolor[rgb]{0.25,0.44,0.63}{{#1}}}
    \newcommand{\StringTok}[1]{\textcolor[rgb]{0.25,0.44,0.63}{{#1}}}
    \newcommand{\CommentTok}[1]{\textcolor[rgb]{0.38,0.63,0.69}{\textit{{#1}}}}
    \newcommand{\OtherTok}[1]{\textcolor[rgb]{0.00,0.44,0.13}{{#1}}}
    \newcommand{\AlertTok}[1]{\textcolor[rgb]{1.00,0.00,0.00}{\textbf{{#1}}}}
    \newcommand{\FunctionTok}[1]{\textcolor[rgb]{0.02,0.16,0.49}{{#1}}}
    \newcommand{\RegionMarkerTok}[1]{{#1}}
    \newcommand{\ErrorTok}[1]{\textcolor[rgb]{1.00,0.00,0.00}{\textbf{{#1}}}}
    \newcommand{\NormalTok}[1]{{#1}}

    % Additional commands for more recent versions of Pandoc
    \newcommand{\ConstantTok}[1]{\textcolor[rgb]{0.53,0.00,0.00}{{#1}}}
    \newcommand{\SpecialCharTok}[1]{\textcolor[rgb]{0.25,0.44,0.63}{{#1}}}
    \newcommand{\VerbatimStringTok}[1]{\textcolor[rgb]{0.25,0.44,0.63}{{#1}}}
    \newcommand{\SpecialStringTok}[1]{\textcolor[rgb]{0.73,0.40,0.53}{{#1}}}
    \newcommand{\ImportTok}[1]{{#1}}
    \newcommand{\DocumentationTok}[1]{\textcolor[rgb]{0.73,0.13,0.13}{\textit{{#1}}}}
    \newcommand{\AnnotationTok}[1]{\textcolor[rgb]{0.38,0.63,0.69}{\textbf{\textit{{#1}}}}}
    \newcommand{\CommentVarTok}[1]{\textcolor[rgb]{0.38,0.63,0.69}{\textbf{\textit{{#1}}}}}
    \newcommand{\VariableTok}[1]{\textcolor[rgb]{0.10,0.09,0.49}{{#1}}}
    \newcommand{\ControlFlowTok}[1]{\textcolor[rgb]{0.00,0.44,0.13}{\textbf{{#1}}}}
    \newcommand{\OperatorTok}[1]{\textcolor[rgb]{0.40,0.40,0.40}{{#1}}}
    \newcommand{\BuiltInTok}[1]{{#1}}
    \newcommand{\ExtensionTok}[1]{{#1}}
    \newcommand{\PreprocessorTok}[1]{\textcolor[rgb]{0.74,0.48,0.00}{{#1}}}
    \newcommand{\AttributeTok}[1]{\textcolor[rgb]{0.49,0.56,0.16}{{#1}}}
    \newcommand{\InformationTok}[1]{\textcolor[rgb]{0.38,0.63,0.69}{\textbf{\textit{{#1}}}}}
    \newcommand{\WarningTok}[1]{\textcolor[rgb]{0.38,0.63,0.69}{\textbf{\textit{{#1}}}}}


    % Define a nice break command that doesn't care if a line doesn't already
    % exist.
    \def\br{\hspace*{\fill} \\* }
    % Math Jax compatibility definitions
    \def\gt{>}
    \def\lt{<}
    \let\Oldtex\TeX
    \let\Oldlatex\LaTeX
    \renewcommand{\TeX}{\textrm{\Oldtex}}
    \renewcommand{\LaTeX}{\textrm{\Oldlatex}}
    % Document parameters
    % Document title
    \title{06\_Intro\_Sympy}
    
    
    
    
    
    
    
% Pygments definitions
\makeatletter
\def\PY@reset{\let\PY@it=\relax \let\PY@bf=\relax%
    \let\PY@ul=\relax \let\PY@tc=\relax%
    \let\PY@bc=\relax \let\PY@ff=\relax}
\def\PY@tok#1{\csname PY@tok@#1\endcsname}
\def\PY@toks#1+{\ifx\relax#1\empty\else%
    \PY@tok{#1}\expandafter\PY@toks\fi}
\def\PY@do#1{\PY@bc{\PY@tc{\PY@ul{%
    \PY@it{\PY@bf{\PY@ff{#1}}}}}}}
\def\PY#1#2{\PY@reset\PY@toks#1+\relax+\PY@do{#2}}

\@namedef{PY@tok@w}{\def\PY@tc##1{\textcolor[rgb]{0.73,0.73,0.73}{##1}}}
\@namedef{PY@tok@c}{\let\PY@it=\textit\def\PY@tc##1{\textcolor[rgb]{0.24,0.48,0.48}{##1}}}
\@namedef{PY@tok@cp}{\def\PY@tc##1{\textcolor[rgb]{0.61,0.40,0.00}{##1}}}
\@namedef{PY@tok@k}{\let\PY@bf=\textbf\def\PY@tc##1{\textcolor[rgb]{0.00,0.50,0.00}{##1}}}
\@namedef{PY@tok@kp}{\def\PY@tc##1{\textcolor[rgb]{0.00,0.50,0.00}{##1}}}
\@namedef{PY@tok@kt}{\def\PY@tc##1{\textcolor[rgb]{0.69,0.00,0.25}{##1}}}
\@namedef{PY@tok@o}{\def\PY@tc##1{\textcolor[rgb]{0.40,0.40,0.40}{##1}}}
\@namedef{PY@tok@ow}{\let\PY@bf=\textbf\def\PY@tc##1{\textcolor[rgb]{0.67,0.13,1.00}{##1}}}
\@namedef{PY@tok@nb}{\def\PY@tc##1{\textcolor[rgb]{0.00,0.50,0.00}{##1}}}
\@namedef{PY@tok@nf}{\def\PY@tc##1{\textcolor[rgb]{0.00,0.00,1.00}{##1}}}
\@namedef{PY@tok@nc}{\let\PY@bf=\textbf\def\PY@tc##1{\textcolor[rgb]{0.00,0.00,1.00}{##1}}}
\@namedef{PY@tok@nn}{\let\PY@bf=\textbf\def\PY@tc##1{\textcolor[rgb]{0.00,0.00,1.00}{##1}}}
\@namedef{PY@tok@ne}{\let\PY@bf=\textbf\def\PY@tc##1{\textcolor[rgb]{0.80,0.25,0.22}{##1}}}
\@namedef{PY@tok@nv}{\def\PY@tc##1{\textcolor[rgb]{0.10,0.09,0.49}{##1}}}
\@namedef{PY@tok@no}{\def\PY@tc##1{\textcolor[rgb]{0.53,0.00,0.00}{##1}}}
\@namedef{PY@tok@nl}{\def\PY@tc##1{\textcolor[rgb]{0.46,0.46,0.00}{##1}}}
\@namedef{PY@tok@ni}{\let\PY@bf=\textbf\def\PY@tc##1{\textcolor[rgb]{0.44,0.44,0.44}{##1}}}
\@namedef{PY@tok@na}{\def\PY@tc##1{\textcolor[rgb]{0.41,0.47,0.13}{##1}}}
\@namedef{PY@tok@nt}{\let\PY@bf=\textbf\def\PY@tc##1{\textcolor[rgb]{0.00,0.50,0.00}{##1}}}
\@namedef{PY@tok@nd}{\def\PY@tc##1{\textcolor[rgb]{0.67,0.13,1.00}{##1}}}
\@namedef{PY@tok@s}{\def\PY@tc##1{\textcolor[rgb]{0.73,0.13,0.13}{##1}}}
\@namedef{PY@tok@sd}{\let\PY@it=\textit\def\PY@tc##1{\textcolor[rgb]{0.73,0.13,0.13}{##1}}}
\@namedef{PY@tok@si}{\let\PY@bf=\textbf\def\PY@tc##1{\textcolor[rgb]{0.64,0.35,0.47}{##1}}}
\@namedef{PY@tok@se}{\let\PY@bf=\textbf\def\PY@tc##1{\textcolor[rgb]{0.67,0.36,0.12}{##1}}}
\@namedef{PY@tok@sr}{\def\PY@tc##1{\textcolor[rgb]{0.64,0.35,0.47}{##1}}}
\@namedef{PY@tok@ss}{\def\PY@tc##1{\textcolor[rgb]{0.10,0.09,0.49}{##1}}}
\@namedef{PY@tok@sx}{\def\PY@tc##1{\textcolor[rgb]{0.00,0.50,0.00}{##1}}}
\@namedef{PY@tok@m}{\def\PY@tc##1{\textcolor[rgb]{0.40,0.40,0.40}{##1}}}
\@namedef{PY@tok@gh}{\let\PY@bf=\textbf\def\PY@tc##1{\textcolor[rgb]{0.00,0.00,0.50}{##1}}}
\@namedef{PY@tok@gu}{\let\PY@bf=\textbf\def\PY@tc##1{\textcolor[rgb]{0.50,0.00,0.50}{##1}}}
\@namedef{PY@tok@gd}{\def\PY@tc##1{\textcolor[rgb]{0.63,0.00,0.00}{##1}}}
\@namedef{PY@tok@gi}{\def\PY@tc##1{\textcolor[rgb]{0.00,0.52,0.00}{##1}}}
\@namedef{PY@tok@gr}{\def\PY@tc##1{\textcolor[rgb]{0.89,0.00,0.00}{##1}}}
\@namedef{PY@tok@ge}{\let\PY@it=\textit}
\@namedef{PY@tok@gs}{\let\PY@bf=\textbf}
\@namedef{PY@tok@gp}{\let\PY@bf=\textbf\def\PY@tc##1{\textcolor[rgb]{0.00,0.00,0.50}{##1}}}
\@namedef{PY@tok@go}{\def\PY@tc##1{\textcolor[rgb]{0.44,0.44,0.44}{##1}}}
\@namedef{PY@tok@gt}{\def\PY@tc##1{\textcolor[rgb]{0.00,0.27,0.87}{##1}}}
\@namedef{PY@tok@err}{\def\PY@bc##1{{\setlength{\fboxsep}{\string -\fboxrule}\fcolorbox[rgb]{1.00,0.00,0.00}{1,1,1}{\strut ##1}}}}
\@namedef{PY@tok@kc}{\let\PY@bf=\textbf\def\PY@tc##1{\textcolor[rgb]{0.00,0.50,0.00}{##1}}}
\@namedef{PY@tok@kd}{\let\PY@bf=\textbf\def\PY@tc##1{\textcolor[rgb]{0.00,0.50,0.00}{##1}}}
\@namedef{PY@tok@kn}{\let\PY@bf=\textbf\def\PY@tc##1{\textcolor[rgb]{0.00,0.50,0.00}{##1}}}
\@namedef{PY@tok@kr}{\let\PY@bf=\textbf\def\PY@tc##1{\textcolor[rgb]{0.00,0.50,0.00}{##1}}}
\@namedef{PY@tok@bp}{\def\PY@tc##1{\textcolor[rgb]{0.00,0.50,0.00}{##1}}}
\@namedef{PY@tok@fm}{\def\PY@tc##1{\textcolor[rgb]{0.00,0.00,1.00}{##1}}}
\@namedef{PY@tok@vc}{\def\PY@tc##1{\textcolor[rgb]{0.10,0.09,0.49}{##1}}}
\@namedef{PY@tok@vg}{\def\PY@tc##1{\textcolor[rgb]{0.10,0.09,0.49}{##1}}}
\@namedef{PY@tok@vi}{\def\PY@tc##1{\textcolor[rgb]{0.10,0.09,0.49}{##1}}}
\@namedef{PY@tok@vm}{\def\PY@tc##1{\textcolor[rgb]{0.10,0.09,0.49}{##1}}}
\@namedef{PY@tok@sa}{\def\PY@tc##1{\textcolor[rgb]{0.73,0.13,0.13}{##1}}}
\@namedef{PY@tok@sb}{\def\PY@tc##1{\textcolor[rgb]{0.73,0.13,0.13}{##1}}}
\@namedef{PY@tok@sc}{\def\PY@tc##1{\textcolor[rgb]{0.73,0.13,0.13}{##1}}}
\@namedef{PY@tok@dl}{\def\PY@tc##1{\textcolor[rgb]{0.73,0.13,0.13}{##1}}}
\@namedef{PY@tok@s2}{\def\PY@tc##1{\textcolor[rgb]{0.73,0.13,0.13}{##1}}}
\@namedef{PY@tok@sh}{\def\PY@tc##1{\textcolor[rgb]{0.73,0.13,0.13}{##1}}}
\@namedef{PY@tok@s1}{\def\PY@tc##1{\textcolor[rgb]{0.73,0.13,0.13}{##1}}}
\@namedef{PY@tok@mb}{\def\PY@tc##1{\textcolor[rgb]{0.40,0.40,0.40}{##1}}}
\@namedef{PY@tok@mf}{\def\PY@tc##1{\textcolor[rgb]{0.40,0.40,0.40}{##1}}}
\@namedef{PY@tok@mh}{\def\PY@tc##1{\textcolor[rgb]{0.40,0.40,0.40}{##1}}}
\@namedef{PY@tok@mi}{\def\PY@tc##1{\textcolor[rgb]{0.40,0.40,0.40}{##1}}}
\@namedef{PY@tok@il}{\def\PY@tc##1{\textcolor[rgb]{0.40,0.40,0.40}{##1}}}
\@namedef{PY@tok@mo}{\def\PY@tc##1{\textcolor[rgb]{0.40,0.40,0.40}{##1}}}
\@namedef{PY@tok@ch}{\let\PY@it=\textit\def\PY@tc##1{\textcolor[rgb]{0.24,0.48,0.48}{##1}}}
\@namedef{PY@tok@cm}{\let\PY@it=\textit\def\PY@tc##1{\textcolor[rgb]{0.24,0.48,0.48}{##1}}}
\@namedef{PY@tok@cpf}{\let\PY@it=\textit\def\PY@tc##1{\textcolor[rgb]{0.24,0.48,0.48}{##1}}}
\@namedef{PY@tok@c1}{\let\PY@it=\textit\def\PY@tc##1{\textcolor[rgb]{0.24,0.48,0.48}{##1}}}
\@namedef{PY@tok@cs}{\let\PY@it=\textit\def\PY@tc##1{\textcolor[rgb]{0.24,0.48,0.48}{##1}}}

\def\PYZbs{\char`\\}
\def\PYZus{\char`\_}
\def\PYZob{\char`\{}
\def\PYZcb{\char`\}}
\def\PYZca{\char`\^}
\def\PYZam{\char`\&}
\def\PYZlt{\char`\<}
\def\PYZgt{\char`\>}
\def\PYZsh{\char`\#}
\def\PYZpc{\char`\%}
\def\PYZdl{\char`\$}
\def\PYZhy{\char`\-}
\def\PYZsq{\char`\'}
\def\PYZdq{\char`\"}
\def\PYZti{\char`\~}
% for compatibility with earlier versions
\def\PYZat{@}
\def\PYZlb{[}
\def\PYZrb{]}
\makeatother


    % For linebreaks inside Verbatim environment from package fancyvrb.
    \makeatletter
        \newbox\Wrappedcontinuationbox
        \newbox\Wrappedvisiblespacebox
        \newcommand*\Wrappedvisiblespace {\textcolor{red}{\textvisiblespace}}
        \newcommand*\Wrappedcontinuationsymbol {\textcolor{red}{\llap{\tiny$\m@th\hookrightarrow$}}}
        \newcommand*\Wrappedcontinuationindent {3ex }
        \newcommand*\Wrappedafterbreak {\kern\Wrappedcontinuationindent\copy\Wrappedcontinuationbox}
        % Take advantage of the already applied Pygments mark-up to insert
        % potential linebreaks for TeX processing.
        %        {, <, #, %, $, ' and ": go to next line.
        %        _, }, ^, &, >, - and ~: stay at end of broken line.
        % Use of \textquotesingle for straight quote.
        \newcommand*\Wrappedbreaksatspecials {%
            \def\PYGZus{\discretionary{\char`\_}{\Wrappedafterbreak}{\char`\_}}%
            \def\PYGZob{\discretionary{}{\Wrappedafterbreak\char`\{}{\char`\{}}%
            \def\PYGZcb{\discretionary{\char`\}}{\Wrappedafterbreak}{\char`\}}}%
            \def\PYGZca{\discretionary{\char`\^}{\Wrappedafterbreak}{\char`\^}}%
            \def\PYGZam{\discretionary{\char`\&}{\Wrappedafterbreak}{\char`\&}}%
            \def\PYGZlt{\discretionary{}{\Wrappedafterbreak\char`\<}{\char`\<}}%
            \def\PYGZgt{\discretionary{\char`\>}{\Wrappedafterbreak}{\char`\>}}%
            \def\PYGZsh{\discretionary{}{\Wrappedafterbreak\char`\#}{\char`\#}}%
            \def\PYGZpc{\discretionary{}{\Wrappedafterbreak\char`\%}{\char`\%}}%
            \def\PYGZdl{\discretionary{}{\Wrappedafterbreak\char`\$}{\char`\$}}%
            \def\PYGZhy{\discretionary{\char`\-}{\Wrappedafterbreak}{\char`\-}}%
            \def\PYGZsq{\discretionary{}{\Wrappedafterbreak\textquotesingle}{\textquotesingle}}%
            \def\PYGZdq{\discretionary{}{\Wrappedafterbreak\char`\"}{\char`\"}}%
            \def\PYGZti{\discretionary{\char`\~}{\Wrappedafterbreak}{\char`\~}}%
        }
        % Some characters . , ; ? ! / are not pygmentized.
        % This macro makes them "active" and they will insert potential linebreaks
        \newcommand*\Wrappedbreaksatpunct {%
            \lccode`\~`\.\lowercase{\def~}{\discretionary{\hbox{\char`\.}}{\Wrappedafterbreak}{\hbox{\char`\.}}}%
            \lccode`\~`\,\lowercase{\def~}{\discretionary{\hbox{\char`\,}}{\Wrappedafterbreak}{\hbox{\char`\,}}}%
            \lccode`\~`\;\lowercase{\def~}{\discretionary{\hbox{\char`\;}}{\Wrappedafterbreak}{\hbox{\char`\;}}}%
            \lccode`\~`\:\lowercase{\def~}{\discretionary{\hbox{\char`\:}}{\Wrappedafterbreak}{\hbox{\char`\:}}}%
            \lccode`\~`\?\lowercase{\def~}{\discretionary{\hbox{\char`\?}}{\Wrappedafterbreak}{\hbox{\char`\?}}}%
            \lccode`\~`\!\lowercase{\def~}{\discretionary{\hbox{\char`\!}}{\Wrappedafterbreak}{\hbox{\char`\!}}}%
            \lccode`\~`\/\lowercase{\def~}{\discretionary{\hbox{\char`\/}}{\Wrappedafterbreak}{\hbox{\char`\/}}}%
            \catcode`\.\active
            \catcode`\,\active
            \catcode`\;\active
            \catcode`\:\active
            \catcode`\?\active
            \catcode`\!\active
            \catcode`\/\active
            \lccode`\~`\~
        }
    \makeatother

    \let\OriginalVerbatim=\Verbatim
    \makeatletter
    \renewcommand{\Verbatim}[1][1]{%
        %\parskip\z@skip
        \sbox\Wrappedcontinuationbox {\Wrappedcontinuationsymbol}%
        \sbox\Wrappedvisiblespacebox {\FV@SetupFont\Wrappedvisiblespace}%
        \def\FancyVerbFormatLine ##1{\hsize\linewidth
            \vtop{\raggedright\hyphenpenalty\z@\exhyphenpenalty\z@
                \doublehyphendemerits\z@\finalhyphendemerits\z@
                \strut ##1\strut}%
        }%
        % If the linebreak is at a space, the latter will be displayed as visible
        % space at end of first line, and a continuation symbol starts next line.
        % Stretch/shrink are however usually zero for typewriter font.
        \def\FV@Space {%
            \nobreak\hskip\z@ plus\fontdimen3\font minus\fontdimen4\font
            \discretionary{\copy\Wrappedvisiblespacebox}{\Wrappedafterbreak}
            {\kern\fontdimen2\font}%
        }%

        % Allow breaks at special characters using \PYG... macros.
        \Wrappedbreaksatspecials
        % Breaks at punctuation characters . , ; ? ! and / need catcode=\active
        \OriginalVerbatim[#1,codes*=\Wrappedbreaksatpunct]%
    }
    \makeatother

    % Exact colors from NB
    \definecolor{incolor}{HTML}{303F9F}
    \definecolor{outcolor}{HTML}{D84315}
    \definecolor{cellborder}{HTML}{CFCFCF}
    \definecolor{cellbackground}{HTML}{F7F7F7}

    % prompt
    \makeatletter
    \newcommand{\boxspacing}{\kern\kvtcb@left@rule\kern\kvtcb@boxsep}
    \makeatother
    \newcommand{\prompt}[4]{
        {\ttfamily\llap{{\color{#2}[#3]:\hspace{3pt}#4}}\vspace{-\baselineskip}}
    }
    

    
    % Prevent overflowing lines due to hard-to-break entities
    \sloppy
    % Setup hyperref package
    \hypersetup{
      breaklinks=true,  % so long urls are correctly broken across lines
      colorlinks=true,
      urlcolor=urlcolor,
      linkcolor=linkcolor,
      citecolor=citecolor,
      }
    % Slightly bigger margins than the latex defaults
    
    \geometry{verbose,tmargin=1in,bmargin=1in,lmargin=1in,rmargin=1in}
    
    

\begin{document}
    
    \maketitle
    
    

    
    \hypertarget{sistemas-de-uxe1lgebra-computacional}{%
\section{Sistemas de Álgebra
Computacional}\label{sistemas-de-uxe1lgebra-computacional}}

    Los Sistemas de Computación Algebraica (CAS, por sus siglas en inglés)
son herramientas poderosas que utilizan algoritmos y técnicas
computacionales para realizar cálculos simbólicos en matemáticas y
ciencias afines. Esto significa que el computador puede efectuar
operaciones con ecuaciones y fórmulas simbólicamente, es decir,
\(a+b=c\) se interpreta como la suma de variables y no como la suma de
números previamente asignados. Por lo tanto, estos sistemas son capaces
de manipular expresiones algebraicas, resolver ecuaciones, derivar e
integrar funciones, realizar operaciones matriciales, entre otras
tareas, todo de manera simbólica.

En el ámbito de la enseñanza, los CAS facilitan el aprendizaje y la
comprensión de conceptos matemáticos y científicos. Al proporcionar una
plataforma interactiva y dinámica, los estudiantes pueden explorar
conceptos abstractos de manera visual y práctica, lo que les ayuda a
internalizar y aplicar esos conceptos de manera más efectiva. Además,
los CAS permiten a los estudiantes experimentar con diferentes
escenarios y casos, lo que fomenta la experimentación y la resolución de
problemas. En la investigación, son herramientas indispensables para
realizar cálculos complejos y llevar a cabo análisis simbólicos
detallados. Los investigadores pueden utilizar estos sistemas para
explorar nuevas teorías, validar resultados teóricos, resolver
ecuaciones diferenciales y realizar simulaciones, entre otras
aplicaciones. La capacidad de automatizar tareas repetitivas y manipular
expresiones simbólicas de manera eficiente ahorra tiempo y esfuerzo, lo
que permite a los investigadores concentrarse en aspectos más creativos
y analíticos de su trabajo.

Estas herramientas computacionales permiten operar de manera exacta con
sí­mbolos que representan objetos matemáticos tales como: - Números
(Enteros, racionales, reales, complejos\ldots) - Polinómios, Funciones
Racionales, Sistemas de Ecuaciones. - Grupos, Anillos, Algebras . . .

A diferencia de los sistemas tradicionales de computación numérica:

\begin{itemize}
\tightlist
\item
  FORTRAN, Basic, C, C++, Java \(=>\) Precisión fija (Punto Flotante)
\end{itemize}

Otra característica principal radica en el hecho de que son interactivos
(interpretados o ejecutados al momento de proveer una instrucción).

Existe, o han existido, una enorme cantidad de sistemas algebráicos
(Ver:
https://en.wikipedia.org/wiki/List\_of\_computer\_algebra\_systems)
donde destacan: Maple, Mathematica, Matlab que son pagos y los que son
software libre como Maxima y SymPy.

    \hypertarget{sympy}{%
\subsection{SymPy}\label{sympy}}

\textbf{SymPy} es una biblioteca de computación simbólica en Python que
ofrece una amplia gama de herramientas para realizar cálculos
matemáticos y simbólicos. Se destaca por ser gratuita, de código abierto
y altamente flexible, lo que la convierte en una opción popular tanto
para estudiantes como para profesionales en matemáticas, física,
ingeniería y otras disciplinas científicas.

Esta biblioteca permite realizar una variedad de operaciones simbólicas,
como manipulación de expresiones algebraicas, resolución de ecuaciones,
cálculo diferencial e integral, álgebra lineal, optimización, entre
otras. Además, SymPy está diseñada para ser fácilmente extensible, lo
que permite a los usuarios crear y agregar sus propias funciones y
módulos personalizados.

Una de las características más destacadas de SymPy es su integración con
el lenguaje de programación Python. Esto significa que los usuarios
pueden combinar la potencia de SymPy con las capacidades de programación
de Python, lo que facilita la automatización de tareas, la creación de
scripts personalizados y el desarrollo de aplicaciones más complejas.

SymPy es ampliamente utilizado en entornos académicos, de investigación
y de desarrollo, tanto para el aprendizaje y la enseñanza de conceptos
matemáticos como para la resolución de problemas del mundo real. Su
naturaleza de código abierto fomenta la colaboración y el intercambio de
ideas entre la comunidad de usuarios, lo que contribuye a su constante
mejora y evolución.

Por ser una biblioteca de Python, puede ser utilizada en una variedad de
entornos de desarrollo y plataformas que admitan Python. Algunas de las
opciones comunes para usar SymPy incluyen:

Entornos de desarrollo integrado (IDE):

\begin{itemize}
\item
  Visual Studio Code: Visual Studio Code es un popular IDE de código
  abierto que admite Python y proporciona funcionalidades avanzadas para
  el desarrollo de software, incluida la integración con SymPy.
\item
  PyCharm: PyCharm es otro IDE ampliamente utilizado para el desarrollo
  de Python que ofrece características avanzadas como depuración,
  análisis estático y soporte para SymPy.
\end{itemize}

Notebooks de Jupyter

\begin{itemize}
\item
  Jupyter Notebook: Jupyter Notebook es una herramienta poderosa para la
  computación interactiva que permite crear y compartir documentos que
  contienen código ejecutable, visualizaciones y texto explicativo.
  SymPy puede ser utilizado en cuadernos Jupyter para realizar cálculos
  simbólicos interactivos.
\item
  Google Colab: Google Colab es una plataforma de Google que permite
  ejecutar cuadernos Jupyter en la nube de forma gratuita. Proporciona
  acceso gratuito a recursos de cómputo, lo que lo convierte en una
  opción popular para trabajar con SymPy y otros paquetes de Python.
\end{itemize}

Entornos de desarrollo Python:

\begin{itemize}
\item
  Python en línea de comandos: SymPy se puede utilizar directamente en
  el intérprete de Python en la línea de comandos para realizar cálculos
  rápidos y experimentar con la biblioteca.
\item
  Scripts de Python: SymPy también se puede utilizar en scripts de
  Python para automatizar tareas y realizar cálculos en lotes.
  Plataformas en la nube y servicios de alojamiento:
\item
  Plataformas en la nube como AWS, Azure y otros servicios de
  alojamiento de Python ofrecen la posibilidad de ejecutar código
  Python, incluido SymPy, en entornos en la nube escalables y seguros.
\end{itemize}

Este tutorial tiene como objetivo brindar una introducción a SymPy para
usuarios no especializados en el uso de herramientas computacionales. Y
la documentación se encuentra en:
https://docs.sympy.org/latest/index.html

    Primero que todo debemos incorporar, de la enorme cantidad de librerías
que existen para Python, la librería \textbf{\emph{sympy}}

    \begin{tcolorbox}[breakable, size=fbox, boxrule=1pt, pad at break*=1mm,colback=cellbackground, colframe=cellborder]
\prompt{In}{incolor}{1}{\boxspacing}
\begin{Verbatim}[commandchars=\\\{\}]
\PY{c+c1}{\PYZsh{} 6/4/2024 \PYZlt{}= Esta linea es un comentario gracias al \PYZsh{}}
\PY{k+kn}{import} \PY{n+nn}{sympy}
\PY{k+kn}{from} \PY{n+nn}{sympy} \PY{k+kn}{import} \PY{o}{*}
\end{Verbatim}
\end{tcolorbox}

    \begin{tcolorbox}[breakable, size=fbox, boxrule=1pt, pad at break*=1mm,colback=cellbackground, colframe=cellborder]
\prompt{In}{incolor}{2}{\boxspacing}
\begin{Verbatim}[commandchars=\\\{\}]
\PY{n}{\PYZus{}\PYZus{}version\PYZus{}\PYZus{}}  \PY{c+c1}{\PYZsh{} Esto es opcional y permite ver la versión de Sympy que estamos usando}
\end{Verbatim}
\end{tcolorbox}

            \begin{tcolorbox}[breakable, size=fbox, boxrule=.5pt, pad at break*=1mm, opacityfill=0]
\prompt{Out}{outcolor}{2}{\boxspacing}
\begin{Verbatim}[commandchars=\\\{\}]
'1.12'
\end{Verbatim}
\end{tcolorbox}
        
    \hypertarget{sintuxe1xis-buxe1sica}{%
\section{1) Sintáxis básica}\label{sintuxe1xis-buxe1sica}}

    Si queremos calcular: \(3!+2^3-1\) debemos escribir:

    \begin{tcolorbox}[breakable, size=fbox, boxrule=1pt, pad at break*=1mm,colback=cellbackground, colframe=cellborder]
\prompt{In}{incolor}{3}{\boxspacing}
\begin{Verbatim}[commandchars=\\\{\}]
\PY{n}{factorial}\PY{p}{(}\PY{l+m+mi}{3}\PY{p}{)} \PY{o}{+} \PY{l+m+mi}{2}\PY{o}{*}\PY{o}{*}\PY{l+m+mi}{3} \PY{o}{\PYZhy{}} \PY{l+m+mi}{1}
\end{Verbatim}
\end{tcolorbox}
 
            
\prompt{Out}{outcolor}{3}{}
    
    $\displaystyle 13$

    

    El valor de: \(\sqrt{8}\)

    \begin{tcolorbox}[breakable, size=fbox, boxrule=1pt, pad at break*=1mm,colback=cellbackground, colframe=cellborder]
\prompt{In}{incolor}{4}{\boxspacing}
\begin{Verbatim}[commandchars=\\\{\}]
\PY{n}{sqrt}\PY{p}{(}\PY{l+m+mi}{8}\PY{p}{)}
\end{Verbatim}
\end{tcolorbox}
 
            
\prompt{Out}{outcolor}{4}{}
    
    $\displaystyle 2 \sqrt{2}$

    

    Si queremos el valor numérico podemosmos usar \textbf{float}

    \begin{tcolorbox}[breakable, size=fbox, boxrule=1pt, pad at break*=1mm,colback=cellbackground, colframe=cellborder]
\prompt{In}{incolor}{5}{\boxspacing}
\begin{Verbatim}[commandchars=\\\{\}]
\PY{n+nb}{float}\PY{p}{(}\PY{n}{sqrt}\PY{p}{(}\PY{l+m+mi}{8}\PY{p}{)}\PY{p}{)}
\end{Verbatim}
\end{tcolorbox}

            \begin{tcolorbox}[breakable, size=fbox, boxrule=.5pt, pad at break*=1mm, opacityfill=0]
\prompt{Out}{outcolor}{5}{\boxspacing}
\begin{Verbatim}[commandchars=\\\{\}]
2.8284271247461903
\end{Verbatim}
\end{tcolorbox}
        
    Otras variantes

    \begin{tcolorbox}[breakable, size=fbox, boxrule=1pt, pad at break*=1mm,colback=cellbackground, colframe=cellborder]
\prompt{In}{incolor}{6}{\boxspacing}
\begin{Verbatim}[commandchars=\\\{\}]
\PY{n}{N}\PY{p}{(}\PY{n}{sqrt}\PY{p}{(}\PY{l+m+mi}{8}\PY{p}{)}\PY{p}{,}\PY{l+m+mi}{10}\PY{p}{)}
\end{Verbatim}
\end{tcolorbox}
 
            
\prompt{Out}{outcolor}{6}{}
    
    $\displaystyle 2.828427125$

    

    \begin{tcolorbox}[breakable, size=fbox, boxrule=1pt, pad at break*=1mm,colback=cellbackground, colframe=cellborder]
\prompt{In}{incolor}{7}{\boxspacing}
\begin{Verbatim}[commandchars=\\\{\}]
\PY{n}{sqrt}\PY{p}{(}\PY{l+m+mi}{8}\PY{p}{)}\PY{o}{.}\PY{n}{evalf}\PY{p}{(}\PY{l+m+mi}{10}\PY{p}{)}
\end{Verbatim}
\end{tcolorbox}
 
            
\prompt{Out}{outcolor}{7}{}
    
    $\displaystyle 2.828427125$

    

    \begin{tcolorbox}[breakable, size=fbox, boxrule=1pt, pad at break*=1mm,colback=cellbackground, colframe=cellborder]
\prompt{In}{incolor}{8}{\boxspacing}
\begin{Verbatim}[commandchars=\\\{\}]
\PY{n}{N}\PY{p}{(}\PY{n}{sqrt}\PY{p}{(}\PY{l+m+mi}{8}\PY{p}{)}\PY{p}{,}\PY{l+m+mi}{3}\PY{p}{)}
\end{Verbatim}
\end{tcolorbox}
 
            
\prompt{Out}{outcolor}{8}{}
    
    $\displaystyle 2.83$

    

    \begin{tcolorbox}[breakable, size=fbox, boxrule=1pt, pad at break*=1mm,colback=cellbackground, colframe=cellborder]
\prompt{In}{incolor}{9}{\boxspacing}
\begin{Verbatim}[commandchars=\\\{\}]
\PY{n}{pi} \PY{c+c1}{\PYZsh{} La constante π}
\end{Verbatim}
\end{tcolorbox}
 
            
\prompt{Out}{outcolor}{9}{}
    
    $\displaystyle \pi$

    

    \begin{tcolorbox}[breakable, size=fbox, boxrule=1pt, pad at break*=1mm,colback=cellbackground, colframe=cellborder]
\prompt{In}{incolor}{10}{\boxspacing}
\begin{Verbatim}[commandchars=\\\{\}]
\PY{n}{pi}\PY{o}{.}\PY{n}{evalf}\PY{p}{(}\PY{l+m+mi}{50}\PY{p}{)}
\end{Verbatim}
\end{tcolorbox}
 
            
\prompt{Out}{outcolor}{10}{}
    
    $\displaystyle 3.1415926535897932384626433832795028841971693993751$

    

    \begin{tcolorbox}[breakable, size=fbox, boxrule=1pt, pad at break*=1mm,colback=cellbackground, colframe=cellborder]
\prompt{In}{incolor}{11}{\boxspacing}
\begin{Verbatim}[commandchars=\\\{\}]
\PY{n}{ln}\PY{p}{(}\PY{n}{E}\PY{p}{)}
\end{Verbatim}
\end{tcolorbox}
 
            
\prompt{Out}{outcolor}{11}{}
    
    $\displaystyle 1$

    

    \begin{tcolorbox}[breakable, size=fbox, boxrule=1pt, pad at break*=1mm,colback=cellbackground, colframe=cellborder]
\prompt{In}{incolor}{12}{\boxspacing}
\begin{Verbatim}[commandchars=\\\{\}]
\PY{n}{log}\PY{p}{(}\PY{n}{E}\PY{p}{)}
\end{Verbatim}
\end{tcolorbox}
 
            
\prompt{Out}{outcolor}{12}{}
    
    $\displaystyle 1$

    

    \begin{tcolorbox}[breakable, size=fbox, boxrule=1pt, pad at break*=1mm,colback=cellbackground, colframe=cellborder]
\prompt{In}{incolor}{13}{\boxspacing}
\begin{Verbatim}[commandchars=\\\{\}]
\PY{n}{ln}\PY{p}{(}\PY{n}{E}\PY{p}{)}\PY{o}{.}\PY{n}{evalf}\PY{p}{(}\PY{l+m+mi}{2}\PY{p}{)}
\end{Verbatim}
\end{tcolorbox}
 
            
\prompt{Out}{outcolor}{13}{}
    
    $\displaystyle 1.0$

    

    \begin{tcolorbox}[breakable, size=fbox, boxrule=1pt, pad at break*=1mm,colback=cellbackground, colframe=cellborder]
\prompt{In}{incolor}{14}{\boxspacing}
\begin{Verbatim}[commandchars=\\\{\}]
\PY{n}{log}\PY{p}{(}\PY{l+m+mi}{10}\PY{p}{)}
\end{Verbatim}
\end{tcolorbox}
 
            
\prompt{Out}{outcolor}{14}{}
    
    $\displaystyle \log{\left(10 \right)}$

    

    \begin{tcolorbox}[breakable, size=fbox, boxrule=1pt, pad at break*=1mm,colback=cellbackground, colframe=cellborder]
\prompt{In}{incolor}{15}{\boxspacing}
\begin{Verbatim}[commandchars=\\\{\}]
\PY{n}{log}\PY{p}{(}\PY{l+m+mi}{10}\PY{p}{)}\PY{o}{.}\PY{n}{evalf}\PY{p}{(}\PY{l+m+mi}{5}\PY{p}{)}
\end{Verbatim}
\end{tcolorbox}
 
            
\prompt{Out}{outcolor}{15}{}
    
    $\displaystyle 2.3026$

    

    \begin{tcolorbox}[breakable, size=fbox, boxrule=1pt, pad at break*=1mm,colback=cellbackground, colframe=cellborder]
\prompt{In}{incolor}{16}{\boxspacing}
\begin{Verbatim}[commandchars=\\\{\}]
\PY{n}{log}\PY{p}{(}\PY{l+m+mi}{10}\PY{p}{,}\PY{l+m+mi}{10}\PY{p}{)}
\end{Verbatim}
\end{tcolorbox}
 
            
\prompt{Out}{outcolor}{16}{}
    
    $\displaystyle 1$

    

    Consideremos ahora una combinación de operaciones matemáticas

    \[
\frac{\sqrt{8}}{3} + \ln(e+1) + \log_{10}(3) + e^{\pi^2} + (e^{\pi})^2 - 6! \sin\left(\frac{\pi}{3}\right)  + \sqrt{-1}
\]

    \begin{tcolorbox}[breakable, size=fbox, boxrule=1pt, pad at break*=1mm,colback=cellbackground, colframe=cellborder]
\prompt{In}{incolor}{17}{\boxspacing}
\begin{Verbatim}[commandchars=\\\{\}]
\PY{n}{sqrt}\PY{p}{(}\PY{l+m+mi}{8}\PY{p}{)}\PY{o}{/}\PY{l+m+mi}{3}\PY{o}{+}\PY{n}{ln}\PY{p}{(}\PY{n}{E}\PY{o}{+}\PY{l+m+mi}{1}\PY{p}{)}\PY{o}{+}\PY{n}{log}\PY{p}{(}\PY{l+m+mi}{3}\PY{p}{,}\PY{l+m+mi}{10}\PY{p}{)}\PY{o}{+}\PY{n}{E}\PY{o}{*}\PY{o}{*}\PY{p}{(}\PY{n}{pi}\PY{o}{*}\PY{o}{*}\PY{l+m+mi}{2}\PY{p}{)}\PY{o}{+}\PY{n}{exp}\PY{p}{(}\PY{n}{pi}\PY{p}{)}\PY{o}{*}\PY{o}{*}\PY{l+m+mi}{2}\PY{o}{\PYZhy{}}\PY{n}{factorial}\PY{p}{(}\PY{l+m+mi}{6}\PY{p}{)}\PY{o}{*}\PY{n}{sin}\PY{p}{(}\PY{n}{pi}\PY{o}{/}\PY{l+m+mi}{3}\PY{p}{)}\PY{o}{+}\PY{n}{sqrt}\PY{p}{(}\PY{o}{\PYZhy{}}\PY{l+m+mi}{1}\PY{p}{)}
\end{Verbatim}
\end{tcolorbox}
 
            
\prompt{Out}{outcolor}{17}{}
    
    $\displaystyle - 360 \sqrt{3} + \frac{\log{\left(3 \right)}}{\log{\left(10 \right)}} + \frac{2 \sqrt{2}}{3} + \log{\left(1 + e \right)} + e^{2 \pi} + e^{\pi^{2}} + i$

    

    Aquí aprenderemos un atajo. Queremos reutilizar la última salida en el
siguiente comando.

Para hacer esto se utiliza \_ como el argumento del comando. En este
caso el comando que utilizaremos es \textbf{round}, que nos dará el
valor númerico de la expresión anterior con los decimales que
especifiquemos

    \textbf{round}() devuelve un número en coma flotante pero redondeado y
con el número de decimales especificado.

    \begin{tcolorbox}[breakable, size=fbox, boxrule=1pt, pad at break*=1mm,colback=cellbackground, colframe=cellborder]
\prompt{In}{incolor}{18}{\boxspacing}
\begin{Verbatim}[commandchars=\\\{\}]
\PY{n+nb}{round}\PY{p}{(}\PY{n}{\PYZus{}}\PY{p}{,}\PY{l+m+mi}{8}\PY{p}{)}
\end{Verbatim}
\end{tcolorbox}
 
            
\prompt{Out}{outcolor}{18}{}
    
    $\displaystyle 19248.37563115 + i$

    

    Escribamos nuevamente la expresión, pero sin la parte imaginaria, y la
asignaremos a la variable \(x\)

    \begin{tcolorbox}[breakable, size=fbox, boxrule=1pt, pad at break*=1mm,colback=cellbackground, colframe=cellborder]
\prompt{In}{incolor}{19}{\boxspacing}
\begin{Verbatim}[commandchars=\\\{\}]
\PY{n}{x}\PY{o}{=}\PY{n}{sqrt}\PY{p}{(}\PY{l+m+mi}{8}\PY{p}{)}\PY{o}{/}\PY{l+m+mi}{3}\PY{o}{+}\PY{n}{ln}\PY{p}{(}\PY{n}{E}\PY{o}{+}\PY{l+m+mi}{1}\PY{p}{)}\PY{o}{+}\PY{n}{log}\PY{p}{(}\PY{l+m+mi}{3}\PY{p}{,}\PY{l+m+mi}{10}\PY{p}{)}\PY{o}{+}\PY{n}{E}\PY{o}{*}\PY{o}{*}\PY{p}{(}\PY{n}{pi}\PY{o}{*}\PY{o}{*}\PY{l+m+mi}{2}\PY{p}{)}\PY{o}{+}\PY{n}{exp}\PY{p}{(}\PY{n}{pi}\PY{p}{)}\PY{o}{*}\PY{o}{*}\PY{l+m+mi}{2}
\end{Verbatim}
\end{tcolorbox}

    \begin{tcolorbox}[breakable, size=fbox, boxrule=1pt, pad at break*=1mm,colback=cellbackground, colframe=cellborder]
\prompt{In}{incolor}{20}{\boxspacing}
\begin{Verbatim}[commandchars=\\\{\}]
\PY{n}{x}
\end{Verbatim}
\end{tcolorbox}
 
            
\prompt{Out}{outcolor}{20}{}
    
    $\displaystyle \frac{\log{\left(3 \right)}}{\log{\left(10 \right)}} + \frac{2 \sqrt{2}}{3} + \log{\left(1 + e \right)} + e^{2 \pi} + e^{\pi^{2}}$

    

    \begin{tcolorbox}[breakable, size=fbox, boxrule=1pt, pad at break*=1mm,colback=cellbackground, colframe=cellborder]
\prompt{In}{incolor}{21}{\boxspacing}
\begin{Verbatim}[commandchars=\\\{\}]
\PY{n+nb}{float}\PY{p}{(}\PY{n}{\PYZus{}}\PY{p}{)}
\end{Verbatim}
\end{tcolorbox}

            \begin{tcolorbox}[breakable, size=fbox, boxrule=.5pt, pad at break*=1mm, opacityfill=0]
\prompt{Out}{outcolor}{21}{\boxspacing}
\begin{Verbatim}[commandchars=\\\{\}]
19871.91392187373
\end{Verbatim}
\end{tcolorbox}
        
    \begin{tcolorbox}[breakable, size=fbox, boxrule=1pt, pad at break*=1mm,colback=cellbackground, colframe=cellborder]
\prompt{In}{incolor}{22}{\boxspacing}
\begin{Verbatim}[commandchars=\\\{\}]
\PY{n+nb}{round}\PY{p}{(}\PY{n}{x}\PY{p}{,} \PY{l+m+mi}{2}\PY{p}{)}
\end{Verbatim}
\end{tcolorbox}
 
            
\prompt{Out}{outcolor}{22}{}
    
    $\displaystyle 19871.91$

    

    \begin{tcolorbox}[breakable, size=fbox, boxrule=1pt, pad at break*=1mm,colback=cellbackground, colframe=cellborder]
\prompt{In}{incolor}{23}{\boxspacing}
\begin{Verbatim}[commandchars=\\\{\}]
\PY{n}{N}\PY{p}{(}\PY{n}{\PYZus{}}\PY{p}{,}\PY{l+m+mi}{2}\PY{p}{)}
\end{Verbatim}
\end{tcolorbox}
 
            
\prompt{Out}{outcolor}{23}{}
    
    $\displaystyle 2.0 \cdot 10^{4}$

    

    Como pudimos ver, el programa se entiende perfectamente con los números
imaginarios, por ejemplo: \(\sqrt{-1}+2i\)

    \begin{tcolorbox}[breakable, size=fbox, boxrule=1pt, pad at break*=1mm,colback=cellbackground, colframe=cellborder]
\prompt{In}{incolor}{24}{\boxspacing}
\begin{Verbatim}[commandchars=\\\{\}]
\PY{n}{sqrt}\PY{p}{(}\PY{o}{\PYZhy{}}\PY{l+m+mi}{1}\PY{p}{)}\PY{o}{+}\PY{l+m+mi}{2}\PY{o}{*}\PY{n}{I}
\end{Verbatim}
\end{tcolorbox}
 
            
\prompt{Out}{outcolor}{24}{}
    
    $\displaystyle 3 i$

    

    Una de las ecuaciones más bonitas de las matemáticas:

    \(e^{2\pi i} =1\)

    \begin{tcolorbox}[breakable, size=fbox, boxrule=1pt, pad at break*=1mm,colback=cellbackground, colframe=cellborder]
\prompt{In}{incolor}{25}{\boxspacing}
\begin{Verbatim}[commandchars=\\\{\}]
\PY{n}{exp}\PY{p}{(}\PY{l+m+mi}{2}\PY{o}{*}\PY{n}{pi}\PY{o}{*}\PY{n}{I}\PY{p}{)}\PY{o}{\PYZhy{}}\PY{l+m+mi}{1}
\end{Verbatim}
\end{tcolorbox}
 
            
\prompt{Out}{outcolor}{25}{}
    
    $\displaystyle 0$

    

    A diferencia de muchos sistemas de manipulación simbólica, en SymPy las
variables deben definirse antes de usarse.

    \begin{tcolorbox}[breakable, size=fbox, boxrule=1pt, pad at break*=1mm,colback=cellbackground, colframe=cellborder]
\prompt{In}{incolor}{26}{\boxspacing}
\begin{Verbatim}[commandchars=\\\{\}]
\PY{n}{t}\PY{p}{,} \PY{n}{x}\PY{p}{,} \PY{n}{y}\PY{p}{,} \PY{n}{z}\PY{p}{,} \PY{n}{n}\PY{p}{,} \PY{n}{a}\PY{p}{,} \PY{n}{b}\PY{p}{,} \PY{n}{c} \PY{o}{=} \PY{n}{symbols}\PY{p}{(}\PY{l+s+s1}{\PYZsq{}}\PY{l+s+s1}{t x y z n a b c}\PY{l+s+s1}{\PYZsq{}}\PY{p}{)}
\end{Verbatim}
\end{tcolorbox}

    \begin{tcolorbox}[breakable, size=fbox, boxrule=1pt, pad at break*=1mm,colback=cellbackground, colframe=cellborder]
\prompt{In}{incolor}{27}{\boxspacing}
\begin{Verbatim}[commandchars=\\\{\}]
\PY{n}{r} \PY{o}{=} \PY{n}{x} \PY{o}{+} \PY{l+m+mi}{2}\PY{o}{*}\PY{n}{y}
\end{Verbatim}
\end{tcolorbox}

    \begin{tcolorbox}[breakable, size=fbox, boxrule=1pt, pad at break*=1mm,colback=cellbackground, colframe=cellborder]
\prompt{In}{incolor}{28}{\boxspacing}
\begin{Verbatim}[commandchars=\\\{\}]
\PY{n}{r}
\end{Verbatim}
\end{tcolorbox}
 
            
\prompt{Out}{outcolor}{28}{}
    
    $\displaystyle x + 2 y$

    

    \begin{tcolorbox}[breakable, size=fbox, boxrule=1pt, pad at break*=1mm,colback=cellbackground, colframe=cellborder]
\prompt{In}{incolor}{29}{\boxspacing}
\begin{Verbatim}[commandchars=\\\{\}]
\PY{n}{p}\PY{o}{=}\PY{n}{x}\PY{o}{*}\PY{o}{*}\PY{l+m+mi}{2}\PY{o}{*}\PY{n}{r}\PY{o}{*}\PY{o}{*}\PY{l+m+mi}{3}
\PY{n}{p}
\end{Verbatim}
\end{tcolorbox}
 
            
\prompt{Out}{outcolor}{29}{}
    
    $\displaystyle x^{2} \left(x + 2 y\right)^{3}$

    

    Algunas funciones de manipulación simbolica

    \begin{tcolorbox}[breakable, size=fbox, boxrule=1pt, pad at break*=1mm,colback=cellbackground, colframe=cellborder]
\prompt{In}{incolor}{30}{\boxspacing}
\begin{Verbatim}[commandchars=\\\{\}]
\PY{n}{expand}\PY{p}{(}\PY{n}{p}\PY{p}{)}
\end{Verbatim}
\end{tcolorbox}
 
            
\prompt{Out}{outcolor}{30}{}
    
    $\displaystyle x^{5} + 6 x^{4} y + 12 x^{3} y^{2} + 8 x^{2} y^{3}$

    

    Lo inverso de expandir es factorizar

    \begin{tcolorbox}[breakable, size=fbox, boxrule=1pt, pad at break*=1mm,colback=cellbackground, colframe=cellborder]
\prompt{In}{incolor}{31}{\boxspacing}
\begin{Verbatim}[commandchars=\\\{\}]
\PY{n}{factor}\PY{p}{(}\PY{n}{\PYZus{}}\PY{p}{)}
\end{Verbatim}
\end{tcolorbox}
 
            
\prompt{Out}{outcolor}{31}{}
    
    $\displaystyle x^{2} \left(x + 2 y\right)^{3}$

    

    \begin{tcolorbox}[breakable, size=fbox, boxrule=1pt, pad at break*=1mm,colback=cellbackground, colframe=cellborder]
\prompt{In}{incolor}{32}{\boxspacing}
\begin{Verbatim}[commandchars=\\\{\}]
\PY{n}{expand}\PY{p}{(}\PY{n}{\PYZus{}}\PY{p}{)}
\end{Verbatim}
\end{tcolorbox}
 
            
\prompt{Out}{outcolor}{32}{}
    
    $\displaystyle x^{5} + 6 x^{4} y + 12 x^{3} y^{2} + 8 x^{2} y^{3}$

    

    \begin{tcolorbox}[breakable, size=fbox, boxrule=1pt, pad at break*=1mm,colback=cellbackground, colframe=cellborder]
\prompt{In}{incolor}{33}{\boxspacing}
\begin{Verbatim}[commandchars=\\\{\}]
\PY{n}{collect}\PY{p}{(}\PY{n}{\PYZus{}}\PY{p}{,}\PY{n}{x}\PY{o}{*}\PY{n}{y}\PY{p}{)}
\end{Verbatim}
\end{tcolorbox}
 
            
\prompt{Out}{outcolor}{33}{}
    
    $\displaystyle x^{5} + 6 x^{4} y + x^{2} y^{2} \cdot \left(12 x + 8 y\right)$

    

    \begin{tcolorbox}[breakable, size=fbox, boxrule=1pt, pad at break*=1mm,colback=cellbackground, colframe=cellborder]
\prompt{In}{incolor}{34}{\boxspacing}
\begin{Verbatim}[commandchars=\\\{\}]
\PY{n}{simplify}\PY{p}{(}\PY{n}{\PYZus{}}\PY{p}{)}
\end{Verbatim}
\end{tcolorbox}
 
            
\prompt{Out}{outcolor}{34}{}
    
    $\displaystyle x^{2} \left(x^{3} + 6 x^{2} y + y^{2} \cdot \left(12 x + 8 y\right)\right)$

    

    Escribamos el siguiente polinomio

    \begin{tcolorbox}[breakable, size=fbox, boxrule=1pt, pad at break*=1mm,colback=cellbackground, colframe=cellborder]
\prompt{In}{incolor}{35}{\boxspacing}
\begin{Verbatim}[commandchars=\\\{\}]
\PY{n}{P} \PY{o}{=} \PY{p}{(}\PY{n}{x}\PY{o}{\PYZhy{}}\PY{l+m+mi}{1}\PY{p}{)}\PY{o}{*}\PY{p}{(}\PY{n}{x}\PY{o}{\PYZhy{}}\PY{l+m+mi}{2}\PY{p}{)}\PY{o}{*}\PY{p}{(}\PY{n}{x}\PY{o}{\PYZhy{}}\PY{l+m+mi}{3}\PY{p}{)}
\PY{n}{P}
\end{Verbatim}
\end{tcolorbox}
 
            
\prompt{Out}{outcolor}{35}{}
    
    $\displaystyle \left(x - 3\right) \left(x - 2\right) \left(x - 1\right)$

    

    \begin{tcolorbox}[breakable, size=fbox, boxrule=1pt, pad at break*=1mm,colback=cellbackground, colframe=cellborder]
\prompt{In}{incolor}{36}{\boxspacing}
\begin{Verbatim}[commandchars=\\\{\}]
\PY{n}{P}\PY{o}{.}\PY{n}{expand}\PY{p}{(}\PY{p}{)}
\end{Verbatim}
\end{tcolorbox}
 
            
\prompt{Out}{outcolor}{36}{}
    
    $\displaystyle x^{3} - 6 x^{2} + 11 x - 6$

    

    \begin{tcolorbox}[breakable, size=fbox, boxrule=1pt, pad at break*=1mm,colback=cellbackground, colframe=cellborder]
\prompt{In}{incolor}{37}{\boxspacing}
\begin{Verbatim}[commandchars=\\\{\}]
\PY{n}{raices} \PY{o}{=} \PY{n}{solve}\PY{p}{(}\PY{n}{P}\PY{p}{,} \PY{n}{x}\PY{p}{)}
\PY{n}{raices}
\end{Verbatim}
\end{tcolorbox}

            \begin{tcolorbox}[breakable, size=fbox, boxrule=.5pt, pad at break*=1mm, opacityfill=0]
\prompt{Out}{outcolor}{37}{\boxspacing}
\begin{Verbatim}[commandchars=\\\{\}]
[1, 2, 3]
\end{Verbatim}
\end{tcolorbox}
        
    La solución es escrita en el formato de una lista: \([a, b, c, d]\),
donde el primer elemento está etiquetado con cero, el segundo con uno,
el tercero cosconn dos\ldots{}

Por lo tanto, para extraer los elementos de una lista podemos hacer lo
siguiente:

    \begin{tcolorbox}[breakable, size=fbox, boxrule=1pt, pad at break*=1mm,colback=cellbackground, colframe=cellborder]
\prompt{In}{incolor}{38}{\boxspacing}
\begin{Verbatim}[commandchars=\\\{\}]
\PY{n}{raices}\PY{p}{[}\PY{l+m+mi}{0}\PY{p}{]}
\end{Verbatim}
\end{tcolorbox}
 
            
\prompt{Out}{outcolor}{38}{}
    
    $\displaystyle 1$

    

    \begin{tcolorbox}[breakable, size=fbox, boxrule=1pt, pad at break*=1mm,colback=cellbackground, colframe=cellborder]
\prompt{In}{incolor}{39}{\boxspacing}
\begin{Verbatim}[commandchars=\\\{\}]
\PY{n}{raices}\PY{p}{[}\PY{l+m+mi}{0}\PY{p}{]}\PY{o}{+}\PY{n}{raices}\PY{p}{[}\PY{l+m+mi}{1}\PY{p}{]}\PY{o}{+}\PY{n}{raices}\PY{p}{[}\PY{l+m+mi}{2}\PY{p}{]}
\end{Verbatim}
\end{tcolorbox}
 
            
\prompt{Out}{outcolor}{39}{}
    
    $\displaystyle 6$

    

    Por lo tanto: \(P - (x-x_0)(x-x_1)(x-x_2)=0\)

    \begin{tcolorbox}[breakable, size=fbox, boxrule=1pt, pad at break*=1mm,colback=cellbackground, colframe=cellborder]
\prompt{In}{incolor}{40}{\boxspacing}
\begin{Verbatim}[commandchars=\\\{\}]
\PY{n}{simplify}\PY{p}{(}\PY{n}{P} \PY{o}{\PYZhy{}} \PY{p}{(}\PY{n}{x}\PY{o}{\PYZhy{}}\PY{n}{raices}\PY{p}{[}\PY{l+m+mi}{0}\PY{p}{]}\PY{p}{)}\PY{o}{*}\PY{p}{(}\PY{n}{x}\PY{o}{\PYZhy{}}\PY{n}{raices}\PY{p}{[}\PY{l+m+mi}{1}\PY{p}{]}\PY{p}{)}\PY{o}{*}\PY{p}{(}\PY{n}{x}\PY{o}{\PYZhy{}}\PY{n}{raices}\PY{p}{[}\PY{l+m+mi}{2}\PY{p}{]}\PY{p}{)} \PY{p}{)}
\end{Verbatim}
\end{tcolorbox}
 
            
\prompt{Out}{outcolor}{40}{}
    
    $\displaystyle 0$

    

    Podemos simplificar expresiones trigonométricas

    \begin{tcolorbox}[breakable, size=fbox, boxrule=1pt, pad at break*=1mm,colback=cellbackground, colframe=cellborder]
\prompt{In}{incolor}{41}{\boxspacing}
\begin{Verbatim}[commandchars=\\\{\}]
\PY{n}{simplify}\PY{p}{(} \PY{n}{sin}\PY{p}{(}\PY{n}{x}\PY{p}{)}\PY{o}{*}\PY{n}{cos}\PY{p}{(}\PY{n}{y}\PY{p}{)}\PY{o}{+}\PY{n}{cos}\PY{p}{(}\PY{n}{x}\PY{p}{)}\PY{o}{*}\PY{n}{sin}\PY{p}{(}\PY{n}{y}\PY{p}{)} \PY{p}{)}
\end{Verbatim}
\end{tcolorbox}
 
            
\prompt{Out}{outcolor}{41}{}
    
    $\displaystyle \sin{\left(x + y \right)}$

    

    \begin{tcolorbox}[breakable, size=fbox, boxrule=1pt, pad at break*=1mm,colback=cellbackground, colframe=cellborder]
\prompt{In}{incolor}{42}{\boxspacing}
\begin{Verbatim}[commandchars=\\\{\}]
\PY{n}{expr} \PY{o}{=} \PY{l+m+mi}{2}\PY{o}{*}\PY{n}{sin}\PY{p}{(}\PY{n}{x}\PY{p}{)}\PY{o}{*}\PY{o}{*}\PY{l+m+mi}{2}\PY{o}{+}\PY{l+m+mi}{2}\PY{o}{*}\PY{n}{cos}\PY{p}{(}\PY{n}{x}\PY{p}{)}\PY{o}{*}\PY{o}{*}\PY{l+m+mi}{2}
\PY{n}{expr}
\end{Verbatim}
\end{tcolorbox}
 
            
\prompt{Out}{outcolor}{42}{}
    
    $\displaystyle 2 \sin^{2}{\left(x \right)} + 2 \cos^{2}{\left(x \right)}$

    

    \begin{tcolorbox}[breakable, size=fbox, boxrule=1pt, pad at break*=1mm,colback=cellbackground, colframe=cellborder]
\prompt{In}{incolor}{43}{\boxspacing}
\begin{Verbatim}[commandchars=\\\{\}]
\PY{n}{trigsimp}\PY{p}{(}\PY{n}{expr}\PY{p}{)}
\end{Verbatim}
\end{tcolorbox}
 
            
\prompt{Out}{outcolor}{43}{}
    
    $\displaystyle 2$

    

    \begin{tcolorbox}[breakable, size=fbox, boxrule=1pt, pad at break*=1mm,colback=cellbackground, colframe=cellborder]
\prompt{In}{incolor}{44}{\boxspacing}
\begin{Verbatim}[commandchars=\\\{\}]
\PY{p}{(}\PY{n}{sin}\PY{p}{(}\PY{n}{x}\PY{p}{)}\PY{o}{*}\PY{o}{*}\PY{l+m+mi}{4}\PY{o}{\PYZhy{}}\PY{l+m+mi}{2}\PY{o}{*}\PY{n}{cos}\PY{p}{(}\PY{n}{x}\PY{p}{)}\PY{o}{*}\PY{o}{*}\PY{l+m+mi}{2}\PY{o}{*}\PY{n}{sin}\PY{p}{(}\PY{n}{x}\PY{p}{)}\PY{o}{*}\PY{o}{*}\PY{l+m+mi}{2}\PY{o}{+}\PY{n}{cos}\PY{p}{(}\PY{n}{x}\PY{p}{)}\PY{o}{*}\PY{o}{*}\PY{l+m+mi}{4}\PY{p}{)}\PY{o}{.}\PY{n}{simplify}\PY{p}{(}\PY{p}{)}
\end{Verbatim}
\end{tcolorbox}
 
            
\prompt{Out}{outcolor}{44}{}
    
    $\displaystyle \frac{\cos{\left(4 x \right)}}{2} + \frac{1}{2}$

    

    \begin{tcolorbox}[breakable, size=fbox, boxrule=1pt, pad at break*=1mm,colback=cellbackground, colframe=cellborder]
\prompt{In}{incolor}{45}{\boxspacing}
\begin{Verbatim}[commandchars=\\\{\}]
\PY{n}{expand\PYZus{}trig}\PY{p}{(}\PY{n}{sin}\PY{p}{(}\PY{l+m+mi}{2}\PY{o}{*}\PY{n}{x}\PY{p}{)}\PY{p}{)}
\end{Verbatim}
\end{tcolorbox}
 
            
\prompt{Out}{outcolor}{45}{}
    
    $\displaystyle 2 \sin{\left(x \right)} \cos{\left(x \right)}$

    

    \hypertarget{cuxe1lculos-elementales}{%
\section{2) Cálculos elementales}\label{cuxe1lculos-elementales}}

    Vamos a ver una forma de trabajar con funciones como una asignación a
una variable. Luego mostraremos otras posibilidades

    \begin{tcolorbox}[breakable, size=fbox, boxrule=1pt, pad at break*=1mm,colback=cellbackground, colframe=cellborder]
\prompt{In}{incolor}{46}{\boxspacing}
\begin{Verbatim}[commandchars=\\\{\}]
\PY{n}{f}\PY{o}{=}\PY{n}{sin}\PY{p}{(}\PY{n}{x}\PY{p}{)}\PY{o}{/}\PY{n}{exp}\PY{p}{(}\PY{n}{x}\PY{p}{)}
\end{Verbatim}
\end{tcolorbox}

    \begin{tcolorbox}[breakable, size=fbox, boxrule=1pt, pad at break*=1mm,colback=cellbackground, colframe=cellborder]
\prompt{In}{incolor}{47}{\boxspacing}
\begin{Verbatim}[commandchars=\\\{\}]
\PY{n}{f}
\end{Verbatim}
\end{tcolorbox}
 
            
\prompt{Out}{outcolor}{47}{}
    
    $\displaystyle e^{- x} \sin{\left(x \right)}$

    

    La primera derivada

    \begin{tcolorbox}[breakable, size=fbox, boxrule=1pt, pad at break*=1mm,colback=cellbackground, colframe=cellborder]
\prompt{In}{incolor}{48}{\boxspacing}
\begin{Verbatim}[commandchars=\\\{\}]
\PY{n}{df}\PY{o}{=}\PY{n}{diff}\PY{p}{(}\PY{n}{f}\PY{p}{,}\PY{n}{x}\PY{p}{)}
\end{Verbatim}
\end{tcolorbox}

    \begin{tcolorbox}[breakable, size=fbox, boxrule=1pt, pad at break*=1mm,colback=cellbackground, colframe=cellborder]
\prompt{In}{incolor}{49}{\boxspacing}
\begin{Verbatim}[commandchars=\\\{\}]
\PY{n}{df}
\end{Verbatim}
\end{tcolorbox}
 
            
\prompt{Out}{outcolor}{49}{}
    
    $\displaystyle - e^{- x} \sin{\left(x \right)} + e^{- x} \cos{\left(x \right)}$

    

    Otra manera de hacer lo mismo es

    \begin{tcolorbox}[breakable, size=fbox, boxrule=1pt, pad at break*=1mm,colback=cellbackground, colframe=cellborder]
\prompt{In}{incolor}{50}{\boxspacing}
\begin{Verbatim}[commandchars=\\\{\}]
\PY{n}{df}\PY{o}{=}\PY{n}{f}\PY{o}{.}\PY{n}{diff}\PY{p}{(}\PY{n}{x}\PY{p}{)}
\PY{n}{df}
\end{Verbatim}
\end{tcolorbox}
 
            
\prompt{Out}{outcolor}{50}{}
    
    $\displaystyle - e^{- x} \sin{\left(x \right)} + e^{- x} \cos{\left(x \right)}$

    

    Simplificamos

    \begin{tcolorbox}[breakable, size=fbox, boxrule=1pt, pad at break*=1mm,colback=cellbackground, colframe=cellborder]
\prompt{In}{incolor}{51}{\boxspacing}
\begin{Verbatim}[commandchars=\\\{\}]
\PY{n}{df}\PY{o}{.}\PY{n}{simplify}\PY{p}{(}\PY{p}{)}
\end{Verbatim}
\end{tcolorbox}
 
            
\prompt{Out}{outcolor}{51}{}
    
    $\displaystyle \sqrt{2} e^{- x} \cos{\left(x + \frac{\pi}{4} \right)}$

    

    La integral

    \begin{tcolorbox}[breakable, size=fbox, boxrule=1pt, pad at break*=1mm,colback=cellbackground, colframe=cellborder]
\prompt{In}{incolor}{52}{\boxspacing}
\begin{Verbatim}[commandchars=\\\{\}]
\PY{n}{integrate}\PY{p}{(}\PY{n}{df}\PY{p}{,}\PY{n}{x}\PY{p}{)}
\end{Verbatim}
\end{tcolorbox}
 
            
\prompt{Out}{outcolor}{52}{}
    
    $\displaystyle e^{- x} \sin{\left(x \right)}$

    

    Derivadas de orden superior

    \begin{tcolorbox}[breakable, size=fbox, boxrule=1pt, pad at break*=1mm,colback=cellbackground, colframe=cellborder]
\prompt{In}{incolor}{53}{\boxspacing}
\begin{Verbatim}[commandchars=\\\{\}]
\PY{n}{diff}\PY{p}{(}\PY{n}{f}\PY{p}{,}\PY{n}{x}\PY{p}{,}\PY{l+m+mi}{5}\PY{p}{)}
\end{Verbatim}
\end{tcolorbox}
 
            
\prompt{Out}{outcolor}{53}{}
    
    $\displaystyle 4 \left(\sin{\left(x \right)} - \cos{\left(x \right)}\right) e^{- x}$

    

    Para evaluar la funcíon en un punto

    \begin{tcolorbox}[breakable, size=fbox, boxrule=1pt, pad at break*=1mm,colback=cellbackground, colframe=cellborder]
\prompt{In}{incolor}{54}{\boxspacing}
\begin{Verbatim}[commandchars=\\\{\}]
\PY{n}{f}\PY{o}{.}\PY{n}{subs}\PY{p}{(}\PY{n}{x}\PY{p}{,} \PY{n}{pi}\PY{o}{/}\PY{l+m+mi}{2}\PY{p}{)}
\end{Verbatim}
\end{tcolorbox}
 
            
\prompt{Out}{outcolor}{54}{}
    
    $\displaystyle e^{- \frac{\pi}{2}}$

    

    Para funciones de varias variables

    \begin{tcolorbox}[breakable, size=fbox, boxrule=1pt, pad at break*=1mm,colback=cellbackground, colframe=cellborder]
\prompt{In}{incolor}{55}{\boxspacing}
\begin{Verbatim}[commandchars=\\\{\}]
\PY{n}{f}\PY{o}{=}\PY{n}{exp}\PY{p}{(}\PY{n}{x}\PY{o}{*}\PY{o}{*}\PY{l+m+mi}{2}\PY{o}{+}\PY{n}{y}\PY{o}{*}\PY{o}{*}\PY{l+m+mi}{2}\PY{p}{)}\PY{o}{/}\PY{p}{(}\PY{n}{x}\PY{o}{\PYZhy{}}\PY{n}{y}\PY{p}{)}
\PY{n}{f}
\end{Verbatim}
\end{tcolorbox}
 
            
\prompt{Out}{outcolor}{55}{}
    
    $\displaystyle \frac{e^{x^{2} + y^{2}}}{x - y}$

    

    La derivadas cruzadas:

    \begin{tcolorbox}[breakable, size=fbox, boxrule=1pt, pad at break*=1mm,colback=cellbackground, colframe=cellborder]
\prompt{In}{incolor}{56}{\boxspacing}
\begin{Verbatim}[commandchars=\\\{\}]
\PY{n}{diff}\PY{p}{(}\PY{n}{diff}\PY{p}{(}\PY{n}{f}\PY{p}{,}\PY{n}{y}\PY{p}{)}\PY{p}{,}\PY{n}{x}\PY{p}{)}\PY{o}{.}\PY{n}{factor}\PY{p}{(}\PY{p}{)}
\end{Verbatim}
\end{tcolorbox}
 
            
\prompt{Out}{outcolor}{56}{}
    
    $\displaystyle \frac{2 \cdot \left(2 x^{3} y - 4 x^{2} y^{2} + x^{2} + 2 x y^{3} - 2 x y + y^{2} - 1\right) e^{x^{2}} e^{y^{2}}}{\left(x - y\right)^{3}}$

    

    \begin{tcolorbox}[breakable, size=fbox, boxrule=1pt, pad at break*=1mm,colback=cellbackground, colframe=cellborder]
\prompt{In}{incolor}{57}{\boxspacing}
\begin{Verbatim}[commandchars=\\\{\}]
\PY{p}{(}\PY{n}{diff}\PY{p}{(}\PY{n}{f}\PY{p}{,}\PY{n}{x}\PY{p}{)} \PY{o}{+} \PY{n}{diff}\PY{p}{(}\PY{n}{f}\PY{p}{,}\PY{n}{y}\PY{p}{)}\PY{p}{)}\PY{o}{.}\PY{n}{factor}\PY{p}{(}\PY{p}{)}
\end{Verbatim}
\end{tcolorbox}
 
            
\prompt{Out}{outcolor}{57}{}
    
    $\displaystyle \frac{2 \left(x + y\right) e^{x^{2}} e^{y^{2}}}{x - y}$

    

    Las derivadas también pueden ejecutarse de la siguiente manera

    \begin{tcolorbox}[breakable, size=fbox, boxrule=1pt, pad at break*=1mm,colback=cellbackground, colframe=cellborder]
\prompt{In}{incolor}{58}{\boxspacing}
\begin{Verbatim}[commandchars=\\\{\}]
\PY{n}{f}\PY{o}{.}\PY{n}{diff}\PY{p}{(}\PY{n}{x}\PY{p}{)} \PY{o}{+} \PY{n}{f}\PY{o}{.}\PY{n}{diff}\PY{p}{(}\PY{n}{y}\PY{p}{)}
\end{Verbatim}
\end{tcolorbox}
 
            
\prompt{Out}{outcolor}{58}{}
    
    $\displaystyle \frac{2 x e^{x^{2} + y^{2}}}{x - y} + \frac{2 y e^{x^{2} + y^{2}}}{x - y}$

    

    \begin{tcolorbox}[breakable, size=fbox, boxrule=1pt, pad at break*=1mm,colback=cellbackground, colframe=cellborder]
\prompt{In}{incolor}{59}{\boxspacing}
\begin{Verbatim}[commandchars=\\\{\}]
\PY{n}{factor}\PY{p}{(}\PY{n}{\PYZus{}}\PY{p}{)}
\end{Verbatim}
\end{tcolorbox}
 
            
\prompt{Out}{outcolor}{59}{}
    
    $\displaystyle \frac{2 \left(x + y\right) e^{x^{2}} e^{y^{2}}}{x - y}$

    

    Si las queremos dejar indicadas las operaciones matemáticas:

    \begin{tcolorbox}[breakable, size=fbox, boxrule=1pt, pad at break*=1mm,colback=cellbackground, colframe=cellborder]
\prompt{In}{incolor}{60}{\boxspacing}
\begin{Verbatim}[commandchars=\\\{\}]
\PY{n}{Derivative}\PY{p}{(}\PY{n}{f}\PY{p}{,}\PY{n}{x}\PY{p}{)}\PY{o}{+}\PY{n}{Derivative}\PY{p}{(}\PY{n}{f}\PY{p}{,}\PY{n}{y}\PY{p}{)}
\end{Verbatim}
\end{tcolorbox}
 
            
\prompt{Out}{outcolor}{60}{}
    
    $\displaystyle \frac{\partial}{\partial x} \frac{e^{x^{2} + y^{2}}}{x - y} + \frac{\partial}{\partial y} \frac{e^{x^{2} + y^{2}}}{x - y}$

    

    Y luego le pedimos al programa que ejecute la operación indicada con
anterioridad

    \begin{tcolorbox}[breakable, size=fbox, boxrule=1pt, pad at break*=1mm,colback=cellbackground, colframe=cellborder]
\prompt{In}{incolor}{61}{\boxspacing}
\begin{Verbatim}[commandchars=\\\{\}]
\PY{p}{(}\PY{n}{\PYZus{}}\PY{p}{)}\PY{o}{.}\PY{n}{doit}\PY{p}{(}\PY{p}{)}
\end{Verbatim}
\end{tcolorbox}
 
            
\prompt{Out}{outcolor}{61}{}
    
    $\displaystyle \frac{2 x e^{x^{2} + y^{2}}}{x - y} + \frac{2 y e^{x^{2} + y^{2}}}{x - y}$

    

    \begin{tcolorbox}[breakable, size=fbox, boxrule=1pt, pad at break*=1mm,colback=cellbackground, colframe=cellborder]
\prompt{In}{incolor}{62}{\boxspacing}
\begin{Verbatim}[commandchars=\\\{\}]
\PY{n}{Integral}\PY{p}{(}\PY{n}{f}\PY{p}{,}\PY{n}{x}\PY{p}{)}
\end{Verbatim}
\end{tcolorbox}
 
            
\prompt{Out}{outcolor}{62}{}
    
    $\displaystyle \int \frac{e^{x^{2} + y^{2}}}{x - y}\, dx$

    

    Para asignarle valores a las variables

    \begin{tcolorbox}[breakable, size=fbox, boxrule=1pt, pad at break*=1mm,colback=cellbackground, colframe=cellborder]
\prompt{In}{incolor}{63}{\boxspacing}
\begin{Verbatim}[commandchars=\\\{\}]
\PY{n}{f}\PY{o}{.}\PY{n}{subs}\PY{p}{(}\PY{p}{[}\PY{p}{(}\PY{n}{x}\PY{p}{,} \PY{l+m+mi}{2}\PY{p}{)}\PY{p}{,} \PY{p}{(}\PY{n}{y}\PY{p}{,} \PY{l+m+mi}{1}\PY{p}{)}\PY{p}{]}\PY{p}{)}
\end{Verbatim}
\end{tcolorbox}
 
            
\prompt{Out}{outcolor}{63}{}
    
    $\displaystyle e^{5}$

    

    \begin{tcolorbox}[breakable, size=fbox, boxrule=1pt, pad at break*=1mm,colback=cellbackground, colframe=cellborder]
\prompt{In}{incolor}{64}{\boxspacing}
\begin{Verbatim}[commandchars=\\\{\}]
\PY{n}{f}\PY{o}{.}\PY{n}{evalf}\PY{p}{(}\PY{n}{subs}\PY{o}{=}\PY{p}{\PYZob{}}\PY{n}{x}\PY{p}{:}\PY{l+m+mi}{2}\PY{p}{,}\PY{n}{y}\PY{p}{:}\PY{l+m+mi}{1}\PY{p}{\PYZcb{}}\PY{p}{)}
\end{Verbatim}
\end{tcolorbox}
 
            
\prompt{Out}{outcolor}{64}{}
    
    $\displaystyle 148.413159102577$

    

    Consideremos otra función y varios cálculos con ella.

    \begin{tcolorbox}[breakable, size=fbox, boxrule=1pt, pad at break*=1mm,colback=cellbackground, colframe=cellborder]
\prompt{In}{incolor}{65}{\boxspacing}
\begin{Verbatim}[commandchars=\\\{\}]
\PY{n}{σ} \PY{o}{=}\PY{l+m+mi}{2}\PY{o}{*}\PY{n}{x}\PY{o}{/}\PY{n}{sqrt}\PY{p}{(}\PY{n}{x}\PY{o}{*}\PY{o}{*}\PY{l+m+mi}{2}\PY{o}{+}\PY{l+m+mi}{1}\PY{p}{)}
\PY{n}{σ}
\end{Verbatim}
\end{tcolorbox}
 
            
\prompt{Out}{outcolor}{65}{}
    
    $\displaystyle \frac{2 x}{\sqrt{x^{2} + 1}}$

    

    Calculamos la primera derivada y la asignaremos a la variable
\(d\sigma\)

    \begin{tcolorbox}[breakable, size=fbox, boxrule=1pt, pad at break*=1mm,colback=cellbackground, colframe=cellborder]
\prompt{In}{incolor}{66}{\boxspacing}
\begin{Verbatim}[commandchars=\\\{\}]
\PY{n}{dσ}\PY{o}{=} \PY{n}{σ}\PY{o}{.}\PY{n}{diff}\PY{p}{(}\PY{n}{x}\PY{p}{)}\PY{o}{.}\PY{n}{factor}\PY{p}{(}\PY{p}{)}
\PY{n}{dσ}
\end{Verbatim}
\end{tcolorbox}
 
            
\prompt{Out}{outcolor}{66}{}
    
    $\displaystyle \frac{2}{\left(x^{2} + 1\right)^{\frac{3}{2}}}$

    

    La derivada cuarta y al mismo tiempo que se factorice

    \begin{tcolorbox}[breakable, size=fbox, boxrule=1pt, pad at break*=1mm,colback=cellbackground, colframe=cellborder]
\prompt{In}{incolor}{67}{\boxspacing}
\begin{Verbatim}[commandchars=\\\{\}]
\PY{n}{σ}\PY{o}{.}\PY{n}{diff}\PY{p}{(}\PY{n}{x}\PY{p}{,}\PY{l+m+mi}{4}\PY{p}{)}\PY{o}{.}\PY{n}{factor}\PY{p}{(}\PY{p}{)}
\end{Verbatim}
\end{tcolorbox}
 
            
\prompt{Out}{outcolor}{67}{}
    
    $\displaystyle - \frac{30 x \left(4 x^{2} - 3\right)}{\left(x^{2} + 1\right)^{\frac{9}{2}}}$

    

    \begin{tcolorbox}[breakable, size=fbox, boxrule=1pt, pad at break*=1mm,colback=cellbackground, colframe=cellborder]
\prompt{In}{incolor}{68}{\boxspacing}
\begin{Verbatim}[commandchars=\\\{\}]
\PY{n}{integrate}\PY{p}{(}\PY{n}{σ} \PY{p}{,}\PY{n}{x}\PY{p}{)}
\end{Verbatim}
\end{tcolorbox}
 
            
\prompt{Out}{outcolor}{68}{}
    
    $\displaystyle 2 \sqrt{x^{2} + 1}$

    

    La integral definida

    \$ \int\_a\^{}b \sigma dx \$

    \begin{tcolorbox}[breakable, size=fbox, boxrule=1pt, pad at break*=1mm,colback=cellbackground, colframe=cellborder]
\prompt{In}{incolor}{69}{\boxspacing}
\begin{Verbatim}[commandchars=\\\{\}]
\PY{n}{integrate}\PY{p}{(}\PY{n}{σ} \PY{p}{,}\PY{p}{[}\PY{n}{x}\PY{p}{,}\PY{n}{a}\PY{p}{,}\PY{n}{b}\PY{p}{]}\PY{p}{)}\PY{o}{.}\PY{n}{factor}\PY{p}{(}\PY{p}{)}
\end{Verbatim}
\end{tcolorbox}
 
            
\prompt{Out}{outcolor}{69}{}
    
    $\displaystyle - 2 \left(\sqrt{a^{2} + 1} - \sqrt{b^{2} + 1}\right)$

    

    \$\lim\_\{x\rightarrow 1/2\} \sigma \$

    \begin{tcolorbox}[breakable, size=fbox, boxrule=1pt, pad at break*=1mm,colback=cellbackground, colframe=cellborder]
\prompt{In}{incolor}{70}{\boxspacing}
\begin{Verbatim}[commandchars=\\\{\}]
\PY{n}{limit}\PY{p}{(}\PY{n}{σ} \PY{p}{,}\PY{n}{x}\PY{p}{,} \PY{l+m+mi}{1}\PY{o}{/}\PY{l+m+mi}{2} \PY{p}{)}
\end{Verbatim}
\end{tcolorbox}
 
            
\prompt{Out}{outcolor}{70}{}
    
    $\displaystyle 0.894427190999916$

    

    SymPy no nos devolvió un resultado simbólico sino un número punto
flotante. Esto es porque SymPy define tres tipos de números: Real,
Racional y Entero.

Para SymPy el \(\frac12\) significa \(0,5\). Para evitar que haga esta
interpretación podemos escribir

    \begin{tcolorbox}[breakable, size=fbox, boxrule=1pt, pad at break*=1mm,colback=cellbackground, colframe=cellborder]
\prompt{In}{incolor}{71}{\boxspacing}
\begin{Verbatim}[commandchars=\\\{\}]
\PY{n}{Rational}\PY{p}{(}\PY{l+m+mi}{1}\PY{p}{,}\PY{l+m+mi}{2}\PY{p}{)}
\end{Verbatim}
\end{tcolorbox}
 
            
\prompt{Out}{outcolor}{71}{}
    
    $\displaystyle \frac{1}{2}$

    

    De manera equivalente, también se puede escribir:

    \begin{tcolorbox}[breakable, size=fbox, boxrule=1pt, pad at break*=1mm,colback=cellbackground, colframe=cellborder]
\prompt{In}{incolor}{72}{\boxspacing}
\begin{Verbatim}[commandchars=\\\{\}]
\PY{n}{S}\PY{p}{(}\PY{l+m+mi}{1}\PY{p}{)}\PY{o}{/}\PY{l+m+mi}{2}
\end{Verbatim}
\end{tcolorbox}
 
            
\prompt{Out}{outcolor}{72}{}
    
    $\displaystyle \frac{1}{2}$

    

    Por lo tanto:

    \begin{tcolorbox}[breakable, size=fbox, boxrule=1pt, pad at break*=1mm,colback=cellbackground, colframe=cellborder]
\prompt{In}{incolor}{73}{\boxspacing}
\begin{Verbatim}[commandchars=\\\{\}]
\PY{n}{limit}\PY{p}{(}\PY{n}{σ} \PY{p}{,}\PY{n}{x}\PY{p}{,} \PY{n}{S}\PY{p}{(}\PY{l+m+mi}{1}\PY{p}{)}\PY{o}{/}\PY{l+m+mi}{2} \PY{p}{)}
\end{Verbatim}
\end{tcolorbox}
 
            
\prompt{Out}{outcolor}{73}{}
    
    $\displaystyle \frac{2 \sqrt{5}}{5}$

    

    Los límites por la izquierda y por la derecha

    \begin{tcolorbox}[breakable, size=fbox, boxrule=1pt, pad at break*=1mm,colback=cellbackground, colframe=cellborder]
\prompt{In}{incolor}{74}{\boxspacing}
\begin{Verbatim}[commandchars=\\\{\}]
\PY{n}{limit}\PY{p}{(}\PY{n}{σ} \PY{p}{,}\PY{n}{x}\PY{p}{,} \PY{l+m+mi}{1} \PY{p}{,} \PY{n+nb}{dir}\PY{o}{=}\PY{l+s+s1}{\PYZsq{}}\PY{l+s+s1}{+}\PY{l+s+s1}{\PYZsq{}}\PY{p}{)}
\end{Verbatim}
\end{tcolorbox}
 
            
\prompt{Out}{outcolor}{74}{}
    
    $\displaystyle \sqrt{2}$

    

    \begin{tcolorbox}[breakable, size=fbox, boxrule=1pt, pad at break*=1mm,colback=cellbackground, colframe=cellborder]
\prompt{In}{incolor}{75}{\boxspacing}
\begin{Verbatim}[commandchars=\\\{\}]
\PY{n}{limit}\PY{p}{(}\PY{n}{σ} \PY{p}{,}\PY{n}{x}\PY{p}{,} \PY{l+m+mi}{1} \PY{p}{,} \PY{n+nb}{dir}\PY{o}{=}\PY{l+s+s1}{\PYZsq{}}\PY{l+s+s1}{\PYZhy{}}\PY{l+s+s1}{\PYZsq{}}\PY{p}{)}
\end{Verbatim}
\end{tcolorbox}
 
            
\prompt{Out}{outcolor}{75}{}
    
    $\displaystyle \sqrt{2}$

    

    \begin{tcolorbox}[breakable, size=fbox, boxrule=1pt, pad at break*=1mm,colback=cellbackground, colframe=cellborder]
\prompt{In}{incolor}{76}{\boxspacing}
\begin{Verbatim}[commandchars=\\\{\}]
\PY{n}{Limit}\PY{p}{(}\PY{n}{σ} \PY{p}{,}\PY{n}{x}\PY{p}{,} \PY{n}{oo}\PY{p}{)}
\end{Verbatim}
\end{tcolorbox}
 
            
\prompt{Out}{outcolor}{76}{}
    
    $\displaystyle \lim_{x \to \infty}\left(\frac{2 x}{\sqrt{x^{2} + 1}}\right)$

    

    \begin{tcolorbox}[breakable, size=fbox, boxrule=1pt, pad at break*=1mm,colback=cellbackground, colframe=cellborder]
\prompt{In}{incolor}{77}{\boxspacing}
\begin{Verbatim}[commandchars=\\\{\}]
\PY{p}{(}\PY{n}{\PYZus{}}\PY{p}{)}\PY{o}{.}\PY{n}{doit}\PY{p}{(}\PY{p}{)}
\end{Verbatim}
\end{tcolorbox}
 
            
\prompt{Out}{outcolor}{77}{}
    
    $\displaystyle 2$

    

    Para las series de Taylor al rededor de \(x=0\)

    \begin{tcolorbox}[breakable, size=fbox, boxrule=1pt, pad at break*=1mm,colback=cellbackground, colframe=cellborder]
\prompt{In}{incolor}{78}{\boxspacing}
\begin{Verbatim}[commandchars=\\\{\}]
\PY{n}{series}\PY{p}{(}\PY{n}{σ} \PY{p}{,}\PY{n}{x}\PY{p}{,}\PY{l+m+mi}{0}\PY{p}{,}\PY{l+m+mi}{8}\PY{p}{)}
\end{Verbatim}
\end{tcolorbox}
 
            
\prompt{Out}{outcolor}{78}{}
    
    $\displaystyle 2 x - x^{3} + \frac{3 x^{5}}{4} - \frac{5 x^{7}}{8} + O\left(x^{8}\right)$

    

    Al rededor de \(x=4\)

    \begin{tcolorbox}[breakable, size=fbox, boxrule=1pt, pad at break*=1mm,colback=cellbackground, colframe=cellborder]
\prompt{In}{incolor}{79}{\boxspacing}
\begin{Verbatim}[commandchars=\\\{\}]
\PY{n}{series}\PY{p}{(}\PY{n}{σ} \PY{p}{,}\PY{n}{x}\PY{p}{,}\PY{l+m+mi}{4}\PY{p}{,}\PY{l+m+mi}{3}\PY{p}{)}
\end{Verbatim}
\end{tcolorbox}
 
            
\prompt{Out}{outcolor}{79}{}
    
    $\displaystyle \frac{8 \sqrt{17}}{17} + \frac{2 \sqrt{17} \left(x - 4\right)}{289} - \frac{12 \sqrt{17} \left(x - 4\right)^{2}}{4913} + O\left(\left(x - 4\right)^{3}; x\rightarrow 4\right)$

    

    Series de Taylor de funciones elementales

    \begin{tcolorbox}[breakable, size=fbox, boxrule=1pt, pad at break*=1mm,colback=cellbackground, colframe=cellborder]
\prompt{In}{incolor}{80}{\boxspacing}
\begin{Verbatim}[commandchars=\\\{\}]
\PY{n}{series}\PY{p}{(}\PY{n}{sin}\PY{p}{(}\PY{n}{x}\PY{p}{)}\PY{p}{,} \PY{n}{x}\PY{p}{,} \PY{l+m+mi}{0}\PY{p}{,} \PY{l+m+mi}{8}\PY{p}{)}
\end{Verbatim}
\end{tcolorbox}
 
            
\prompt{Out}{outcolor}{80}{}
    
    $\displaystyle x - \frac{x^{3}}{6} + \frac{x^{5}}{120} - \frac{x^{7}}{5040} + O\left(x^{8}\right)$

    

    \begin{tcolorbox}[breakable, size=fbox, boxrule=1pt, pad at break*=1mm,colback=cellbackground, colframe=cellborder]
\prompt{In}{incolor}{81}{\boxspacing}
\begin{Verbatim}[commandchars=\\\{\}]
\PY{n}{series}\PY{p}{(}\PY{n}{cos}\PY{p}{(}\PY{n}{x}\PY{p}{)}\PY{p}{,} \PY{n}{x}\PY{p}{,} \PY{l+m+mi}{0}\PY{p}{,} \PY{l+m+mi}{8}\PY{p}{)}
\end{Verbatim}
\end{tcolorbox}
 
            
\prompt{Out}{outcolor}{81}{}
    
    $\displaystyle 1 - \frac{x^{2}}{2} + \frac{x^{4}}{24} - \frac{x^{6}}{720} + O\left(x^{8}\right)$

    

    \begin{tcolorbox}[breakable, size=fbox, boxrule=1pt, pad at break*=1mm,colback=cellbackground, colframe=cellborder]
\prompt{In}{incolor}{82}{\boxspacing}
\begin{Verbatim}[commandchars=\\\{\}]
\PY{n}{series}\PY{p}{(}\PY{n}{E}\PY{o}{*}\PY{o}{*}\PY{p}{(}\PY{n}{x}\PY{p}{)}\PY{p}{,} \PY{n}{x}\PY{p}{,} \PY{l+m+mi}{0}\PY{p}{,} \PY{l+m+mi}{8}\PY{p}{)}
\end{Verbatim}
\end{tcolorbox}
 
            
\prompt{Out}{outcolor}{82}{}
    
    $\displaystyle 1 + x + \frac{x^{2}}{2} + \frac{x^{3}}{6} + \frac{x^{4}}{24} + \frac{x^{5}}{120} + \frac{x^{6}}{720} + \frac{x^{7}}{5040} + O\left(x^{8}\right)$

    

    \begin{tcolorbox}[breakable, size=fbox, boxrule=1pt, pad at break*=1mm,colback=cellbackground, colframe=cellborder]
\prompt{In}{incolor}{83}{\boxspacing}
\begin{Verbatim}[commandchars=\\\{\}]
\PY{n}{series}\PY{p}{(}\PY{n}{ln}\PY{p}{(}\PY{n}{x}\PY{o}{+}\PY{l+m+mi}{1}\PY{p}{)}\PY{p}{,} \PY{n}{x}\PY{p}{,} \PY{l+m+mi}{0}\PY{p}{,} \PY{l+m+mi}{6}\PY{p}{)}
\end{Verbatim}
\end{tcolorbox}
 
            
\prompt{Out}{outcolor}{83}{}
    
    $\displaystyle x - \frac{x^{2}}{2} + \frac{x^{3}}{3} - \frac{x^{4}}{4} + \frac{x^{5}}{5} + O\left(x^{6}\right)$

    

    \begin{tcolorbox}[breakable, size=fbox, boxrule=1pt, pad at break*=1mm,colback=cellbackground, colframe=cellborder]
\prompt{In}{incolor}{84}{\boxspacing}
\begin{Verbatim}[commandchars=\\\{\}]
\PY{n}{summation}\PY{p}{(}\PY{n}{σ}\PY{p}{,}\PY{p}{(}\PY{n}{x}\PY{p}{,}\PY{l+m+mi}{0}\PY{p}{,}\PY{l+m+mi}{6}\PY{p}{)}\PY{p}{)}
\end{Verbatim}
\end{tcolorbox}
 
            
\prompt{Out}{outcolor}{84}{}
    
    $\displaystyle \sqrt{2} + \frac{4 \sqrt{5}}{5} + \frac{3 \sqrt{10}}{5} + \frac{8 \sqrt{17}}{17} + \frac{5 \sqrt{26}}{13} + \frac{12 \sqrt{37}}{37}$

    

    La gráfica más sencilla que podemos hacer es de la manera siguiente

    \begin{tcolorbox}[breakable, size=fbox, boxrule=1pt, pad at break*=1mm,colback=cellbackground, colframe=cellborder]
\prompt{In}{incolor}{85}{\boxspacing}
\begin{Verbatim}[commandchars=\\\{\}]
\PY{n}{plot}\PY{p}{(}\PY{n}{σ}\PY{p}{,}\PY{p}{(}\PY{n}{x}\PY{p}{,}\PY{o}{\PYZhy{}}\PY{l+m+mi}{5}\PY{p}{,}\PY{l+m+mi}{5}\PY{p}{)}\PY{p}{)}
\end{Verbatim}
\end{tcolorbox}

    \begin{center}
    \adjustimage{max size={0.9\linewidth}{0.9\paperheight}}{06_Intro_Sympy_files/06_Intro_Sympy_136_0.png}
    \end{center}
    { \hspace*{\fill} \\}
    
            \begin{tcolorbox}[breakable, size=fbox, boxrule=.5pt, pad at break*=1mm, opacityfill=0]
\prompt{Out}{outcolor}{85}{\boxspacing}
\begin{Verbatim}[commandchars=\\\{\}]
<sympy.plotting.plot.Plot at 0x118aad110>
\end{Verbatim}
\end{tcolorbox}
        
    O si lo preferimos

    \begin{tcolorbox}[breakable, size=fbox, boxrule=1pt, pad at break*=1mm,colback=cellbackground, colframe=cellborder]
\prompt{In}{incolor}{86}{\boxspacing}
\begin{Verbatim}[commandchars=\\\{\}]
\PY{c+c1}{\PYZsh{} Creamos la gráfica}
\PY{n}{p} \PY{o}{=} \PY{n}{plot}\PY{p}{(}\PY{n}{σ}\PY{p}{,}\PY{p}{(}\PY{n}{x}\PY{p}{,}\PY{o}{\PYZhy{}}\PY{l+m+mi}{5}\PY{p}{,}\PY{l+m+mi}{5}\PY{p}{)}\PY{p}{,} \PY{n}{show}\PY{o}{=}\PY{k+kc}{False}\PY{p}{)}
\PY{c+c1}{\PYZsh{} Cambiamos los colores y etiquetas de cada función}
\PY{n}{p}\PY{p}{[}\PY{l+m+mi}{0}\PY{p}{]}\PY{o}{.}\PY{n}{line\PYZus{}color} \PY{o}{=} \PY{l+s+s1}{\PYZsq{}}\PY{l+s+s1}{red}\PY{l+s+s1}{\PYZsq{}}
\PY{n}{p}\PY{p}{[}\PY{l+m+mi}{0}\PY{p}{]}\PY{o}{.}\PY{n}{label} \PY{o}{=} \PY{l+s+s1}{\PYZsq{}}\PY{l+s+s1}{σ(x)}\PY{l+s+s1}{\PYZsq{}}
\PY{c+c1}{\PYZsh{} Agregamos un título a la gráfica y a los ejes}
\PY{n}{p}\PY{o}{.}\PY{n}{title} \PY{o}{=} \PY{l+s+s1}{\PYZsq{}}\PY{l+s+s1}{σ(x)}\PY{l+s+s1}{\PYZsq{}}
\PY{n}{p}\PY{o}{.}\PY{n}{xlabel} \PY{o}{=} \PY{l+s+s1}{\PYZsq{}}\PY{l+s+s1}{x}\PY{l+s+s1}{\PYZsq{}}
\PY{n}{p}\PY{o}{.}\PY{n}{ylabel} \PY{o}{=} \PY{k+kc}{False}
\PY{c+c1}{\PYZsh{} Agregamos una leyenda}
\PY{n}{p}\PY{o}{.}\PY{n}{legend} \PY{o}{=} \PY{k+kc}{True}
\PY{n}{p}\PY{o}{.}\PY{n}{legend\PYZus{}loc} \PY{o}{=} \PY{l+s+s1}{\PYZsq{}}\PY{l+s+s1}{upper left}\PY{l+s+s1}{\PYZsq{}}
\PY{c+c1}{\PYZsh{} Mostramos la gráfica}
\PY{n}{p}\PY{o}{.}\PY{n}{show}\PY{p}{(}\PY{p}{)}
\end{Verbatim}
\end{tcolorbox}

    \begin{center}
    \adjustimage{max size={0.9\linewidth}{0.9\paperheight}}{06_Intro_Sympy_files/06_Intro_Sympy_138_0.png}
    \end{center}
    { \hspace*{\fill} \\}
    
    Las funciones podemos definirlas de manera abstracta para usarlas como
objetos matemáticos

    \begin{tcolorbox}[breakable, size=fbox, boxrule=1pt, pad at break*=1mm,colback=cellbackground, colframe=cellborder]
\prompt{In}{incolor}{87}{\boxspacing}
\begin{Verbatim}[commandchars=\\\{\}]
\PY{n}{f} \PY{o}{=} \PY{n}{Function}\PY{p}{(}\PY{l+s+s1}{\PYZsq{}}\PY{l+s+s1}{f}\PY{l+s+s1}{\PYZsq{}}\PY{p}{)}
\PY{n}{g} \PY{o}{=} \PY{n}{Function}\PY{p}{(}\PY{l+s+s1}{\PYZsq{}}\PY{l+s+s1}{g}\PY{l+s+s1}{\PYZsq{}}\PY{p}{)}\PY{p}{(}\PY{n}{x}\PY{p}{)}
\end{Verbatim}
\end{tcolorbox}

    \begin{tcolorbox}[breakable, size=fbox, boxrule=1pt, pad at break*=1mm,colback=cellbackground, colframe=cellborder]
\prompt{In}{incolor}{88}{\boxspacing}
\begin{Verbatim}[commandchars=\\\{\}]
\PY{n}{f}
\end{Verbatim}
\end{tcolorbox}

            \begin{tcolorbox}[breakable, size=fbox, boxrule=.5pt, pad at break*=1mm, opacityfill=0]
\prompt{Out}{outcolor}{88}{\boxspacing}
\begin{Verbatim}[commandchars=\\\{\}]
f
\end{Verbatim}
\end{tcolorbox}
        
    \begin{tcolorbox}[breakable, size=fbox, boxrule=1pt, pad at break*=1mm,colback=cellbackground, colframe=cellborder]
\prompt{In}{incolor}{89}{\boxspacing}
\begin{Verbatim}[commandchars=\\\{\}]
\PY{n}{g}
\end{Verbatim}
\end{tcolorbox}
 
            
\prompt{Out}{outcolor}{89}{}
    
    $\displaystyle g{\left(x \right)}$

    

    \begin{tcolorbox}[breakable, size=fbox, boxrule=1pt, pad at break*=1mm,colback=cellbackground, colframe=cellborder]
\prompt{In}{incolor}{90}{\boxspacing}
\begin{Verbatim}[commandchars=\\\{\}]
\PY{p}{(}\PY{n}{f}\PY{p}{(}\PY{n}{x}\PY{p}{)}\PY{o}{+}\PY{n}{g}\PY{p}{)}\PY{o}{.}\PY{n}{diff}\PY{p}{(}\PY{p}{)}
\end{Verbatim}
\end{tcolorbox}
 
            
\prompt{Out}{outcolor}{90}{}
    
    $\displaystyle \frac{d}{d x} f{\left(x \right)} + \frac{d}{d x} g{\left(x \right)}$

    

    \begin{tcolorbox}[breakable, size=fbox, boxrule=1pt, pad at break*=1mm,colback=cellbackground, colframe=cellborder]
\prompt{In}{incolor}{91}{\boxspacing}
\begin{Verbatim}[commandchars=\\\{\}]
\PY{n}{f} \PY{o}{=} \PY{n}{Function}\PY{p}{(}\PY{l+s+s1}{\PYZsq{}}\PY{l+s+s1}{f}\PY{l+s+s1}{\PYZsq{}}\PY{p}{)}\PY{p}{(}\PY{n}{x}\PY{p}{)}
\PY{n}{f}
\end{Verbatim}
\end{tcolorbox}
 
            
\prompt{Out}{outcolor}{91}{}
    
    $\displaystyle f{\left(x \right)}$

    

    \begin{tcolorbox}[breakable, size=fbox, boxrule=1pt, pad at break*=1mm,colback=cellbackground, colframe=cellborder]
\prompt{In}{incolor}{92}{\boxspacing}
\begin{Verbatim}[commandchars=\\\{\}]
\PY{n}{integrate}\PY{p}{(}\PY{n}{g}\PY{p}{,}\PY{n}{x}\PY{p}{)}
\end{Verbatim}
\end{tcolorbox}
 
            
\prompt{Out}{outcolor}{92}{}
    
    $\displaystyle \int g{\left(x \right)}\, dx$

    

    La regla de la cadena

    \begin{tcolorbox}[breakable, size=fbox, boxrule=1pt, pad at break*=1mm,colback=cellbackground, colframe=cellborder]
\prompt{In}{incolor}{93}{\boxspacing}
\begin{Verbatim}[commandchars=\\\{\}]
\PY{n}{diff}\PY{p}{(}\PY{n}{f}\PY{o}{*}\PY{n}{g}\PY{p}{,}\PY{n}{x}\PY{p}{)}
\end{Verbatim}
\end{tcolorbox}
 
            
\prompt{Out}{outcolor}{93}{}
    
    $\displaystyle f{\left(x \right)} \frac{d}{d x} g{\left(x \right)} + g{\left(x \right)} \frac{d}{d x} f{\left(x \right)}$

    

    \begin{tcolorbox}[breakable, size=fbox, boxrule=1pt, pad at break*=1mm,colback=cellbackground, colframe=cellborder]
\prompt{In}{incolor}{94}{\boxspacing}
\begin{Verbatim}[commandchars=\\\{\}]
\PY{n}{diff}\PY{p}{(}\PY{n}{g}\PY{o}{/}\PY{n}{f}\PY{p}{,}\PY{n}{x}\PY{p}{)}\PY{o}{.}\PY{n}{factor}\PY{p}{(}\PY{p}{)}
\end{Verbatim}
\end{tcolorbox}
 
            
\prompt{Out}{outcolor}{94}{}
    
    $\displaystyle - \frac{- f{\left(x \right)} \frac{d}{d x} g{\left(x \right)} + g{\left(x \right)} \frac{d}{d x} f{\left(x \right)}}{f^{2}{\left(x \right)}}$

    

    \hypertarget{ecuaciones-algebruxe1icas}{%
\section{3) Ecuaciones algebráicas}\label{ecuaciones-algebruxe1icas}}

    Para escribir ecuaciones usamos la siguiente sintáxis

    \begin{tcolorbox}[breakable, size=fbox, boxrule=1pt, pad at break*=1mm,colback=cellbackground, colframe=cellborder]
\prompt{In}{incolor}{95}{\boxspacing}
\begin{Verbatim}[commandchars=\\\{\}]
\PY{n}{Eq}\PY{p}{(}\PY{n}{Function}\PY{p}{(}\PY{l+s+s2}{\PYZdq{}}\PY{l+s+s2}{f}\PY{l+s+s2}{\PYZdq{}}\PY{p}{)}\PY{p}{(}\PY{n}{x}\PY{p}{)}\PY{p}{,}\PY{n}{Sum}\PY{p}{(}\PY{n}{x}\PY{p}{,}\PY{p}{(}\PY{n}{x}\PY{p}{,}\PY{l+m+mi}{1}\PY{p}{,}\PY{l+m+mi}{10}\PY{p}{)}\PY{p}{)}\PY{p}{)}
\end{Verbatim}
\end{tcolorbox}
 
            
\prompt{Out}{outcolor}{95}{}
    
    $\displaystyle f{\left(x \right)} = \sum_{x=1}^{10} x$

    

    \begin{tcolorbox}[breakable, size=fbox, boxrule=1pt, pad at break*=1mm,colback=cellbackground, colframe=cellborder]
\prompt{In}{incolor}{96}{\boxspacing}
\begin{Verbatim}[commandchars=\\\{\}]
\PY{n}{Eq}\PY{p}{(}\PY{n}{x}\PY{o}{+}\PY{l+m+mi}{5}\PY{p}{,} \PY{l+m+mi}{3}\PY{p}{)}
\end{Verbatim}
\end{tcolorbox}
 
            
\prompt{Out}{outcolor}{96}{}
    
    $\displaystyle x + 5 = 3$

    

    \begin{tcolorbox}[breakable, size=fbox, boxrule=1pt, pad at break*=1mm,colback=cellbackground, colframe=cellborder]
\prompt{In}{incolor}{97}{\boxspacing}
\begin{Verbatim}[commandchars=\\\{\}]
\PY{n}{solveset}\PY{p}{(}\PY{n}{Eq}\PY{p}{(}\PY{n}{x}\PY{o}{+}\PY{l+m+mi}{5}\PY{p}{,} \PY{l+m+mi}{3}\PY{p}{)}\PY{p}{,} \PY{n}{x}\PY{p}{)}
\end{Verbatim}
\end{tcolorbox}
 
            
\prompt{Out}{outcolor}{97}{}
    
    $\displaystyle \left\{-2\right\}$

    

    Otro ejemplo la ecuación: \(cos(x)=1\), la resolveremos para \(x\)

    \begin{tcolorbox}[breakable, size=fbox, boxrule=1pt, pad at break*=1mm,colback=cellbackground, colframe=cellborder]
\prompt{In}{incolor}{98}{\boxspacing}
\begin{Verbatim}[commandchars=\\\{\}]
\PY{n}{solveset}\PY{p}{(}\PY{n}{Eq}\PY{p}{(}\PY{n}{cos}\PY{p}{(}\PY{n}{x}\PY{p}{)}\PY{p}{,}\PY{l+m+mi}{1} \PY{p}{)}\PY{p}{,} \PY{n}{x}\PY{p}{,} \PY{n}{domain}\PY{o}{=}\PY{n}{S}\PY{o}{.}\PY{n}{Reals}\PY{p}{)}
\end{Verbatim}
\end{tcolorbox}
 
            
\prompt{Out}{outcolor}{98}{}
    
    $\displaystyle \left\{2 n \pi\; \middle|\; n \in \mathbb{Z}\right\}$

    

    ``solveset'' devuelve un objeto ``set''. Para los casos en los que no
``conoce'' todas las soluciones devuelve un ``ConditionSet'' con una
solución parcial. Para la entrada sólo toma la ecuación, las variables a
resolver y el argumento opcional ``dominio'' sobre el que se va a
resolver la ecuación.

    Consideremos el siguiente ejemplo

    \begin{tcolorbox}[breakable, size=fbox, boxrule=1pt, pad at break*=1mm,colback=cellbackground, colframe=cellborder]
\prompt{In}{incolor}{99}{\boxspacing}
\begin{Verbatim}[commandchars=\\\{\}]
\PY{n}{Ec1}\PY{o}{=}\PY{n}{Eq}\PY{p}{(}\PY{n}{x}\PY{o}{*}\PY{o}{*}\PY{l+m+mi}{2}\PY{o}{+}\PY{l+m+mi}{2}\PY{o}{*}\PY{n}{x}\PY{p}{,} \PY{l+m+mi}{1}\PY{p}{)}
\PY{n}{Ec1}
\end{Verbatim}
\end{tcolorbox}
 
            
\prompt{Out}{outcolor}{99}{}
    
    $\displaystyle x^{2} + 2 x = 1$

    

    La función ``solve'' se utiliza principalmente para resolver ecuaciones
algebraicas y sistemas de ecuaciones algebraicas. Puede manejar una
amplia gama de ecuaciones algebraicas y puede devolver soluciones en
forma de símbolos, números reales o complejos. La salida es una lista.

    \begin{tcolorbox}[breakable, size=fbox, boxrule=1pt, pad at break*=1mm,colback=cellbackground, colframe=cellborder]
\prompt{In}{incolor}{100}{\boxspacing}
\begin{Verbatim}[commandchars=\\\{\}]
\PY{n}{solve}\PY{p}{(}\PY{n}{Ec1}\PY{p}{,}\PY{n}{x}\PY{p}{)}
\end{Verbatim}
\end{tcolorbox}

            \begin{tcolorbox}[breakable, size=fbox, boxrule=.5pt, pad at break*=1mm, opacityfill=0]
\prompt{Out}{outcolor}{100}{\boxspacing}
\begin{Verbatim}[commandchars=\\\{\}]
[-1 + sqrt(2), -sqrt(2) - 1]
\end{Verbatim}
\end{tcolorbox}
        
    \begin{tcolorbox}[breakable, size=fbox, boxrule=1pt, pad at break*=1mm,colback=cellbackground, colframe=cellborder]
\prompt{In}{incolor}{101}{\boxspacing}
\begin{Verbatim}[commandchars=\\\{\}]
\PY{n}{Ec}\PY{o}{=}\PY{n}{Eq}\PY{p}{(}\PY{n}{a}\PY{o}{*}\PY{n}{x}\PY{o}{*}\PY{o}{*}\PY{l+m+mi}{2}\PY{o}{+}\PY{n}{b}\PY{o}{*}\PY{n}{x}\PY{o}{+}\PY{n}{c}\PY{p}{,}\PY{l+m+mi}{0}\PY{p}{)}
\PY{n}{Ec}
\end{Verbatim}
\end{tcolorbox}
 
            
\prompt{Out}{outcolor}{101}{}
    
    $\displaystyle a x^{2} + b x + c = 0$

    

    \begin{tcolorbox}[breakable, size=fbox, boxrule=1pt, pad at break*=1mm,colback=cellbackground, colframe=cellborder]
\prompt{In}{incolor}{102}{\boxspacing}
\begin{Verbatim}[commandchars=\\\{\}]
\PY{n}{solveset}\PY{p}{(}\PY{n}{Ec}\PY{p}{,} \PY{n}{x}\PY{p}{)}
\end{Verbatim}
\end{tcolorbox}
 
            
\prompt{Out}{outcolor}{102}{}
    
    $\displaystyle \left\{- \frac{b}{2 a} - \frac{\sqrt{- 4 a c + b^{2}}}{2 a}, - \frac{b}{2 a} + \frac{\sqrt{- 4 a c + b^{2}}}{2 a}\right\}$

    

    \begin{tcolorbox}[breakable, size=fbox, boxrule=1pt, pad at break*=1mm,colback=cellbackground, colframe=cellborder]
\prompt{In}{incolor}{103}{\boxspacing}
\begin{Verbatim}[commandchars=\\\{\}]
\PY{n}{solveset}\PY{p}{(}\PY{n}{Ec1}\PY{p}{,} \PY{n}{x}\PY{p}{)}
\end{Verbatim}
\end{tcolorbox}
 
            
\prompt{Out}{outcolor}{103}{}
    
    $\displaystyle \left\{-1 + \sqrt{2}, - \sqrt{2} - 1\right\}$

    

    \begin{tcolorbox}[breakable, size=fbox, boxrule=1pt, pad at break*=1mm,colback=cellbackground, colframe=cellborder]
\prompt{In}{incolor}{104}{\boxspacing}
\begin{Verbatim}[commandchars=\\\{\}]
\PY{n}{Ec2}\PY{o}{=}\PY{n}{Eq}\PY{p}{(}\PY{l+m+mi}{2}\PY{o}{*}\PY{n}{x}\PY{o}{*}\PY{o}{*}\PY{l+m+mi}{2}\PY{o}{+}\PY{l+m+mi}{2}\PY{o}{*}\PY{n}{x}\PY{p}{,} \PY{o}{\PYZhy{}}\PY{l+m+mi}{1}\PY{p}{)}
\PY{n}{Ec2}
\end{Verbatim}
\end{tcolorbox}
 
            
\prompt{Out}{outcolor}{104}{}
    
    $\displaystyle 2 x^{2} + 2 x = -1$

    

    \begin{tcolorbox}[breakable, size=fbox, boxrule=1pt, pad at break*=1mm,colback=cellbackground, colframe=cellborder]
\prompt{In}{incolor}{105}{\boxspacing}
\begin{Verbatim}[commandchars=\\\{\}]
\PY{n}{solve}\PY{p}{(}\PY{n}{Ec2}\PY{p}{,} \PY{n}{x}\PY{p}{)}
\end{Verbatim}
\end{tcolorbox}

            \begin{tcolorbox}[breakable, size=fbox, boxrule=.5pt, pad at break*=1mm, opacityfill=0]
\prompt{Out}{outcolor}{105}{\boxspacing}
\begin{Verbatim}[commandchars=\\\{\}]
[-1/2 - I/2, -1/2 + I/2]
\end{Verbatim}
\end{tcolorbox}
        
    \begin{tcolorbox}[breakable, size=fbox, boxrule=1pt, pad at break*=1mm,colback=cellbackground, colframe=cellborder]
\prompt{In}{incolor}{106}{\boxspacing}
\begin{Verbatim}[commandchars=\\\{\}]
\PY{n}{Ec3}\PY{o}{=} \PY{n}{Eq}\PY{p}{(}\PY{n}{x}\PY{o}{*}\PY{o}{*}\PY{l+m+mi}{6}\PY{o}{+}\PY{n}{x}\PY{o}{*}\PY{o}{*}\PY{l+m+mi}{4}\PY{o}{\PYZhy{}}\PY{n}{x}\PY{o}{*}\PY{o}{*}\PY{l+m+mi}{3}\PY{o}{+}\PY{n}{x}\PY{o}{\PYZhy{}}\PY{l+m+mi}{2}\PY{p}{,}\PY{l+m+mi}{0}\PY{p}{)}
\PY{n}{Ec3}
\end{Verbatim}
\end{tcolorbox}
 
            
\prompt{Out}{outcolor}{106}{}
    
    $\displaystyle x^{6} + x^{4} - x^{3} + x - 2 = 0$

    

    \begin{tcolorbox}[breakable, size=fbox, boxrule=1pt, pad at break*=1mm,colback=cellbackground, colframe=cellborder]
\prompt{In}{incolor}{107}{\boxspacing}
\begin{Verbatim}[commandchars=\\\{\}]
\PY{n}{factor}\PY{p}{(}\PY{n}{Ec3}\PY{p}{)}
\end{Verbatim}
\end{tcolorbox}
 
            
\prompt{Out}{outcolor}{107}{}
    
    $\displaystyle \left(x - 1\right) \left(x + 1\right) \left(x^{2} - x + 1\right) \left(x^{2} + x + 2\right) = 0$

    

    \begin{tcolorbox}[breakable, size=fbox, boxrule=1pt, pad at break*=1mm,colback=cellbackground, colframe=cellborder]
\prompt{In}{incolor}{108}{\boxspacing}
\begin{Verbatim}[commandchars=\\\{\}]
\PY{n}{solveset}\PY{p}{(}\PY{n}{Ec3}\PY{p}{,} \PY{n}{x}\PY{p}{)}
\end{Verbatim}
\end{tcolorbox}
 
            
\prompt{Out}{outcolor}{108}{}
    
    $\displaystyle \left\{-1, 1, - \frac{1}{2} - \frac{\sqrt{7} i}{2}, - \frac{1}{2} + \frac{\sqrt{7} i}{2}, \frac{1}{2} - \frac{\sqrt{3} i}{2}, \frac{1}{2} + \frac{\sqrt{3} i}{2}\right\}$

    

    \begin{tcolorbox}[breakable, size=fbox, boxrule=1pt, pad at break*=1mm,colback=cellbackground, colframe=cellborder]
\prompt{In}{incolor}{109}{\boxspacing}
\begin{Verbatim}[commandchars=\\\{\}]
\PY{n}{Ec4}\PY{o}{=} \PY{n}{Eq}\PY{p}{(}\PY{n}{x}\PY{o}{*}\PY{o}{*}\PY{l+m+mi}{7}\PY{o}{+}\PY{n}{x}\PY{o}{*}\PY{o}{*}\PY{l+m+mi}{4}\PY{o}{\PYZhy{}}\PY{n}{x}\PY{o}{*}\PY{o}{*}\PY{l+m+mi}{3}\PY{o}{+}\PY{n}{x}\PY{p}{,}\PY{l+m+mi}{2}\PY{p}{)}
\PY{n}{Ec4}
\end{Verbatim}
\end{tcolorbox}
 
            
\prompt{Out}{outcolor}{109}{}
    
    $\displaystyle x^{7} + x^{4} - x^{3} + x = 2$

    

    \begin{tcolorbox}[breakable, size=fbox, boxrule=1pt, pad at break*=1mm,colback=cellbackground, colframe=cellborder]
\prompt{In}{incolor}{110}{\boxspacing}
\begin{Verbatim}[commandchars=\\\{\}]
\PY{n}{solve}\PY{p}{(}\PY{n}{Ec4}\PY{p}{,} \PY{n}{x}\PY{p}{)}
\end{Verbatim}
\end{tcolorbox}

            \begin{tcolorbox}[breakable, size=fbox, boxrule=.5pt, pad at break*=1mm, opacityfill=0]
\prompt{Out}{outcolor}{110}{\boxspacing}
\begin{Verbatim}[commandchars=\\\{\}]
[1,
 CRootOf(x**6 + x**5 + x**4 + 2*x**3 + x**2 + x + 2, 0),
 CRootOf(x**6 + x**5 + x**4 + 2*x**3 + x**2 + x + 2, 1),
 CRootOf(x**6 + x**5 + x**4 + 2*x**3 + x**2 + x + 2, 2),
 CRootOf(x**6 + x**5 + x**4 + 2*x**3 + x**2 + x + 2, 3),
 CRootOf(x**6 + x**5 + x**4 + 2*x**3 + x**2 + x + 2, 4),
 CRootOf(x**6 + x**5 + x**4 + 2*x**3 + x**2 + x + 2, 5)]
\end{Verbatim}
\end{tcolorbox}
        
    \begin{tcolorbox}[breakable, size=fbox, boxrule=1pt, pad at break*=1mm,colback=cellbackground, colframe=cellborder]
\prompt{In}{incolor}{111}{\boxspacing}
\begin{Verbatim}[commandchars=\\\{\}]
\PY{n}{sols} \PY{o}{=} \PY{n}{solveset}\PY{p}{(}\PY{n}{Ec4}\PY{p}{,} \PY{n}{x}\PY{p}{)}
\end{Verbatim}
\end{tcolorbox}

    \begin{tcolorbox}[breakable, size=fbox, boxrule=1pt, pad at break*=1mm,colback=cellbackground, colframe=cellborder]
\prompt{In}{incolor}{112}{\boxspacing}
\begin{Verbatim}[commandchars=\\\{\}]
\PY{n}{sols}\PY{o}{.}\PY{n}{evalf}\PY{p}{(}\PY{l+m+mi}{3}\PY{p}{)}
\end{Verbatim}
\end{tcolorbox}
 
            
\prompt{Out}{outcolor}{112}{}
    
    $\displaystyle \left\{1.0, -1.12 - 0.339 i, -1.12 + 0.339 i, -0.0158 - 1.16 i, -0.0158 + 1.16 i, 0.633 - 0.83 i, 0.633 + 0.83 i\right\}$

    

    Para resolver sistemas de ecuaciones, como por ejemplo: \[
x+y+z-1=0 \\
x+y+2z-3=0 \\
x-y+z-1=0
\]

    \begin{tcolorbox}[breakable, size=fbox, boxrule=1pt, pad at break*=1mm,colback=cellbackground, colframe=cellborder]
\prompt{In}{incolor}{113}{\boxspacing}
\begin{Verbatim}[commandchars=\\\{\}]
\PY{n}{linsolve}\PY{p}{(}\PY{p}{[}\PY{n}{x} \PY{o}{+} \PY{n}{y} \PY{o}{+} \PY{n}{z} \PY{o}{\PYZhy{}} \PY{l+m+mi}{1}\PY{p}{,} \PY{n}{x} \PY{o}{+} \PY{n}{y} \PY{o}{+} \PY{l+m+mi}{2}\PY{o}{*}\PY{n}{z} \PY{o}{\PYZhy{}} \PY{l+m+mi}{3}\PY{p}{,} \PY{n}{x}\PY{o}{\PYZhy{}}\PY{n}{y}\PY{o}{+}\PY{n}{z}\PY{o}{\PYZhy{}}\PY{l+m+mi}{1} \PY{p}{]}\PY{p}{,} \PY{p}{(}\PY{n}{x}\PY{p}{,} \PY{n}{y}\PY{p}{,} \PY{n}{z}\PY{p}{)}\PY{p}{)}
\end{Verbatim}
\end{tcolorbox}
 
            
\prompt{Out}{outcolor}{113}{}
    
    $\displaystyle \left\{\left( -1, \  0, \  2\right)\right\}$

    

    \begin{tcolorbox}[breakable, size=fbox, boxrule=1pt, pad at break*=1mm,colback=cellbackground, colframe=cellborder]
\prompt{In}{incolor}{114}{\boxspacing}
\begin{Verbatim}[commandchars=\\\{\}]
\PY{n}{Ec1}\PY{o}{=}\PY{l+m+mi}{2}\PY{o}{*}\PY{n}{x}\PY{o}{\PYZhy{}}\PY{l+m+mi}{2}\PY{o}{*}\PY{n}{y}\PY{o}{+}\PY{n}{z}\PY{o}{+}\PY{l+m+mi}{3}
\PY{n}{Ec2}\PY{o}{=}\PY{n}{x}\PY{o}{+}\PY{l+m+mi}{3}\PY{o}{*}\PY{n}{y}\PY{o}{\PYZhy{}}\PY{l+m+mi}{2}\PY{o}{*}\PY{n}{z}\PY{o}{\PYZhy{}}\PY{l+m+mi}{1}
\PY{n}{Ec3}\PY{o}{=}\PY{l+m+mi}{3}\PY{o}{*}\PY{n}{x}\PY{o}{\PYZhy{}}\PY{n}{y}\PY{o}{\PYZhy{}}\PY{n}{z}\PY{o}{\PYZhy{}}\PY{l+m+mi}{2}
\PY{n}{linsolve}\PY{p}{(}\PY{p}{[}\PY{n}{Ec1}\PY{p}{,} \PY{n}{Ec2}\PY{p}{,} \PY{n}{Ec3} \PY{p}{]}\PY{p}{,} \PY{p}{(}\PY{n}{x}\PY{p}{,} \PY{n}{y}\PY{p}{,} \PY{n}{z}\PY{p}{)}\PY{p}{)}
\end{Verbatim}
\end{tcolorbox}
 
            
\prompt{Out}{outcolor}{114}{}
    
    $\displaystyle \left\{\left( - \frac{7}{5}, \  -2, \  - \frac{21}{5}\right)\right\}$

    

    \hypertarget{ecuaciones-diferenciales}{%
\section{4) Ecuaciones diferenciales}\label{ecuaciones-diferenciales}}

    Definamos la función \(f=f(x)\)

    \begin{tcolorbox}[breakable, size=fbox, boxrule=1pt, pad at break*=1mm,colback=cellbackground, colframe=cellborder]
\prompt{In}{incolor}{115}{\boxspacing}
\begin{Verbatim}[commandchars=\\\{\}]
\PY{n}{f} \PY{o}{=} \PY{n}{Function}\PY{p}{(}\PY{l+s+s1}{\PYZsq{}}\PY{l+s+s1}{f}\PY{l+s+s1}{\PYZsq{}}\PY{p}{)}\PY{p}{(}\PY{n}{x}\PY{p}{)}
\end{Verbatim}
\end{tcolorbox}

    Resolvamosla ecuación diferencial \[
f(x)-\frac{df(x)}{dx}=0
\]

    \begin{tcolorbox}[breakable, size=fbox, boxrule=1pt, pad at break*=1mm,colback=cellbackground, colframe=cellborder]
\prompt{In}{incolor}{116}{\boxspacing}
\begin{Verbatim}[commandchars=\\\{\}]
\PY{n}{dsolve}\PY{p}{(} \PY{n}{f} \PY{o}{\PYZhy{}} \PY{n}{diff}\PY{p}{(}\PY{n}{f}\PY{p}{,} \PY{n}{x}\PY{p}{)}\PY{p}{,} \PY{n}{f} \PY{p}{)}
\end{Verbatim}
\end{tcolorbox}
 
            
\prompt{Out}{outcolor}{116}{}
    
    $\displaystyle f{\left(x \right)} = C_{1} e^{x}$

    

    También podemos escribir primero la ecuación diferencial

    \begin{tcolorbox}[breakable, size=fbox, boxrule=1pt, pad at break*=1mm,colback=cellbackground, colframe=cellborder]
\prompt{In}{incolor}{117}{\boxspacing}
\begin{Verbatim}[commandchars=\\\{\}]
\PY{n}{ed} \PY{o}{=} \PY{n}{Eq}\PY{p}{(}\PY{n}{f} \PY{o}{\PYZhy{}} \PY{n}{diff}\PY{p}{(}\PY{n}{f}\PY{p}{,} \PY{n}{x}\PY{p}{)}\PY{p}{,}\PY{l+m+mi}{0}\PY{p}{)}
\PY{n}{ed}
\end{Verbatim}
\end{tcolorbox}
 
            
\prompt{Out}{outcolor}{117}{}
    
    $\displaystyle f{\left(x \right)} - \frac{d}{d x} f{\left(x \right)} = 0$

    

    \begin{tcolorbox}[breakable, size=fbox, boxrule=1pt, pad at break*=1mm,colback=cellbackground, colframe=cellborder]
\prompt{In}{incolor}{118}{\boxspacing}
\begin{Verbatim}[commandchars=\\\{\}]
\PY{n}{dsolve}\PY{p}{(}\PY{n}{ed}\PY{p}{,} \PY{n}{f}\PY{p}{)}
\end{Verbatim}
\end{tcolorbox}
 
            
\prompt{Out}{outcolor}{118}{}
    
    $\displaystyle f{\left(x \right)} = C_{1} e^{x}$

    

    Veamos la ecuación para el oscilador armónico simple \[
\frac{d^2y(t)}{dt^2}+\omega_0^2 y(t)=0
\]

    \begin{tcolorbox}[breakable, size=fbox, boxrule=1pt, pad at break*=1mm,colback=cellbackground, colframe=cellborder]
\prompt{In}{incolor}{119}{\boxspacing}
\begin{Verbatim}[commandchars=\\\{\}]
\PY{n}{ω0} \PY{o}{=} \PY{n}{symbols}\PY{p}{(}\PY{l+s+s1}{\PYZsq{}}\PY{l+s+s1}{ω0}\PY{l+s+s1}{\PYZsq{}}\PY{p}{)}
\PY{n}{y} \PY{o}{=} \PY{n}{Function}\PY{p}{(}\PY{l+s+s1}{\PYZsq{}}\PY{l+s+s1}{y}\PY{l+s+s1}{\PYZsq{}}\PY{p}{)}\PY{p}{(}\PY{n}{t}\PY{p}{)}
\PY{n}{ed1} \PY{o}{=} \PY{n}{Eq}\PY{p}{(}\PY{n}{y}\PY{o}{.}\PY{n}{diff}\PY{p}{(}\PY{n}{t}\PY{p}{,}\PY{l+m+mi}{2}\PY{p}{)}  \PY{o}{+} \PY{n}{ω0}\PY{o}{*}\PY{o}{*}\PY{l+m+mi}{2}\PY{o}{*}\PY{n}{y}\PY{p}{,}\PY{l+m+mi}{0}\PY{p}{)}
\PY{n}{ed1}
\end{Verbatim}
\end{tcolorbox}
 
            
\prompt{Out}{outcolor}{119}{}
    
    $\displaystyle ω_{0}^{2} y{\left(t \right)} + \frac{d^{2}}{d t^{2}} y{\left(t \right)} = 0$

    

    \begin{tcolorbox}[breakable, size=fbox, boxrule=1pt, pad at break*=1mm,colback=cellbackground, colframe=cellborder]
\prompt{In}{incolor}{120}{\boxspacing}
\begin{Verbatim}[commandchars=\\\{\}]
\PY{n}{dsolve}\PY{p}{(}\PY{n}{ed1}\PY{p}{,} \PY{n}{y}\PY{p}{)}
\end{Verbatim}
\end{tcolorbox}
 
            
\prompt{Out}{outcolor}{120}{}
    
    $\displaystyle y{\left(t \right)} = C_{1} e^{- i t ω_{0}} + C_{2} e^{i t ω_{0}}$

    

    Una de las ventajas de los sistemas de manipulación simbólica es que son
de gran utilidad cuando necesitamos hacer cálculos largos y tediosos.

Por ejemplo, queremos demostrar que la función \[
F(x,y,z) =\frac{\sin \left(\frac{n z \sqrt{z^2+y^2+x^2}}{\sqrt{z^2+y^2}}\right)}{\sqrt{z^2+y^2+x^2}}
\] satisface la ecuación diferencial: \[
\frac{\partial^4 F}{\partial x^4}+\frac{\partial^4 F}{\partial y^2 x^2}+\frac{\partial^4 F}{\partial z^2 x^2}+n^2\left[\frac{\partial^2 F}{\partial x^2}+\frac{\partial^2 F}{\partial y^2}\right]=0
\]

Primero debemos escribir la función \(F\)

    \begin{tcolorbox}[breakable, size=fbox, boxrule=1pt, pad at break*=1mm,colback=cellbackground, colframe=cellborder]
\prompt{In}{incolor}{121}{\boxspacing}
\begin{Verbatim}[commandchars=\\\{\}]
\PY{n}{F}\PY{o}{=}\PY{n}{sin}\PY{p}{(}\PY{n}{n}\PY{o}{*}\PY{n}{z}\PY{o}{*}\PY{n}{sqrt}\PY{p}{(}\PY{n}{x}\PY{o}{*}\PY{o}{*}\PY{l+m+mi}{2}\PY{o}{+}\PY{n}{y}\PY{o}{*}\PY{o}{*}\PY{l+m+mi}{2}\PY{o}{+}\PY{n}{z}\PY{o}{*}\PY{o}{*}\PY{l+m+mi}{2}\PY{p}{)}\PY{o}{/}\PY{n}{sqrt}\PY{p}{(}\PY{n}{y}\PY{o}{*}\PY{o}{*}\PY{l+m+mi}{2}\PY{o}{+}\PY{n}{z}\PY{o}{*}\PY{o}{*}\PY{l+m+mi}{2}\PY{p}{)}\PY{p}{)}\PY{o}{/}\PY{n}{sqrt}\PY{p}{(}\PY{n}{x}\PY{o}{*}\PY{o}{*}\PY{l+m+mi}{2}\PY{o}{+}\PY{n}{y}\PY{o}{*}\PY{o}{*}\PY{l+m+mi}{2}\PY{o}{+}\PY{n}{z}\PY{o}{*}\PY{o}{*}\PY{l+m+mi}{2}\PY{p}{)}
\PY{n}{F}
\end{Verbatim}
\end{tcolorbox}
 
            
\prompt{Out}{outcolor}{121}{}
    
    $\displaystyle \frac{\sin{\left(\frac{n z \sqrt{x^{2} + z^{2} + y^{2}{\left(t \right)}}}{\sqrt{z^{2} + y^{2}{\left(t \right)}}} \right)}}{\sqrt{x^{2} + z^{2} + y^{2}{\left(t \right)}}}$

    

    Luego podríamos usar \textbf{Derivative} para escribir la ecuación y
verificar que la escribimos bien

    \begin{tcolorbox}[breakable, size=fbox, boxrule=1pt, pad at break*=1mm,colback=cellbackground, colframe=cellborder]
\prompt{In}{incolor}{122}{\boxspacing}
\begin{Verbatim}[commandchars=\\\{\}]
\PY{n}{Derivative}\PY{p}{(}\PY{n}{F}\PY{p}{,}\PY{n}{x}\PY{p}{,}\PY{l+m+mi}{4}\PY{p}{)}\PY{o}{+}\PY{n}{Derivative}\PY{p}{(}\PY{n}{Derivative}\PY{p}{(}\PY{n}{F}\PY{p}{,}\PY{n}{x}\PY{p}{,}\PY{l+m+mi}{2}\PY{p}{)}\PY{p}{,}\PY{n}{y}\PY{p}{,}\PY{l+m+mi}{2}\PY{p}{)}\PY{o}{+} \PYZbs{}
\PY{n}{Derivative}\PY{p}{(}\PY{n}{Derivative}\PY{p}{(}\PY{n}{F}\PY{p}{,}\PY{n}{x}\PY{p}{,}\PY{l+m+mi}{2}\PY{p}{)}\PY{p}{,}\PY{n}{z}\PY{p}{,}\PY{l+m+mi}{2}\PY{p}{)} \PY{o}{+} \PYZbs{}
\PY{n}{n}\PY{o}{*}\PY{o}{*}\PY{l+m+mi}{2}\PY{o}{*}\PY{p}{(}\PY{n}{Derivative}\PY{p}{(}\PY{n}{F}\PY{p}{,}\PY{n}{x}\PY{p}{,}\PY{l+m+mi}{2}\PY{p}{)}\PY{o}{+}\PY{n}{Derivative}\PY{p}{(}\PY{n}{F}\PY{p}{,}\PY{n}{y}\PY{p}{,}\PY{l+m+mi}{2}\PY{p}{)}\PY{p}{)}
\end{Verbatim}
\end{tcolorbox}
 
            
\prompt{Out}{outcolor}{122}{}
    
    $\displaystyle n^{2} \left(\frac{\partial^{2}}{\partial x^{2}} \frac{\sin{\left(\frac{n z \sqrt{x^{2} + z^{2} + y^{2}{\left(t \right)}}}{\sqrt{z^{2} + y^{2}{\left(t \right)}}} \right)}}{\sqrt{x^{2} + z^{2} + y^{2}{\left(t \right)}}} + \frac{\partial^{2}}{\partial y{\left(t \right)}^{2}} \frac{\sin{\left(\frac{n z \sqrt{x^{2} + z^{2} + y^{2}{\left(t \right)}}}{\sqrt{z^{2} + y^{2}{\left(t \right)}}} \right)}}{\sqrt{x^{2} + z^{2} + y^{2}{\left(t \right)}}}\right) + \frac{\partial^{4}}{\partial x^{4}} \frac{\sin{\left(\frac{n z \sqrt{x^{2} + z^{2} + y^{2}{\left(t \right)}}}{\sqrt{z^{2} + y^{2}{\left(t \right)}}} \right)}}{\sqrt{x^{2} + z^{2} + y^{2}{\left(t \right)}}} + \frac{\partial^{4}}{\partial z^{2}\partial x^{2}} \frac{\sin{\left(\frac{n z \sqrt{x^{2} + z^{2} + y^{2}{\left(t \right)}}}{\sqrt{z^{2} + y^{2}{\left(t \right)}}} \right)}}{\sqrt{x^{2} + z^{2} + y^{2}{\left(t \right)}}} + \frac{\partial^{4}}{\partial y{\left(t \right)}^{2}\partial x^{2}} \frac{\sin{\left(\frac{n z \sqrt{x^{2} + z^{2} + y^{2}{\left(t \right)}}}{\sqrt{z^{2} + y^{2}{\left(t \right)}}} \right)}}{\sqrt{x^{2} + z^{2} + y^{2}{\left(t \right)}}}$

    

    Ahora le pedimos al programa que realice todas las derivadas indicadas y
que luego factorice

    \begin{tcolorbox}[breakable, size=fbox, boxrule=1pt, pad at break*=1mm,colback=cellbackground, colframe=cellborder]
\prompt{In}{incolor}{123}{\boxspacing}
\begin{Verbatim}[commandchars=\\\{\}]
\PY{n}{factor}\PY{p}{(}\PY{p}{(}\PY{n}{\PYZus{}}\PY{p}{)}\PY{o}{.}\PY{n}{doit}\PY{p}{(}\PY{p}{)}\PY{p}{)}
\end{Verbatim}
\end{tcolorbox}
 
            
\prompt{Out}{outcolor}{123}{}
    
    $\displaystyle 0$

    

    En este caso el programa demora bastante tiempo en realizar los cálculos
porque ha representado todas las derivadas y luego hace la
factorización.

    También se puede hacer la demostración de manera más directa y el tiempo
de ejecución será mucho menor

    \begin{tcolorbox}[breakable, size=fbox, boxrule=1pt, pad at break*=1mm,colback=cellbackground, colframe=cellborder]
\prompt{In}{incolor}{124}{\boxspacing}
\begin{Verbatim}[commandchars=\\\{\}]
\PY{p}{(}\PY{n}{diff}\PY{p}{(}\PY{n}{F}\PY{p}{,}\PY{n}{x}\PY{p}{,}\PY{l+m+mi}{4}\PY{p}{)}\PY{o}{+}\PY{n}{diff}\PY{p}{(}\PY{n}{diff}\PY{p}{(}\PY{n}{F}\PY{p}{,}\PY{n}{x}\PY{p}{,}\PY{l+m+mi}{2}\PY{p}{)}\PY{p}{,}\PY{n}{y}\PY{p}{,}\PY{l+m+mi}{2}\PY{p}{)}\PY{o}{+}\PY{n}{diff}\PY{p}{(}\PY{n}{diff}\PY{p}{(}\PY{n}{F}\PY{p}{,}\PY{n}{x}\PY{p}{,}\PY{l+m+mi}{2}\PY{p}{)}\PY{p}{,}\PY{n}{z}\PY{p}{,}\PY{l+m+mi}{2}\PY{p}{)}\PY{o}{+}
 \PY{n}{n}\PY{o}{*}\PY{o}{*}\PY{l+m+mi}{2}\PY{o}{*}\PY{p}{(}\PY{n}{diff}\PY{p}{(}\PY{n}{F}\PY{p}{,}\PY{n}{x}\PY{p}{,}\PY{l+m+mi}{2}\PY{p}{)}\PY{o}{+}\PY{n}{diff}\PY{p}{(}\PY{n}{F}\PY{p}{,}\PY{n}{y}\PY{p}{,}\PY{l+m+mi}{2}\PY{p}{)}\PY{p}{)}\PY{p}{)}\PY{o}{.}\PY{n}{factor}\PY{p}{(}\PY{p}{)}
\end{Verbatim}
\end{tcolorbox}
 
            
\prompt{Out}{outcolor}{124}{}
    
    $\displaystyle 0$

    

    \hypertarget{algebra-vectorial-y-matricial}{%
\section{5) Algebra vectorial y
matricial}\label{algebra-vectorial-y-matricial}}

    A los vectores podemos tratarlos como objetos matriciales

    \begin{tcolorbox}[breakable, size=fbox, boxrule=1pt, pad at break*=1mm,colback=cellbackground, colframe=cellborder]
\prompt{In}{incolor}{125}{\boxspacing}
\begin{Verbatim}[commandchars=\\\{\}]
\PY{n}{A} \PY{o}{=} \PY{n}{Matrix}\PY{p}{(}\PY{p}{[}\PY{p}{[}\PY{l+m+mi}{4}\PY{p}{,} \PY{l+m+mi}{5}\PY{p}{,} \PY{l+m+mi}{6}\PY{p}{]}\PY{p}{]}\PY{p}{)} \PY{c+c1}{\PYZsh{} un vector fila 1x3}
\PY{n}{A}
\end{Verbatim}
\end{tcolorbox}
 
            
\prompt{Out}{outcolor}{125}{}
    
    $\displaystyle \left[\begin{matrix}4 & 5 & 6\end{matrix}\right]$

    

    \begin{tcolorbox}[breakable, size=fbox, boxrule=1pt, pad at break*=1mm,colback=cellbackground, colframe=cellborder]
\prompt{In}{incolor}{126}{\boxspacing}
\begin{Verbatim}[commandchars=\\\{\}]
\PY{n}{B} \PY{o}{=} \PY{n}{Matrix} \PY{p}{(}\PY{p}{[}\PY{p}{[}\PY{l+m+mi}{7}\PY{p}{]} \PY{p}{,}\PY{p}{[}\PY{l+m+mi}{8}\PY{p}{]} \PY{p}{,} \PY{p}{[}\PY{l+m+mi}{9}\PY{p}{]}\PY{p}{]}\PY{p}{)} \PY{c+c1}{\PYZsh{} un vector columna 3x1}
\PY{n}{B}
\end{Verbatim}
\end{tcolorbox}
 
            
\prompt{Out}{outcolor}{126}{}
    
    $\displaystyle \left[\begin{matrix}7\\8\\9\end{matrix}\right]$

    

    \begin{tcolorbox}[breakable, size=fbox, boxrule=1pt, pad at break*=1mm,colback=cellbackground, colframe=cellborder]
\prompt{In}{incolor}{127}{\boxspacing}
\begin{Verbatim}[commandchars=\\\{\}]
\PY{n}{B}\PY{o}{.}\PY{n}{T} \PY{c+c1}{\PYZsh{} vector traspuesta de B}
\end{Verbatim}
\end{tcolorbox}
 
            
\prompt{Out}{outcolor}{127}{}
    
    $\displaystyle \left[\begin{matrix}7 & 8 & 9\end{matrix}\right]$

    

    \begin{tcolorbox}[breakable, size=fbox, boxrule=1pt, pad at break*=1mm,colback=cellbackground, colframe=cellborder]
\prompt{In}{incolor}{128}{\boxspacing}
\begin{Verbatim}[commandchars=\\\{\}]
\PY{n}{A}\PY{p}{[}\PY{l+m+mi}{0}\PY{p}{]} \PY{c+c1}{\PYZsh{} Primera componente del vector A (índice 0)}
\end{Verbatim}
\end{tcolorbox}
 
            
\prompt{Out}{outcolor}{128}{}
    
    $\displaystyle 4$

    

    \begin{tcolorbox}[breakable, size=fbox, boxrule=1pt, pad at break*=1mm,colback=cellbackground, colframe=cellborder]
\prompt{In}{incolor}{129}{\boxspacing}
\begin{Verbatim}[commandchars=\\\{\}]
\PY{n}{A}\PY{o}{.}\PY{n}{norm}\PY{p}{(}\PY{p}{)} \PY{c+c1}{\PYZsh{} norma del vector A}
\end{Verbatim}
\end{tcolorbox}
 
            
\prompt{Out}{outcolor}{129}{}
    
    $\displaystyle \sqrt{77}$

    

    \begin{tcolorbox}[breakable, size=fbox, boxrule=1pt, pad at break*=1mm,colback=cellbackground, colframe=cellborder]
\prompt{In}{incolor}{130}{\boxspacing}
\begin{Verbatim}[commandchars=\\\{\}]
\PY{n}{Ahat} \PY{o}{=} \PY{n}{A}\PY{o}{/}\PY{n}{A}\PY{o}{.}\PY{n}{norm}\PY{p}{(}\PY{p}{)} \PY{c+c1}{\PYZsh{} vector unitario asociada a A}
\PY{n}{Ahat}
\end{Verbatim}
\end{tcolorbox}
 
            
\prompt{Out}{outcolor}{130}{}
    
    $\displaystyle \left[\begin{matrix}\frac{4 \sqrt{77}}{77} & \frac{5 \sqrt{77}}{77} & \frac{6 \sqrt{77}}{77}\end{matrix}\right]$

    

    \begin{tcolorbox}[breakable, size=fbox, boxrule=1pt, pad at break*=1mm,colback=cellbackground, colframe=cellborder]
\prompt{In}{incolor}{131}{\boxspacing}
\begin{Verbatim}[commandchars=\\\{\}]
\PY{n}{Ahat}\PY{o}{.}\PY{n}{norm}\PY{p}{(}\PY{p}{)}
\end{Verbatim}
\end{tcolorbox}
 
            
\prompt{Out}{outcolor}{131}{}
    
    $\displaystyle 1$

    

    Definamos los siguientes vectores \[
\mathbf{a}=2 \mathbf{i}+4 \mathbf{j}+6 \mathbf{k}, \quad \mathbf{b}=5 \mathbf{i}+7 \mathbf{j}+9 \mathbf{k} \,\,\text { y }\,\,  \mathbf{c}=\mathbf{i}+3 \mathbf{j}
\]

    \begin{tcolorbox}[breakable, size=fbox, boxrule=1pt, pad at break*=1mm,colback=cellbackground, colframe=cellborder]
\prompt{In}{incolor}{132}{\boxspacing}
\begin{Verbatim}[commandchars=\\\{\}]
\PY{n}{a} \PY{o}{=} \PY{n}{Matrix}\PY{p}{(}\PY{p}{[}\PY{l+m+mi}{2}\PY{p}{,}\PY{l+m+mi}{4}\PY{p}{,}\PY{l+m+mi}{6}\PY{p}{]}\PY{p}{)}
\PY{n}{b} \PY{o}{=} \PY{n}{Matrix}\PY{p}{(}\PY{p}{[}\PY{l+m+mi}{5}\PY{p}{,}\PY{l+m+mi}{7}\PY{p}{,}\PY{l+m+mi}{9}\PY{p}{]}\PY{p}{)}
\PY{n}{c} \PY{o}{=} \PY{n}{Matrix}\PY{p}{(}\PY{p}{[}\PY{l+m+mi}{1}\PY{p}{,}\PY{l+m+mi}{3}\PY{p}{,}\PY{l+m+mi}{0}\PY{p}{]}\PY{p}{)}
\end{Verbatim}
\end{tcolorbox}

    \begin{tcolorbox}[breakable, size=fbox, boxrule=1pt, pad at break*=1mm,colback=cellbackground, colframe=cellborder]
\prompt{In}{incolor}{133}{\boxspacing}
\begin{Verbatim}[commandchars=\\\{\}]
\PY{n}{a}\PY{o}{+}\PY{n}{b}\PY{o}{+}\PY{n}{c}
\end{Verbatim}
\end{tcolorbox}
 
            
\prompt{Out}{outcolor}{133}{}
    
    $\displaystyle \left[\begin{matrix}8\\14\\15\end{matrix}\right]$

    

    \begin{tcolorbox}[breakable, size=fbox, boxrule=1pt, pad at break*=1mm,colback=cellbackground, colframe=cellborder]
\prompt{In}{incolor}{134}{\boxspacing}
\begin{Verbatim}[commandchars=\\\{\}]
\PY{l+m+mi}{3}\PY{o}{*}\PY{n}{a}\PY{o}{+}\PY{l+m+mi}{5}\PY{o}{*}\PY{n}{b}\PY{o}{\PYZhy{}}\PY{n}{c}
\end{Verbatim}
\end{tcolorbox}
 
            
\prompt{Out}{outcolor}{134}{}
    
    $\displaystyle \left[\begin{matrix}30\\44\\63\end{matrix}\right]$

    

    Recordemos que el primer elemento será la primer componente del vector

    El producto escalar: \[
\mathbf{a} \cdot \mathbf{b} \equiv a_x b_x+a_y b_y+a_z b_z \equiv\|\mathbf{a}\|\|\mathbf{b}\| \cos (\varphi) \in \mathbb{R}
\]

    \begin{tcolorbox}[breakable, size=fbox, boxrule=1pt, pad at break*=1mm,colback=cellbackground, colframe=cellborder]
\prompt{In}{incolor}{135}{\boxspacing}
\begin{Verbatim}[commandchars=\\\{\}]
\PY{n}{a}\PY{o}{.}\PY{n}{dot}\PY{p}{(}\PY{n}{b}\PY{p}{)}
\end{Verbatim}
\end{tcolorbox}
 
            
\prompt{Out}{outcolor}{135}{}
    
    $\displaystyle 92$

    

    \begin{tcolorbox}[breakable, size=fbox, boxrule=1pt, pad at break*=1mm,colback=cellbackground, colframe=cellborder]
\prompt{In}{incolor}{136}{\boxspacing}
\begin{Verbatim}[commandchars=\\\{\}]
\PY{n}{b}\PY{o}{.}\PY{n}{dot}\PY{p}{(}\PY{n}{a}\PY{p}{)}
\end{Verbatim}
\end{tcolorbox}
 
            
\prompt{Out}{outcolor}{136}{}
    
    $\displaystyle 92$

    

    El ángulo entre los vectores \[
\varphi= \arccos\left[ \frac{\mathbf{a} \cdot \mathbf{b}}{\|\mathbf{a}\|\|\mathbf{b}\|}\right]
\]

    \begin{tcolorbox}[breakable, size=fbox, boxrule=1pt, pad at break*=1mm,colback=cellbackground, colframe=cellborder]
\prompt{In}{incolor}{137}{\boxspacing}
\begin{Verbatim}[commandchars=\\\{\}]
\PY{n}{acos}\PY{p}{(}\PY{n}{a}\PY{o}{.}\PY{n}{dot}\PY{p}{(}\PY{n}{b}\PY{p}{)}\PY{o}{/}\PY{p}{(}\PY{n}{a}\PY{o}{.}\PY{n}{norm}\PY{p}{(}\PY{p}{)}\PY{o}{*}\PY{n}{b}\PY{o}{.}\PY{n}{norm}\PY{p}{(}\PY{p}{)}\PY{p}{)}\PY{p}{)}\PY{o}{.}\PY{n}{round}\PY{p}{(}\PY{l+m+mi}{3}\PY{p}{)} \PY{c+c1}{\PYZsh{} En radianes}
\end{Verbatim}
\end{tcolorbox}
 
            
\prompt{Out}{outcolor}{137}{}
    
    $\displaystyle 0.158$

    

    En grados sería:

    \begin{tcolorbox}[breakable, size=fbox, boxrule=1pt, pad at break*=1mm,colback=cellbackground, colframe=cellborder]
\prompt{In}{incolor}{138}{\boxspacing}
\begin{Verbatim}[commandchars=\\\{\}]
\PY{p}{(}\PY{p}{(}\PY{n}{\PYZus{}}\PY{p}{)}\PY{o}{*}\PY{l+m+mi}{180}\PY{o}{/}\PY{n}{pi}\PY{p}{)}\PY{o}{.}\PY{n}{evalf}\PY{p}{(}\PY{l+m+mi}{4}\PY{p}{)} \PY{c+c1}{\PYZsh{} grados}
\end{Verbatim}
\end{tcolorbox}
 
            
\prompt{Out}{outcolor}{138}{}
    
    $\displaystyle 9.052$

    

    El producto vectorial de dos vectores en 3 dimensiones es: \[
\mathbf{a} \times \mathbf{b}=a_y b_z-a_z b_y,\,\, a_z b_x-a_x b_z, \,\, a_x b_y-a_y b_x
\]

    \begin{tcolorbox}[breakable, size=fbox, boxrule=1pt, pad at break*=1mm,colback=cellbackground, colframe=cellborder]
\prompt{In}{incolor}{139}{\boxspacing}
\begin{Verbatim}[commandchars=\\\{\}]
\PY{n}{a}\PY{o}{.}\PY{n}{cross}\PY{p}{(}\PY{n}{b}\PY{p}{)}
\end{Verbatim}
\end{tcolorbox}
 
            
\prompt{Out}{outcolor}{139}{}
    
    $\displaystyle \left[\begin{matrix}-6\\12\\-6\end{matrix}\right]$

    

    \begin{tcolorbox}[breakable, size=fbox, boxrule=1pt, pad at break*=1mm,colback=cellbackground, colframe=cellborder]
\prompt{In}{incolor}{140}{\boxspacing}
\begin{Verbatim}[commandchars=\\\{\}]
\PY{n}{b}\PY{o}{.}\PY{n}{cross}\PY{p}{(}\PY{n}{a}\PY{p}{)}
\end{Verbatim}
\end{tcolorbox}
 
            
\prompt{Out}{outcolor}{140}{}
    
    $\displaystyle \left[\begin{matrix}6\\-12\\6\end{matrix}\right]$

    

    El producto tripe: \[
\mathbf{c}\cdot \left(\mathbf{a} \times \mathbf{b}\right)
\]

    \begin{tcolorbox}[breakable, size=fbox, boxrule=1pt, pad at break*=1mm,colback=cellbackground, colframe=cellborder]
\prompt{In}{incolor}{141}{\boxspacing}
\begin{Verbatim}[commandchars=\\\{\}]
\PY{p}{(}\PY{n}{a}\PY{o}{.}\PY{n}{cross}\PY{p}{(}\PY{n}{b}\PY{p}{)}\PY{p}{)}\PY{o}{.}\PY{n}{dot}\PY{p}{(}\PY{n}{c}\PY{p}{)}
\end{Verbatim}
\end{tcolorbox}
 
            
\prompt{Out}{outcolor}{141}{}
    
    $\displaystyle 30$

    

    Tenemos otra opción para el cálculo vectorial, esta vez utilizando una
libreria llamada \textbf{sympy.vector}, como se muestra a continuación

    \begin{tcolorbox}[breakable, size=fbox, boxrule=1pt, pad at break*=1mm,colback=cellbackground, colframe=cellborder]
\prompt{In}{incolor}{142}{\boxspacing}
\begin{Verbatim}[commandchars=\\\{\}]
\PY{k+kn}{from} \PY{n+nn}{sympy}\PY{n+nn}{.}\PY{n+nn}{vector} \PY{k+kn}{import} \PY{o}{*}
\PY{n}{R} \PY{o}{=} \PY{n}{CoordSys3D}\PY{p}{(}\PY{l+s+s1}{\PYZsq{}}\PY{l+s+s1}{R}\PY{l+s+s1}{\PYZsq{}}\PY{p}{)}
\end{Verbatim}
\end{tcolorbox}

    \begin{tcolorbox}[breakable, size=fbox, boxrule=1pt, pad at break*=1mm,colback=cellbackground, colframe=cellborder]
\prompt{In}{incolor}{143}{\boxspacing}
\begin{Verbatim}[commandchars=\\\{\}]
\PY{n}{A} \PY{o}{=} \PY{l+m+mi}{2}\PY{o}{*}\PY{n}{R}\PY{o}{.}\PY{n}{i} \PY{o}{+} \PY{l+m+mi}{4}\PY{o}{*}\PY{n}{R}\PY{o}{.}\PY{n}{j} \PY{o}{\PYZhy{}} \PY{l+m+mi}{6}\PY{o}{*}\PY{n}{R}\PY{o}{.}\PY{n}{k}
\PY{n}{B} \PY{o}{=} \PY{n}{R}\PY{o}{.}\PY{n}{i} \PY{o}{\PYZhy{}} \PY{l+m+mi}{3}\PY{o}{*}\PY{n}{R}\PY{o}{.}\PY{n}{j} \PY{o}{+} \PY{l+m+mi}{5}\PY{o}{*}\PY{n}{R}\PY{o}{.}\PY{n}{k}
\PY{n}{C}\PY{o}{=} \PY{n}{R}\PY{o}{.}\PY{n}{i} \PY{o}{+} \PY{n}{R}\PY{o}{.}\PY{n}{j} \PY{o}{+} \PY{n}{R}\PY{o}{.}\PY{n}{k}
\end{Verbatim}
\end{tcolorbox}

    R.i, R.j y R.k representan los vectores unitarios \(i, j, k\)

    \begin{tcolorbox}[breakable, size=fbox, boxrule=1pt, pad at break*=1mm,colback=cellbackground, colframe=cellborder]
\prompt{In}{incolor}{144}{\boxspacing}
\begin{Verbatim}[commandchars=\\\{\}]
\PY{c+c1}{\PYZsh{} El vector A}
\PY{n}{A}
\end{Verbatim}
\end{tcolorbox}
 
            
\prompt{Out}{outcolor}{144}{}
    
    $\displaystyle \left(2\right)\mathbf{\hat{i}_{R}} + \left(4\right)\mathbf{\hat{j}_{R}} + \left(-6\right)\mathbf{\hat{k}_{R}}$

    

    Operaciones básicas con vectores

    \begin{tcolorbox}[breakable, size=fbox, boxrule=1pt, pad at break*=1mm,colback=cellbackground, colframe=cellborder]
\prompt{In}{incolor}{145}{\boxspacing}
\begin{Verbatim}[commandchars=\\\{\}]
\PY{n}{A} \PY{o}{+} \PY{l+m+mi}{2}\PY{o}{*}\PY{n}{B} \PY{o}{\PYZhy{}} \PY{n}{C}
\end{Verbatim}
\end{tcolorbox}
 
            
\prompt{Out}{outcolor}{145}{}
    
    $\displaystyle \left(3\right)\mathbf{\hat{i}_{R}} + \left(-3\right)\mathbf{\hat{j}_{R}} + \left(3\right)\mathbf{\hat{k}_{R}}$

    

    El módulo del vector \(|\vec{A}|=\sqrt{\vec{A}\cdot \vec{A}}\)

    \begin{tcolorbox}[breakable, size=fbox, boxrule=1pt, pad at break*=1mm,colback=cellbackground, colframe=cellborder]
\prompt{In}{incolor}{146}{\boxspacing}
\begin{Verbatim}[commandchars=\\\{\}]
\PY{n}{sqrt}\PY{p}{(}\PY{n}{A}\PY{o}{.}\PY{n}{dot}\PY{p}{(}\PY{n}{A}\PY{p}{)}\PY{p}{)}
\end{Verbatim}
\end{tcolorbox}
 
            
\prompt{Out}{outcolor}{146}{}
    
    $\displaystyle 2 \sqrt{14}$

    

    Aunque tenemos funciones en Sympy para la magnitud

    \begin{tcolorbox}[breakable, size=fbox, boxrule=1pt, pad at break*=1mm,colback=cellbackground, colframe=cellborder]
\prompt{In}{incolor}{147}{\boxspacing}
\begin{Verbatim}[commandchars=\\\{\}]
\PY{n}{A}\PY{o}{.}\PY{n}{magnitude}\PY{p}{(}\PY{p}{)}
\end{Verbatim}
\end{tcolorbox}
 
            
\prompt{Out}{outcolor}{147}{}
    
    $\displaystyle 2 \sqrt{14}$

    

    El vector unitario asociado a \textbf{A}

    \begin{tcolorbox}[breakable, size=fbox, boxrule=1pt, pad at break*=1mm,colback=cellbackground, colframe=cellborder]
\prompt{In}{incolor}{148}{\boxspacing}
\begin{Verbatim}[commandchars=\\\{\}]
\PY{n}{A}\PY{o}{.}\PY{n}{normalize}\PY{p}{(}\PY{p}{)}
\end{Verbatim}
\end{tcolorbox}
 
            
\prompt{Out}{outcolor}{148}{}
    
    $\displaystyle \left(\frac{\sqrt{14}}{14}\right)\mathbf{\hat{i}_{R}} + \left(\frac{\sqrt{14}}{7}\right)\mathbf{\hat{j}_{R}} + \left(- \frac{3 \sqrt{14}}{14}\right)\mathbf{\hat{k}_{R}}$

    

    \begin{tcolorbox}[breakable, size=fbox, boxrule=1pt, pad at break*=1mm,colback=cellbackground, colframe=cellborder]
\prompt{In}{incolor}{149}{\boxspacing}
\begin{Verbatim}[commandchars=\\\{\}]
\PY{p}{(}\PY{n}{\PYZus{}}\PY{p}{)}\PY{o}{.}\PY{n}{magnitude}\PY{p}{(}\PY{p}{)}
\end{Verbatim}
\end{tcolorbox}
 
            
\prompt{Out}{outcolor}{149}{}
    
    $\displaystyle 1$

    

    El producto escalar \(A.B\)

    \begin{tcolorbox}[breakable, size=fbox, boxrule=1pt, pad at break*=1mm,colback=cellbackground, colframe=cellborder]
\prompt{In}{incolor}{150}{\boxspacing}
\begin{Verbatim}[commandchars=\\\{\}]
\PY{n}{A}\PY{o}{.}\PY{n}{dot}\PY{p}{(}\PY{n}{B}\PY{p}{)}
\end{Verbatim}
\end{tcolorbox}
 
            
\prompt{Out}{outcolor}{150}{}
    
    $\displaystyle -40$

    

    \emph{El} producto vectorial \(A \times B\)

    \begin{tcolorbox}[breakable, size=fbox, boxrule=1pt, pad at break*=1mm,colback=cellbackground, colframe=cellborder]
\prompt{In}{incolor}{151}{\boxspacing}
\begin{Verbatim}[commandchars=\\\{\}]
\PY{n}{A}\PY{o}{.}\PY{n}{cross}\PY{p}{(}\PY{n}{B}\PY{p}{)}
\end{Verbatim}
\end{tcolorbox}
 
            
\prompt{Out}{outcolor}{151}{}
    
    $\displaystyle \left(2\right)\mathbf{\hat{i}_{R}} + \left(-16\right)\mathbf{\hat{j}_{R}} + \left(-10\right)\mathbf{\hat{k}_{R}}$

    

    \begin{tcolorbox}[breakable, size=fbox, boxrule=1pt, pad at break*=1mm,colback=cellbackground, colframe=cellborder]
\prompt{In}{incolor}{152}{\boxspacing}
\begin{Verbatim}[commandchars=\\\{\}]
\PY{n}{B}\PY{o}{.}\PY{n}{cross}\PY{p}{(}\PY{n}{A}\PY{p}{)}
\end{Verbatim}
\end{tcolorbox}
 
            
\prompt{Out}{outcolor}{152}{}
    
    $\displaystyle \left(-2\right)\mathbf{\hat{i}_{R}} + \left(16\right)\mathbf{\hat{j}_{R}} + \left(10\right)\mathbf{\hat{k}_{R}}$

    

    El producto triple \((A\times B) \cdot C\)

    \begin{tcolorbox}[breakable, size=fbox, boxrule=1pt, pad at break*=1mm,colback=cellbackground, colframe=cellborder]
\prompt{In}{incolor}{153}{\boxspacing}
\begin{Verbatim}[commandchars=\\\{\}]
\PY{n}{A}\PY{o}{.}\PY{n}{cross}\PY{p}{(}\PY{n}{B}\PY{p}{)}\PY{o}{.}\PY{n}{dot}\PY{p}{(}\PY{n}{C}\PY{p}{)}
\end{Verbatim}
\end{tcolorbox}
 
            
\prompt{Out}{outcolor}{153}{}
    
    $\displaystyle -24$

    

    Veamos ahora algunos operaciones con matrices

Definamos la matriz A

    \begin{tcolorbox}[breakable, size=fbox, boxrule=1pt, pad at break*=1mm,colback=cellbackground, colframe=cellborder]
\prompt{In}{incolor}{154}{\boxspacing}
\begin{Verbatim}[commandchars=\\\{\}]
\PY{n}{init\PYZus{}printing}\PY{p}{(}\PY{n}{use\PYZus{}unicode}\PY{o}{=}\PY{k+kc}{True}\PY{p}{)}
\PY{n}{A} \PY{o}{=} \PY{n}{Matrix}\PY{p}{(}\PY{p}{[} \PY{p}{[} \PY{l+m+mi}{2}\PY{p}{,}\PY{o}{\PYZhy{}}\PY{l+m+mi}{3}\PY{p}{,}\PY{o}{\PYZhy{}}\PY{l+m+mi}{8}\PY{p}{,} \PY{l+m+mi}{7}\PY{p}{]}\PY{p}{,} \PY{p}{[}\PY{o}{\PYZhy{}}\PY{l+m+mi}{2}\PY{p}{,}\PY{o}{\PYZhy{}}\PY{l+m+mi}{2}\PY{p}{,} \PY{l+m+mi}{2}\PY{p}{,}\PY{o}{\PYZhy{}}\PY{l+m+mi}{7}\PY{p}{]}\PY{p}{,}\PY{p}{[} \PY{l+m+mi}{1}\PY{p}{,} \PY{l+m+mi}{0}\PY{p}{,}\PY{o}{\PYZhy{}}\PY{l+m+mi}{5}\PY{p}{,} \PY{l+m+mi}{6}\PY{p}{]} \PY{p}{]}\PY{p}{)}
\PY{n}{A}
\end{Verbatim}
\end{tcolorbox}
 
            
\prompt{Out}{outcolor}{154}{}
    
    $\displaystyle \left[\begin{matrix}2 & -3 & -8 & 7\\-2 & -2 & 2 & -7\\1 & 0 & -5 & 6\end{matrix}\right]$

    

    La función ``init\_printing(use\_unicode=True)'' configura la salida de
las expresiones simbólicas para que se muestren de manera más legible
cuando se imprimen en la consola o en entornos interactivos como Jupyter
Notebook.

Cuando se agrega ``use\_unicode=True'', SymPy utiliza caracteres Unicode
para representar símbolos matemáticos como letras griegas, operadores y
símbolos especiales. Esto mejora significativamente la legibilidad de
las expresiones matemáticas impresas.

    \begin{tcolorbox}[breakable, size=fbox, boxrule=1pt, pad at break*=1mm,colback=cellbackground, colframe=cellborder]
\prompt{In}{incolor}{155}{\boxspacing}
\begin{Verbatim}[commandchars=\\\{\}]
\PY{n}{A}\PY{p}{[}\PY{l+m+mi}{0}\PY{p}{,}\PY{l+m+mi}{1}\PY{p}{]} \PY{c+c1}{\PYZsh{} fila 0, columna 1 de A}
\end{Verbatim}
\end{tcolorbox}
 
            
\prompt{Out}{outcolor}{155}{}
    
    $\displaystyle -3$

    

    \begin{tcolorbox}[breakable, size=fbox, boxrule=1pt, pad at break*=1mm,colback=cellbackground, colframe=cellborder]
\prompt{In}{incolor}{156}{\boxspacing}
\begin{Verbatim}[commandchars=\\\{\}]
\PY{n}{A}\PY{p}{[}\PY{l+m+mi}{0}\PY{p}{:}\PY{l+m+mi}{2}\PY{p}{,} \PY{l+m+mi}{0}\PY{p}{:}\PY{l+m+mi}{3}\PY{p}{]} \PY{c+c1}{\PYZsh{} submatriz 2x3 de A}
\end{Verbatim}
\end{tcolorbox}
 
            
\prompt{Out}{outcolor}{156}{}
    
    $\displaystyle \left[\begin{matrix}2 & -3 & -8\\-2 & -2 & 2\end{matrix}\right]$

    

    La matiz identidad

    \begin{tcolorbox}[breakable, size=fbox, boxrule=1pt, pad at break*=1mm,colback=cellbackground, colframe=cellborder]
\prompt{In}{incolor}{157}{\boxspacing}
\begin{Verbatim}[commandchars=\\\{\}]
\PY{n}{eye}\PY{p}{(}\PY{l+m+mi}{4}\PY{p}{)} \PY{c+c1}{\PYZsh{} 4x4}
\end{Verbatim}
\end{tcolorbox}
 
            
\prompt{Out}{outcolor}{157}{}
    
    $\displaystyle \left[\begin{matrix}1 & 0 & 0 & 0\\0 & 1 & 0 & 0\\0 & 0 & 1 & 0\\0 & 0 & 0 & 1\end{matrix}\right]$

    

    \begin{tcolorbox}[breakable, size=fbox, boxrule=1pt, pad at break*=1mm,colback=cellbackground, colframe=cellborder]
\prompt{In}{incolor}{158}{\boxspacing}
\begin{Verbatim}[commandchars=\\\{\}]
\PY{n}{zeros}\PY{p}{(}\PY{l+m+mi}{2}\PY{p}{,} \PY{l+m+mi}{3}\PY{p}{)} \PY{c+c1}{\PYZsh{} 2x3}
\end{Verbatim}
\end{tcolorbox}
 
            
\prompt{Out}{outcolor}{158}{}
    
    $\displaystyle \left[\begin{matrix}0 & 0 & 0\\0 & 0 & 0\end{matrix}\right]$

    

    \begin{tcolorbox}[breakable, size=fbox, boxrule=1pt, pad at break*=1mm,colback=cellbackground, colframe=cellborder]
\prompt{In}{incolor}{159}{\boxspacing}
\begin{Verbatim}[commandchars=\\\{\}]
\PY{n}{A}\PY{o}{.}\PY{n}{transpose}\PY{p}{(}\PY{p}{)}
\end{Verbatim}
\end{tcolorbox}
 
            
\prompt{Out}{outcolor}{159}{}
    
    $\displaystyle \left[\begin{matrix}2 & -2 & 1\\-3 & -2 & 0\\-8 & 2 & -5\\7 & -7 & 6\end{matrix}\right]$

    

    \begin{tcolorbox}[breakable, size=fbox, boxrule=1pt, pad at break*=1mm,colback=cellbackground, colframe=cellborder]
\prompt{In}{incolor}{160}{\boxspacing}
\begin{Verbatim}[commandchars=\\\{\}]
\PY{n}{B} \PY{o}{=} \PY{n}{Matrix}\PY{p}{(}\PY{p}{[} \PY{p}{[}\PY{o}{\PYZhy{}}\PY{l+m+mi}{24}\PY{p}{,} \PY{l+m+mi}{18}\PY{p}{,} \PY{l+m+mi}{5}\PY{p}{]}\PY{p}{,}\PY{p}{[}\PY{l+m+mi}{20}\PY{p}{,}\PY{o}{\PYZhy{}}\PY{l+m+mi}{15}\PY{p}{,} \PY{o}{\PYZhy{}}\PY{l+m+mi}{4}\PY{p}{]}\PY{p}{,} \PY{p}{[}\PY{o}{\PYZhy{}}\PY{l+m+mi}{5}\PY{p}{,} \PY{l+m+mi}{4}\PY{p}{,} \PY{l+m+mi}{1}\PY{p}{]} \PY{p}{]}\PY{p}{)}
\PY{n}{B}
\end{Verbatim}
\end{tcolorbox}
 
            
\prompt{Out}{outcolor}{160}{}
    
    $\displaystyle \left[\begin{matrix}-24 & 18 & 5\\20 & -15 & -4\\-5 & 4 & 1\end{matrix}\right]$

    

    \begin{tcolorbox}[breakable, size=fbox, boxrule=1pt, pad at break*=1mm,colback=cellbackground, colframe=cellborder]
\prompt{In}{incolor}{161}{\boxspacing}
\begin{Verbatim}[commandchars=\\\{\}]
\PY{n}{B}\PY{o}{.}\PY{n}{det}\PY{p}{(}\PY{p}{)}
\end{Verbatim}
\end{tcolorbox}
 
            
\prompt{Out}{outcolor}{161}{}
    
    $\displaystyle 1$

    

    \begin{tcolorbox}[breakable, size=fbox, boxrule=1pt, pad at break*=1mm,colback=cellbackground, colframe=cellborder]
\prompt{In}{incolor}{162}{\boxspacing}
\begin{Verbatim}[commandchars=\\\{\}]
\PY{n}{B}\PY{o}{.}\PY{n}{inv}\PY{p}{(}\PY{p}{)}
\end{Verbatim}
\end{tcolorbox}
 
            
\prompt{Out}{outcolor}{162}{}
    
    $\displaystyle \left[\begin{matrix}1 & 2 & 3\\0 & 1 & 4\\5 & 6 & 0\end{matrix}\right]$

    

    La función ``rref()'' calcula la forma escalonada reducida por filas de
una matriz. La forma escalonada reducida por filas (en inglés, Reduced
Row Echelon Form o RREF) es una forma canónica de representar una matriz
en álgebra lineal, que simplifica su análisis y facilita la resolución
de sistemas de ecuaciones lineales y otras operaciones.

La forma RREF de una matriz tiene las siguientes propiedades:

\begin{itemize}
\tightlist
\item
  Cada fila no nula comienza con un ``1'' (llamado pivote), y los ``1''
  de cada fila están a la derecha de los ``1'' de las filas superiores.
\item
  En cada columna que contiene un ``1'' (pivote), todas las otras
  entradas son ``0''.
\item
  Las filas con todos ceros están al final, si existen.
\end{itemize}

    \begin{tcolorbox}[breakable, size=fbox, boxrule=1pt, pad at break*=1mm,colback=cellbackground, colframe=cellborder]
\prompt{In}{incolor}{163}{\boxspacing}
\begin{Verbatim}[commandchars=\\\{\}]
\PY{n}{B}\PY{o}{.}\PY{n}{rref}\PY{p}{(}\PY{p}{)}
\end{Verbatim}
\end{tcolorbox}
 
            
\prompt{Out}{outcolor}{163}{}
    
    $\displaystyle \left( \left[\begin{matrix}1 & 0 & 0\\0 & 1 & 0\\0 & 0 & 1\end{matrix}\right], \  \left( 0, \  1, \  2\right)\right)$

    

    La primera primera salida es la forma escalonada reducida de la matriz y
la segunda salida es una tupla de índices de las columnas pivoteables.
Esto significa que las columnas pivotales están en las columnas con
índices 0, 1 y 2 respectivamente. Las columnas que no están en pivots
son columnas libres (que no contienen pivotes).

    Veamos otros ejemplos

    \begin{tcolorbox}[breakable, size=fbox, boxrule=1pt, pad at break*=1mm,colback=cellbackground, colframe=cellborder]
\prompt{In}{incolor}{164}{\boxspacing}
\begin{Verbatim}[commandchars=\\\{\}]
\PY{n}{M} \PY{o}{=} \PY{n}{Matrix}\PY{p}{(}\PY{p}{[}\PY{p}{[}\PY{l+m+mi}{1}\PY{p}{,} \PY{l+m+mi}{2}\PY{p}{,} \PY{l+m+mi}{3}\PY{p}{]}\PY{p}{,} \PY{p}{[}\PY{o}{\PYZhy{}}\PY{l+m+mi}{2}\PY{p}{,} \PY{l+m+mi}{3}\PY{p}{,} \PY{l+m+mi}{1}\PY{p}{]}\PY{p}{,} \PY{p}{[}\PY{o}{\PYZhy{}}\PY{l+m+mi}{5}\PY{p}{,} \PY{l+m+mi}{4}\PY{p}{,} \PY{l+m+mi}{1}\PY{p}{]}\PY{p}{]}\PY{p}{)}
\PY{n}{N} \PY{o}{=} \PY{n}{Matrix}\PY{p}{(}\PY{p}{[}\PY{p}{[}\PY{l+m+mi}{4}\PY{p}{,} \PY{l+m+mi}{5}\PY{p}{,}\PY{l+m+mi}{6}\PY{p}{]}\PY{p}{,} \PY{p}{[}\PY{l+m+mi}{0}\PY{p}{,} \PY{l+m+mi}{7}\PY{p}{,}\PY{l+m+mi}{1}\PY{p}{]}\PY{p}{,} \PY{p}{[}\PY{o}{\PYZhy{}}\PY{l+m+mi}{5}\PY{p}{,} \PY{l+m+mi}{4}\PY{p}{,} \PY{l+m+mi}{1}\PY{p}{]}\PY{p}{]}\PY{p}{)}
\end{Verbatim}
\end{tcolorbox}

    \begin{tcolorbox}[breakable, size=fbox, boxrule=1pt, pad at break*=1mm,colback=cellbackground, colframe=cellborder]
\prompt{In}{incolor}{165}{\boxspacing}
\begin{Verbatim}[commandchars=\\\{\}]
\PY{n}{α} \PY{p}{,}\PY{n}{β} \PY{o}{=} \PY{n}{symbols}\PY{p}{(}\PY{l+s+s1}{\PYZsq{}}\PY{l+s+s1}{α β}\PY{l+s+s1}{\PYZsq{}}\PY{p}{)}
\PY{n}{α}\PY{o}{*}\PY{n}{M} \PY{o}{+} \PY{n}{β}\PY{o}{*}\PY{n}{N}
\end{Verbatim}
\end{tcolorbox}
 
            
\prompt{Out}{outcolor}{165}{}
    
    $\displaystyle \left[\begin{matrix}α + 4 β & 2 α + 5 β & 3 α + 6 β\\- 2 α & 3 α + 7 β & α + β\\- 5 α - 5 β & 4 α + 4 β & α + β\end{matrix}\right]$

    

    \begin{tcolorbox}[breakable, size=fbox, boxrule=1pt, pad at break*=1mm,colback=cellbackground, colframe=cellborder]
\prompt{In}{incolor}{166}{\boxspacing}
\begin{Verbatim}[commandchars=\\\{\}]
\PY{n}{M}\PY{o}{*}\PY{n}{N}
\end{Verbatim}
\end{tcolorbox}
 
            
\prompt{Out}{outcolor}{166}{}
    
    $\displaystyle \left[\begin{matrix}-11 & 31 & 11\\-13 & 15 & -8\\-25 & 7 & -25\end{matrix}\right]$

    

    \begin{tcolorbox}[breakable, size=fbox, boxrule=1pt, pad at break*=1mm,colback=cellbackground, colframe=cellborder]
\prompt{In}{incolor}{167}{\boxspacing}
\begin{Verbatim}[commandchars=\\\{\}]
\PY{n}{M}\PY{o}{*}\PY{o}{*}\PY{l+m+mi}{2}
\end{Verbatim}
\end{tcolorbox}
 
            
\prompt{Out}{outcolor}{167}{}
    
    $\displaystyle \left[\begin{matrix}-18 & 20 & 8\\-13 & 9 & -2\\-18 & 6 & -10\end{matrix}\right]$

    

    \begin{tcolorbox}[breakable, size=fbox, boxrule=1pt, pad at break*=1mm,colback=cellbackground, colframe=cellborder]
\prompt{In}{incolor}{168}{\boxspacing}
\begin{Verbatim}[commandchars=\\\{\}]
\PY{n}{M}\PY{o}{*}\PY{o}{*}\PY{p}{(}\PY{o}{\PYZhy{}}\PY{l+m+mi}{1}\PY{p}{)}
\end{Verbatim}
\end{tcolorbox}
 
            
\prompt{Out}{outcolor}{168}{}
    
    $\displaystyle \left[\begin{matrix}- \frac{1}{14} & \frac{5}{7} & - \frac{1}{2}\\- \frac{3}{14} & \frac{8}{7} & - \frac{1}{2}\\\frac{1}{2} & -1 & \frac{1}{2}\end{matrix}\right]$

    

    \begin{tcolorbox}[breakable, size=fbox, boxrule=1pt, pad at break*=1mm,colback=cellbackground, colframe=cellborder]
\prompt{In}{incolor}{169}{\boxspacing}
\begin{Verbatim}[commandchars=\\\{\}]
\PY{n}{M}\PY{o}{*}\PY{n}{N}\PY{o}{*}\PY{o}{*}\PY{p}{(}\PY{o}{\PYZhy{}}\PY{l+m+mi}{1}\PY{p}{)}
\end{Verbatim}
\end{tcolorbox}
 
            
\prompt{Out}{outcolor}{169}{}
    
    $\displaystyle \left[\begin{matrix}\frac{98}{197} & - \frac{36}{197} & \frac{39}{197}\\\frac{14}{197} & \frac{23}{197} & \frac{90}{197}\\0 & 0 & 1\end{matrix}\right]$

    

    \hypertarget{gruxe1ficos}{%
\section{6) Gráficos}\label{gruxe1ficos}}

    El hecho de que SymPy sea una biblioteca de Python se abren infinitas
posibilidades para hacer gráficas con diferentes complejidades.

Lo más sencillo es con el uso del comando \textbf{Plot}, pero
mostraremos algunos ejemplos más elaborados.

    \begin{tcolorbox}[breakable, size=fbox, boxrule=1pt, pad at break*=1mm,colback=cellbackground, colframe=cellborder]
\prompt{In}{incolor}{170}{\boxspacing}
\begin{Verbatim}[commandchars=\\\{\}]
\PY{n}{f}\PY{o}{=}\PY{l+m+mi}{4}\PY{o}{*}\PY{n}{sin}\PY{p}{(}\PY{l+m+mi}{2}\PY{o}{*}\PY{n}{t}\PY{p}{)}
\PY{n}{plot}\PY{p}{(}\PY{n}{f}\PY{p}{,} \PY{p}{(}\PY{n}{t}\PY{p}{,} \PY{l+m+mi}{0}\PY{p}{,} \PY{l+m+mi}{10}\PY{p}{)}\PY{p}{)}
\end{Verbatim}
\end{tcolorbox}

    \begin{center}
    \adjustimage{max size={0.9\linewidth}{0.9\paperheight}}{06_Intro_Sympy_files/06_Intro_Sympy_267_0.png}
    \end{center}
    { \hspace*{\fill} \\}
    
            \begin{tcolorbox}[breakable, size=fbox, boxrule=.5pt, pad at break*=1mm, opacityfill=0]
\prompt{Out}{outcolor}{170}{\boxspacing}
\begin{Verbatim}[commandchars=\\\{\}]
<sympy.plotting.plot.Plot at 0x11fbecd90>
\end{Verbatim}
\end{tcolorbox}
        
    La siguiente figura es con más elaboración, en este caso queremos
graficar varias funciones

    \begin{tcolorbox}[breakable, size=fbox, boxrule=1pt, pad at break*=1mm,colback=cellbackground, colframe=cellborder]
\prompt{In}{incolor}{171}{\boxspacing}
\begin{Verbatim}[commandchars=\\\{\}]
\PY{n}{f1}\PY{o}{=}\PY{n}{x}\PY{o}{*}\PY{o}{*}\PY{l+m+mi}{2}\PY{o}{\PYZhy{}}\PY{l+m+mi}{4}
\PY{n}{f2}\PY{o}{=}\PY{l+m+mi}{2}\PY{o}{*}\PY{n}{cos}\PY{p}{(}\PY{n}{x}\PY{p}{)}
\PY{n}{f3}\PY{o}{=}\PY{l+m+mi}{2}\PY{o}{*}\PY{n}{sin}\PY{p}{(}\PY{n}{x}\PY{p}{)}
\end{Verbatim}
\end{tcolorbox}

    \begin{tcolorbox}[breakable, size=fbox, boxrule=1pt, pad at break*=1mm,colback=cellbackground, colframe=cellborder]
\prompt{In}{incolor}{172}{\boxspacing}
\begin{Verbatim}[commandchars=\\\{\}]
\PY{c+c1}{\PYZsh{} Tres funciones en una mísma gráfica}
\PY{n}{p} \PY{o}{=} \PY{n}{plot}\PY{p}{(}\PY{n}{f1}\PY{p}{,} \PY{n}{f2}\PY{p}{,} \PY{n}{f3}\PY{p}{,} \PY{p}{(}\PY{n}{x}\PY{p}{,} \PY{o}{\PYZhy{}}\PY{n}{pi}\PY{p}{,} \PY{n}{pi}\PY{p}{)}\PY{p}{,}\PY{n}{show}\PY{o}{=}\PY{k+kc}{False}\PY{p}{)}
\PY{c+c1}{\PYZsh{} Ponemos colores y etiquetamos cada función}
\PY{n}{p}\PY{p}{[}\PY{l+m+mi}{0}\PY{p}{]}\PY{o}{.}\PY{n}{line\PYZus{}color} \PY{o}{=} \PY{l+s+s1}{\PYZsq{}}\PY{l+s+s1}{red}\PY{l+s+s1}{\PYZsq{}}
\PY{n}{p}\PY{p}{[}\PY{l+m+mi}{0}\PY{p}{]}\PY{o}{.}\PY{n}{label} \PY{o}{=} \PY{l+s+s1}{\PYZsq{}}\PY{l+s+s1}{\PYZdl{}f\PYZus{}1(x)\PYZdl{}}\PY{l+s+s1}{\PYZsq{}}
\PY{n}{p}\PY{p}{[}\PY{l+m+mi}{1}\PY{p}{]}\PY{o}{.}\PY{n}{line\PYZus{}color} \PY{o}{=} \PY{l+s+s1}{\PYZsq{}}\PY{l+s+s1}{blue}\PY{l+s+s1}{\PYZsq{}}
\PY{n}{p}\PY{p}{[}\PY{l+m+mi}{1}\PY{p}{]}\PY{o}{.}\PY{n}{label} \PY{o}{=} \PY{l+s+s1}{\PYZsq{}}\PY{l+s+s1}{\PYZdl{}f\PYZus{}2(x)\PYZdl{}}\PY{l+s+s1}{\PYZsq{}}
\PY{n}{p}\PY{p}{[}\PY{l+m+mi}{2}\PY{p}{]}\PY{o}{.}\PY{n}{line\PYZus{}color} \PY{o}{=} \PY{l+s+s1}{\PYZsq{}}\PY{l+s+s1}{green}\PY{l+s+s1}{\PYZsq{}}
\PY{n}{p}\PY{p}{[}\PY{l+m+mi}{2}\PY{p}{]}\PY{o}{.}\PY{n}{label} \PY{o}{=} \PY{l+s+s1}{\PYZsq{}}\PY{l+s+s1}{\PYZdl{}f\PYZus{}3(x)\PYZdl{}}\PY{l+s+s1}{\PYZsq{}}
\PY{c+c1}{\PYZsh{} El título de la gráfica y de los ejes}
\PY{n}{p}\PY{o}{.}\PY{n}{title} \PY{o}{=} \PY{l+s+s1}{\PYZsq{}}\PY{l+s+s1}{Diferentes funciones}\PY{l+s+s1}{\PYZsq{}}
\PY{n}{p}\PY{o}{.}\PY{n}{xlabel} \PY{o}{=} \PY{l+s+s1}{\PYZsq{}}\PY{l+s+s1}{x}\PY{l+s+s1}{\PYZsq{}}
\PY{n}{p}\PY{o}{.}\PY{n}{ylabel} \PY{o}{=} \PY{k+kc}{False}
\PY{c+c1}{\PYZsh{} Agregamos una leyenda}
\PY{n}{p}\PY{o}{.}\PY{n}{legend} \PY{o}{=} \PY{k+kc}{True}
\PY{n}{p}\PY{o}{.}\PY{n}{legend\PYZus{}loc} \PY{o}{=} \PY{l+s+s1}{\PYZsq{}}\PY{l+s+s1}{upper left}\PY{l+s+s1}{\PYZsq{}}
\PY{c+c1}{\PYZsh{} Mostramos la gráfica}
\PY{n}{p}\PY{o}{.}\PY{n}{show}\PY{p}{(}\PY{p}{)}
\end{Verbatim}
\end{tcolorbox}

    \begin{center}
    \adjustimage{max size={0.9\linewidth}{0.9\paperheight}}{06_Intro_Sympy_files/06_Intro_Sympy_270_0.png}
    \end{center}
    { \hspace*{\fill} \\}
    
    Existe una librería muy potente para graficar que se llama
\textbf{matplotlib}

Más información en: https://matplotlib.org/

    \begin{tcolorbox}[breakable, size=fbox, boxrule=1pt, pad at break*=1mm,colback=cellbackground, colframe=cellborder]
\prompt{In}{incolor}{173}{\boxspacing}
\begin{Verbatim}[commandchars=\\\{\}]
\PY{k+kn}{import} \PY{n+nn}{matplotlib}\PY{n+nn}{.}\PY{n+nn}{pyplot} \PY{k}{as} \PY{n+nn}{plt}
\end{Verbatim}
\end{tcolorbox}

    Grafiquemos un conjunto de datos

    \begin{tcolorbox}[breakable, size=fbox, boxrule=1pt, pad at break*=1mm,colback=cellbackground, colframe=cellborder]
\prompt{In}{incolor}{174}{\boxspacing}
\begin{Verbatim}[commandchars=\\\{\}]
\PY{c+c1}{\PYZsh{} Los datos se escriben como una lista [dato1,dato2,...]}
\PY{n}{yn} \PY{o}{=}\PY{p}{[}\PY{l+m+mf}{1.00}\PY{p}{,} \PY{l+m+mf}{1.40}\PY{p}{,} \PY{l+m+mf}{1.51}\PY{p}{,} \PY{l+m+mf}{2.03}\PY{p}{,} \PY{l+m+mf}{2.75}\PY{p}{,} \PY{l+m+mf}{3.59}\PY{p}{,} \PY{l+m+mf}{4.87}\PY{p}{,} \PY{l+m+mf}{5.23}\PY{p}{,} \PY{l+m+mf}{5.44}\PY{p}{,} \PY{l+m+mf}{5.95}\PY{p}{,} \PY{l+m+mf}{6.37}\PY{p}{]}
\PY{n}{xn} \PY{o}{=}\PY{p}{[}\PY{l+m+mf}{0.00}\PY{p}{,} \PY{l+m+mf}{0.10}\PY{p}{,} \PY{l+m+mf}{0.20}\PY{p}{,} \PY{l+m+mf}{0.30}\PY{p}{,} \PY{l+m+mf}{0.40}\PY{p}{,} \PY{l+m+mf}{0.50}\PY{p}{,} \PY{l+m+mf}{0.60}\PY{p}{,} \PY{l+m+mf}{0.70}\PY{p}{,} \PY{l+m+mf}{0.80}\PY{p}{,} \PY{l+m+mf}{0.90}\PY{p}{,} \PY{l+m+mf}{1.00}\PY{p}{]}
\end{Verbatim}
\end{tcolorbox}

    \begin{tcolorbox}[breakable, size=fbox, boxrule=1pt, pad at break*=1mm,colback=cellbackground, colframe=cellborder]
\prompt{In}{incolor}{175}{\boxspacing}
\begin{Verbatim}[commandchars=\\\{\}]
\PY{n}{plt}\PY{o}{.}\PY{n}{plot}\PY{p}{(}\PY{n}{xn}\PY{p}{,} \PY{n}{yn}\PY{p}{,} \PY{l+s+s1}{\PYZsq{}}\PY{l+s+s1}{*}\PY{l+s+s1}{\PYZsq{}}\PY{p}{,}\PY{n}{color}\PY{o}{=}\PY{l+s+s2}{\PYZdq{}}\PY{l+s+s2}{blue}\PY{l+s+s2}{\PYZdq{}}\PY{p}{)}
\PY{n}{plt}\PY{o}{.}\PY{n}{title}\PY{p}{(}\PY{l+s+s1}{\PYZsq{}}\PY{l+s+s1}{Velocidad vs tiempo}\PY{l+s+s1}{\PYZsq{}}\PY{p}{,} \PY{n}{fontsize}\PY{o}{=}\PY{l+m+mi}{20}\PY{p}{)}
\PY{n}{plt}\PY{o}{.}\PY{n}{xlabel}\PY{p}{(}\PY{l+s+s1}{\PYZsq{}}\PY{l+s+s1}{\PYZdl{}t\PYZdl{}  [s]}\PY{l+s+s1}{\PYZsq{}}\PY{p}{,} \PY{n}{fontsize}\PY{o}{=}\PY{l+m+mi}{16}\PY{p}{)}
\PY{n}{plt}\PY{o}{.}\PY{n}{ylabel}\PY{p}{(}\PY{l+s+s1}{\PYZsq{}}\PY{l+s+s1}{\PYZdl{}v(t)\PYZdl{}    [m/s]}\PY{l+s+s1}{\PYZsq{}}\PY{p}{,} \PY{n}{fontsize}\PY{o}{=}\PY{l+m+mi}{16}\PY{p}{)}
\PY{n}{plt}\PY{o}{.}\PY{n}{show}\PY{p}{(}\PY{p}{)}
\end{Verbatim}
\end{tcolorbox}

    \begin{center}
    \adjustimage{max size={0.9\linewidth}{0.9\paperheight}}{06_Intro_Sympy_files/06_Intro_Sympy_275_0.png}
    \end{center}
    { \hspace*{\fill} \\}
    
    A continuación más ejemplos con datos

    \begin{tcolorbox}[breakable, size=fbox, boxrule=1pt, pad at break*=1mm,colback=cellbackground, colframe=cellborder]
\prompt{In}{incolor}{176}{\boxspacing}
\begin{Verbatim}[commandchars=\\\{\}]
\PY{n}{d} \PY{o}{=} \PY{p}{[}\PY{l+m+mf}{15.12}\PY{p}{,} \PY{l+m+mf}{15.10}\PY{p}{,} \PY{l+m+mf}{15.15}\PY{p}{,} \PY{l+m+mf}{15.17}\PY{p}{,} \PY{l+m+mf}{15.14}\PY{p}{,} \PY{l+m+mf}{15.16}\PY{p}{,} \PY{l+m+mf}{15.14}\PY{p}{,} \PY{l+m+mf}{15.12}\PY{p}{,}
     \PY{l+m+mf}{15.12}\PY{p}{,} \PY{l+m+mf}{15.14}\PY{p}{,} \PY{l+m+mf}{15.15}\PY{p}{,} \PY{l+m+mf}{15.14}\PY{p}{,} \PY{l+m+mf}{15.13}\PY{p}{,} \PY{l+m+mf}{15.14}\PY{p}{,} \PY{l+m+mf}{15.13}\PY{p}{,} \PY{l+m+mf}{15.13}\PY{p}{,}
     \PY{l+m+mf}{15.14}\PY{p}{,} \PY{l+m+mf}{15.14}\PY{p}{,} \PY{l+m+mf}{15.14}\PY{p}{,} \PY{l+m+mf}{15.13}\PY{p}{,} \PY{l+m+mf}{15.15}\PY{p}{,} \PY{l+m+mf}{15.13}\PY{p}{,} \PY{l+m+mf}{15.11}\PY{p}{,} \PY{l+m+mf}{15.13}\PY{p}{,} \PY{l+m+mf}{15.15}\PY{p}{]}
\PY{n}{plt}\PY{o}{.}\PY{n}{hist}\PY{p}{(}\PY{n}{d}\PY{p}{,}\PY{n}{edgecolor} \PY{o}{=} \PY{l+s+s2}{\PYZdq{}}\PY{l+s+s2}{white}\PY{l+s+s2}{\PYZdq{}}\PY{p}{,} \PY{n}{bins}\PY{o}{=}\PY{l+m+mi}{8}\PY{p}{)}
\PY{n}{plt}\PY{o}{.}\PY{n}{xlabel}\PY{p}{(}\PY{l+s+s1}{\PYZsq{}}\PY{l+s+s1}{Diámetros}\PY{l+s+s1}{\PYZsq{}}\PY{p}{)}
\PY{n}{plt}\PY{o}{.}\PY{n}{ylabel}\PY{p}{(}\PY{l+s+s1}{\PYZsq{}}\PY{l+s+s1}{Frecuencia}\PY{l+s+s1}{\PYZsq{}}\PY{p}{)}
\PY{n}{plt}\PY{o}{.}\PY{n}{title}\PY{p}{(}\PY{l+s+s2}{\PYZdq{}}\PY{l+s+s2}{Histograma}\PY{l+s+s2}{\PYZdq{}}\PY{p}{)}
\PY{n}{plt}\PY{o}{.}\PY{n}{show}\PY{p}{(}\PY{p}{)}
\end{Verbatim}
\end{tcolorbox}

    \begin{center}
    \adjustimage{max size={0.9\linewidth}{0.9\paperheight}}{06_Intro_Sympy_files/06_Intro_Sympy_277_0.png}
    \end{center}
    { \hspace*{\fill} \\}
    
    \begin{tcolorbox}[breakable, size=fbox, boxrule=1pt, pad at break*=1mm,colback=cellbackground, colframe=cellborder]
\prompt{In}{incolor}{177}{\boxspacing}
\begin{Verbatim}[commandchars=\\\{\}]
\PY{c+c1}{\PYZsh{} Lista de datos con las temeraturas máxima en algún lugar día por día}
\PY{c+c1}{\PYZsh{} Aquí temp1 serán los datos para representar en el eje de las ordenadas (eje y)}
\PY{n}{temp1} \PY{o}{=}\PY{p}{[}\PY{l+m+mf}{18.40}\PY{p}{,} \PY{l+m+mf}{18.45}\PY{p}{,} \PY{l+m+mf}{18.65}\PY{p}{,} \PY{l+m+mf}{18.85}\PY{p}{,} \PY{l+m+mf}{18.85}\PY{p}{,} \PY{l+m+mf}{18.85}\PY{p}{,} \PY{l+m+mf}{19.05}\PY{p}{,} \PY{l+m+mf}{18.95}\PY{p}{,} \PY{l+m+mf}{18.60}\PY{p}{,} \PY{l+m+mf}{18.70}\PY{p}{,} \PY{l+m+mf}{18.70}\PY{p}{,} \PY{l+m+mf}{18.70}\PY{p}{,}
        \PY{l+m+mf}{18.70}\PY{p}{,} \PY{l+m+mf}{18.75}\PY{p}{,} \PY{l+m+mf}{18.90}\PY{p}{,} \PY{l+m+mf}{18.85}\PY{p}{,} \PY{l+m+mf}{18.85}\PY{p}{,} \PY{l+m+mf}{19.00}\PY{p}{,} \PY{l+m+mf}{19.00}\PY{p}{,} \PY{l+m+mf}{19.00}\PY{p}{,} \PY{l+m+mf}{19.10}\PY{p}{,} \PY{l+m+mf}{19.35}\PY{p}{,} \PY{l+m+mf}{19.65}\PY{p}{,} \PY{l+m+mf}{19.55}\PY{p}{,}
        \PY{l+m+mf}{19.55}\PY{p}{,} \PY{l+m+mf}{19.55}\PY{p}{,} \PY{l+m+mf}{19.55}\PY{p}{,} \PY{l+m+mf}{19.55}\PY{p}{,} \PY{l+m+mf}{19.60}\PY{p}{,} \PY{l+m+mf}{19.65}\PY{p}{,} \PY{l+m+mf}{19.40}\PY{p}{,} \PY{l+m+mf}{19.50}\PY{p}{,} \PY{l+m+mf}{19.50}\PY{p}{,} \PY{l+m+mf}{19.50}\PY{p}{,} \PY{l+m+mf}{19.50}\PY{p}{,} \PY{l+m+mf}{19.60}\PY{p}{,}
        \PY{l+m+mf}{19.65}\PY{p}{,} \PY{l+m+mf}{19.95}\PY{p}{,} \PY{l+m+mf}{20.00}\PY{p}{,} \PY{l+m+mf}{20.00}\PY{p}{,} \PY{l+m+mf}{20.00}\PY{p}{,} \PY{l+m+mf}{20.00}\PY{p}{,} \PY{l+m+mf}{20.00}\PY{p}{,} \PY{l+m+mf}{19.90}\PY{p}{,} \PY{l+m+mf}{19.70}\PY{p}{,} \PY{l+m+mf}{19.75}\PY{p}{,} \PY{l+m+mf}{19.75}\PY{p}{,} \PY{l+m+mf}{19.75}\PY{p}{,}
        \PY{l+m+mf}{19.90}\PY{p}{,} \PY{l+m+mf}{19.85}\PY{p}{,} \PY{l+m+mf}{19.90}\PY{p}{,} \PY{l+m+mf}{19.95}\PY{p}{,} \PY{l+m+mf}{19.95}\PY{p}{,} \PY{l+m+mf}{19.95}\PY{p}{,} \PY{l+m+mf}{19.95}\PY{p}{,} \PY{l+m+mf}{20.20}\PY{p}{,} \PY{l+m+mf}{20.20}\PY{p}{,} \PY{l+m+mf}{20.10}\PY{p}{,} \PY{l+m+mf}{20.15}\PY{p}{,} \PY{l+m+mf}{20.25}\PY{p}{,}
        \PY{l+m+mf}{20.25}\PY{p}{,} \PY{l+m+mf}{20.25}\PY{p}{,} \PY{l+m+mf}{20.20}\PY{p}{,} \PY{l+m+mf}{20.35}\PY{p}{,} \PY{l+m+mf}{20.40}\PY{p}{,} \PY{l+m+mf}{20.40}\PY{p}{,} \PY{l+m+mf}{20.25}\PY{p}{,} \PY{l+m+mf}{20.25}\PY{p}{,} \PY{l+m+mf}{20.25}\PY{p}{,} \PY{l+m+mf}{20.20}\PY{p}{,} \PY{l+m+mf}{20.20}\PY{p}{,} \PY{l+m+mf}{20.20}\PY{p}{,}
        \PY{l+m+mf}{20.35}\PY{p}{,} \PY{l+m+mf}{20.20}\PY{p}{,} \PY{l+m+mf}{20.20}\PY{p}{,} \PY{l+m+mf}{20.20}\PY{p}{,} \PY{l+m+mf}{20.20}\PY{p}{,} \PY{l+m+mf}{20.25}\PY{p}{,} \PY{l+m+mf}{20.30}\PY{p}{,} \PY{l+m+mf}{20.25}\PY{p}{,} \PY{l+m+mf}{20.20}\PY{p}{,} \PY{l+m+mf}{20.20}\PY{p}{,} \PY{l+m+mf}{20.20}\PY{p}{,} \PY{l+m+mf}{20.20}\PY{p}{,}
        \PY{l+m+mf}{20.20}\PY{p}{,} \PY{l+m+mf}{20.15}\PY{p}{,} \PY{l+m+mf}{20.15}\PY{p}{,} \PY{l+m+mf}{20.15}\PY{p}{,} \PY{l+m+mf}{20.15}\PY{p}{,} \PY{l+m+mf}{20.15}\PY{p}{,} \PY{l+m+mf}{20.15}\PY{p}{,} \PY{l+m+mf}{20.15}\PY{p}{,} \PY{l+m+mf}{20.20}\PY{p}{,} \PY{l+m+mf}{20.20}\PY{p}{,} \PY{l+m+mf}{20.20}\PY{p}{,} \PY{l+m+mf}{20.20}\PY{p}{,}
        \PY{l+m+mf}{20.20}\PY{p}{,} \PY{l+m+mf}{20.20}\PY{p}{,} \PY{l+m+mf}{20.20}\PY{p}{,} \PY{l+m+mf}{20.15}\PY{p}{,} \PY{l+m+mf}{20.20}\PY{p}{,} \PY{l+m+mf}{20.20}\PY{p}{,} \PY{l+m+mf}{20.20}\PY{p}{,} \PY{l+m+mf}{20.20}\PY{p}{,} \PY{l+m+mf}{20.20}\PY{p}{,} \PY{l+m+mf}{20.20}\PY{p}{,} \PY{l+m+mf}{20.20}\PY{p}{,} \PY{l+m+mf}{20.15}\PY{p}{,}
        \PY{l+m+mf}{20.20}\PY{p}{,} \PY{l+m+mf}{20.20}\PY{p}{,} \PY{l+m+mf}{20.20}\PY{p}{,} \PY{l+m+mf}{20.25}\PY{p}{,} \PY{l+m+mf}{20.25}\PY{p}{,} \PY{l+m+mf}{20.25}\PY{p}{,} \PY{l+m+mf}{20.55}\PY{p}{,} \PY{l+m+mf}{20.55}\PY{p}{,} \PY{l+m+mf}{20.55}\PY{p}{,} \PY{l+m+mf}{20.55}\PY{p}{,} \PY{l+m+mf}{20.55}\PY{p}{,} \PY{l+m+mf}{20.55}\PY{p}{,}
        \PY{l+m+mf}{21.20}\PY{p}{,} \PY{l+m+mf}{23.00}\PY{p}{,} \PY{l+m+mf}{21.80}\PY{p}{,} \PY{l+m+mf}{21.80}\PY{p}{,} \PY{l+m+mf}{21.80}\PY{p}{,} \PY{l+m+mf}{21.90}\PY{p}{,} \PY{l+m+mf}{22.40}\PY{p}{,} \PY{l+m+mf}{22.70}\PY{p}{,} \PY{l+m+mf}{22.70}\PY{p}{,} \PY{l+m+mf}{23.20}\PY{p}{,} \PY{l+m+mf}{23.20}\PY{p}{,} \PY{l+m+mf}{23.20}\PY{p}{,}
        \PY{l+m+mf}{24.80}\PY{p}{,} \PY{l+m+mf}{24.00}\PY{p}{,} \PY{l+m+mf}{24.30}\PY{p}{,} \PY{l+m+mf}{24.30}\PY{p}{,} \PY{l+m+mf}{24.40}\PY{p}{,} \PY{l+m+mf}{24.40}\PY{p}{,} \PY{l+m+mf}{24.40}\PY{p}{,} \PY{l+m+mf}{24.40}\PY{p}{,} \PY{l+m+mf}{24.30}\PY{p}{,} \PY{l+m+mf}{24.40}\PY{p}{,} \PY{l+m+mf}{24.60}\PY{p}{,} \PY{l+m+mf}{24.60}\PY{p}{,}
        \PY{l+m+mf}{24.60}\PY{p}{,} \PY{l+m+mf}{24.60}\PY{p}{,} \PY{l+m+mf}{24.70}\PY{p}{,} \PY{l+m+mf}{24.90}\PY{p}{,} \PY{l+m+mf}{24.90}\PY{p}{,} \PY{l+m+mf}{24.90}\PY{p}{,} \PY{l+m+mf}{24.90}\PY{p}{,} \PY{l+m+mf}{24.90}\PY{p}{,} \PY{l+m+mf}{24.90}\PY{p}{,} \PY{l+m+mf}{24.90}\PY{p}{,} \PY{l+m+mf}{24.90}\PY{p}{,} \PY{l+m+mf}{24.90}\PY{p}{,}
        \PY{l+m+mf}{24.90}\PY{p}{,} \PY{l+m+mf}{25.30}\PY{p}{,} \PY{l+m+mf}{25.30}\PY{p}{,} \PY{l+m+mf}{25.30}\PY{p}{,} \PY{l+m+mf}{26.00}\PY{p}{,} \PY{l+m+mf}{25.80}\PY{p}{,} \PY{l+m+mf}{26.00}\PY{p}{,} \PY{l+m+mf}{27.70}\PY{p}{,} \PY{l+m+mf}{28.30}\PY{p}{,} \PY{l+m+mf}{28.30}\PY{p}{,} \PY{l+m+mf}{28.30}\PY{p}{,} \PY{l+m+mf}{27.60}\PY{p}{,}
        \PY{l+m+mf}{27.70}\PY{p}{,} \PY{l+m+mf}{27.70}\PY{p}{,} \PY{l+m+mf}{27.50}\PY{p}{,} \PY{l+m+mf}{27.00}\PY{p}{,} \PY{l+m+mf}{27.00}\PY{p}{,} \PY{l+m+mf}{27.00}\PY{p}{,} \PY{l+m+mf}{27.00}\PY{p}{,} \PY{l+m+mf}{27.10}\PY{p}{,} \PY{l+m+mf}{27.40}\PY{p}{,} \PY{l+m+mf}{28.10}\PY{p}{,} \PY{l+m+mf}{28.90}\PY{p}{,} \PY{l+m+mf}{28.90}\PY{p}{,}
        \PY{l+m+mf}{28.90}\PY{p}{,} \PY{l+m+mf}{28.30}\PY{p}{,} \PY{l+m+mf}{27.80}\PY{p}{,} \PY{l+m+mf}{28.10}\PY{p}{,} \PY{l+m+mf}{28.00}\PY{p}{,} \PY{l+m+mf}{27.90}\PY{p}{,} \PY{l+m+mf}{27.90}\PY{p}{,} \PY{l+m+mf}{27.90}\PY{p}{,} \PY{l+m+mf}{27.90}\PY{p}{,} \PY{l+m+mf}{27.30}\PY{p}{,} \PY{l+m+mf}{27.40}\PY{p}{,} \PY{l+m+mf}{27.20}\PY{p}{,}
        \PY{l+m+mf}{27.20}\PY{p}{,} \PY{l+m+mf}{27.20}\PY{p}{,} \PY{l+m+mf}{27.20}\PY{p}{,} \PY{l+m+mf}{27.30}\PY{p}{,} \PY{l+m+mf}{27.50}\PY{p}{,} \PY{l+m+mf}{27.60}\PY{p}{,} \PY{l+m+mf}{27.70}\PY{p}{,} \PY{l+m+mf}{27.60}\PY{p}{,} \PY{l+m+mf}{27.60}\PY{p}{,} \PY{l+m+mf}{27.60}\PY{p}{,} \PY{l+m+mf}{27.60}\PY{p}{,} \PY{l+m+mf}{27.50}\PY{p}{,}
        \PY{l+m+mf}{27.40}\PY{p}{,} \PY{l+m+mf}{27.40}\PY{p}{,} \PY{l+m+mf}{27.40}\PY{p}{,} \PY{l+m+mf}{27.40}\PY{p}{,} \PY{l+m+mf}{27.40}\PY{p}{,} \PY{l+m+mf}{27.30}\PY{p}{,} \PY{l+m+mf}{27.40}\PY{p}{,} \PY{l+m+mf}{27.50}\PY{p}{,} \PY{l+m+mf}{27.50}\PY{p}{,} \PY{l+m+mf}{27.30}\PY{p}{,} \PY{l+m+mf}{27.30}\PY{p}{,} \PY{l+m+mf}{27.30}\PY{p}{,}
        \PY{l+m+mf}{27.40}\PY{p}{,} \PY{l+m+mf}{27.40}\PY{p}{,} \PY{l+m+mf}{27.60}\PY{p}{,} \PY{l+m+mf}{28.10}\PY{p}{,} \PY{l+m+mf}{29.20}\PY{p}{,} \PY{l+m+mf}{29.20}\PY{p}{,} \PY{l+m+mf}{29.20}\PY{p}{,} \PY{l+m+mf}{30.00}\PY{p}{,} \PY{l+m+mf}{29.60}\PY{p}{,} \PY{l+m+mf}{30.00}\PY{p}{,} \PY{l+m+mf}{29.80}\PY{p}{,} \PY{l+m+mf}{29.80}\PY{p}{,}
        \PY{l+m+mf}{29.80}\PY{p}{,} \PY{l+m+mf}{29.80}\PY{p}{,} \PY{l+m+mf}{29.00}\PY{p}{,} \PY{l+m+mf}{30.00}\PY{p}{,} \PY{l+m+mf}{30.20}\PY{p}{,} \PY{l+m+mf}{30.50}\PY{p}{,} \PY{l+m+mf}{30.90}\PY{p}{,} \PY{l+m+mf}{30.90}\PY{p}{,} \PY{l+m+mf}{30.90}\PY{p}{,} \PY{l+m+mf}{30.90}\PY{p}{,} \PY{l+m+mf}{31.40}\PY{p}{,} \PY{l+m+mf}{32.00}\PY{p}{,}
        \PY{l+m+mf}{32.60}\PY{p}{,} \PY{l+m+mf}{32.80}\PY{p}{,} \PY{l+m+mf}{31.00}\PY{p}{,} \PY{l+m+mf}{30.50}\PY{p}{,} \PY{l+m+mf}{30.40}\PY{p}{,} \PY{l+m+mf}{30.20}\PY{p}{]}
\PY{c+c1}{\PYZsh{} La variable \PYZsq{}dia\PYZsq{} lo representarmos en el eje de las abscisas (eje x):}
\PY{n}{dia} \PY{o}{=} \PY{n+nb}{range}\PY{p}{(}\PY{l+m+mi}{1}\PY{p}{,}\PY{l+m+mi}{247}\PY{p}{)}  \PY{c+c1}{\PYZsh{} días del 1 al 247}
\PY{n}{plt}\PY{o}{.}\PY{n}{plot}\PY{p}{(}\PY{n}{dia}\PY{p}{,} \PY{n}{temp1}\PY{p}{)}
\end{Verbatim}
\end{tcolorbox}

            \begin{tcolorbox}[breakable, size=fbox, boxrule=.5pt, pad at break*=1mm, opacityfill=0]
\prompt{Out}{outcolor}{177}{\boxspacing}
\begin{Verbatim}[commandchars=\\\{\}]
[<matplotlib.lines.Line2D at 0x14e81b890>]
\end{Verbatim}
\end{tcolorbox}
        
    \begin{center}
    \adjustimage{max size={0.9\linewidth}{0.9\paperheight}}{06_Intro_Sympy_files/06_Intro_Sympy_278_1.png}
    \end{center}
    { \hspace*{\fill} \\}
    
    \begin{tcolorbox}[breakable, size=fbox, boxrule=1pt, pad at break*=1mm,colback=cellbackground, colframe=cellborder]
\prompt{In}{incolor}{178}{\boxspacing}
\begin{Verbatim}[commandchars=\\\{\}]
\PY{c+c1}{\PYZsh{} Comparamos las temperaturas anteriores con otros datos de temperaturas de otro lugar}
\PY{n}{temp2} \PY{o}{=} \PY{p}{[}\PY{l+m+mf}{15.57}\PY{p}{,} \PY{l+m+mf}{15.92}\PY{p}{,} \PY{l+m+mf}{16.02}\PY{p}{,} \PY{l+m+mf}{16.14}\PY{p}{,} \PY{l+m+mf}{16.19}\PY{p}{,} \PY{l+m+mf}{16.10}\PY{p}{,} \PY{l+m+mf}{15.90}\PY{p}{,} \PY{l+m+mf}{16.18}\PY{p}{,} \PY{l+m+mf}{16.63}\PY{p}{,}
         \PY{l+m+mf}{16.48}\PY{p}{,} \PY{l+m+mf}{17.26}\PY{p}{,} \PY{l+m+mf}{17.01}\PY{p}{,} \PY{l+m+mf}{17.37}\PY{p}{,} \PY{l+m+mf}{17.63}\PY{p}{]}
\PY{c+c1}{\PYZsh{} ... y un registro del tiempo diferente}
\PY{n}{dia\PYZus{}2} \PY{o}{=} \PY{p}{[}\PY{l+m+mi}{1}\PY{p}{,} \PY{l+m+mi}{18}\PY{p}{,} \PY{l+m+mi}{36}\PY{p}{,} \PY{l+m+mi}{53}\PY{p}{,} \PY{l+m+mi}{71}\PY{p}{,} \PY{l+m+mi}{88}\PY{p}{,} \PY{l+m+mi}{106}\PY{p}{,} \PY{l+m+mi}{123}\PY{p}{,} \PY{l+m+mi}{141}\PY{p}{,} \PY{l+m+mi}{158}\PY{p}{,} \PY{l+m+mi}{176}\PY{p}{,} \PY{l+m+mi}{193}\PY{p}{,} \PY{l+m+mi}{211}\PY{p}{,} \PY{l+m+mi}{228}\PY{p}{]}

\PY{n}{plt}\PY{o}{.}\PY{n}{plot}\PY{p}{(}\PY{n}{dia}  \PY{p}{,} \PY{n}{temp1} \PY{p}{,} \PY{l+s+s1}{\PYZsq{}}\PY{l+s+s1}{\PYZhy{}}\PY{l+s+s1}{\PYZsq{}}  \PY{p}{,} \PY{n}{label}\PY{o}{=}\PY{l+s+s1}{\PYZsq{}}\PY{l+s+s1}{Temp. lugar 1}\PY{l+s+s1}{\PYZsq{}}\PY{p}{)}
\PY{n}{plt}\PY{o}{.}\PY{n}{plot}\PY{p}{(}\PY{n}{dia\PYZus{}2}\PY{p}{,} \PY{n}{temp2} \PY{p}{,} \PY{l+s+s1}{\PYZsq{}}\PY{l+s+s1}{o\PYZhy{}}\PY{l+s+s1}{\PYZsq{}} \PY{p}{,} \PY{n}{color}\PY{o}{=}\PY{l+s+s2}{\PYZdq{}}\PY{l+s+s2}{r}\PY{l+s+s2}{\PYZdq{}}\PY{p}{,} \PY{n}{label}\PY{o}{=}\PY{l+s+s1}{\PYZsq{}}\PY{l+s+s1}{Temp. lugar 2}\PY{l+s+s1}{\PYZsq{}}\PY{p}{)}
\PY{n}{plt}\PY{o}{.}\PY{n}{xlabel}\PY{p}{(}\PY{l+s+s1}{\PYZsq{}}\PY{l+s+s1}{Día del año }\PY{l+s+s1}{\PYZsq{}}\PY{p}{,} \PY{n}{fontsize}\PY{o}{=}\PY{l+m+mi}{16}\PY{p}{)}
\PY{n}{plt}\PY{o}{.}\PY{n}{ylabel}\PY{p}{(}\PY{l+s+s1}{\PYZsq{}}\PY{l+s+s1}{Temperatura [ºC]}\PY{l+s+s1}{\PYZsq{}}\PY{p}{,} \PY{n}{fontsize}\PY{o}{=}\PY{l+m+mi}{16}\PY{p}{)}
\PY{n}{plt}\PY{o}{.}\PY{n}{title}\PY{p}{(}\PY{l+s+s1}{\PYZsq{}}\PY{l+s+s1}{Temperaturas en 2023}\PY{l+s+s1}{\PYZsq{}}\PY{p}{,} \PY{n}{fontsize}\PY{o}{=}\PY{l+m+mi}{20}\PY{p}{)}
\PY{n}{plt}\PY{o}{.}\PY{n}{legend}\PY{p}{(}\PY{p}{)}
\PY{n}{plt}\PY{o}{.}\PY{n}{grid}\PY{p}{(}\PY{k+kc}{True}\PY{p}{)}
\end{Verbatim}
\end{tcolorbox}

    \begin{center}
    \adjustimage{max size={0.9\linewidth}{0.9\paperheight}}{06_Intro_Sympy_files/06_Intro_Sympy_279_0.png}
    \end{center}
    { \hspace*{\fill} \\}
    
    \textbf{F I N}


    % Add a bibliography block to the postdoc
    
    
    
\end{document}
