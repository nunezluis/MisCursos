



En este trabajo los autores afirman 
1. que buscan identificar configuraciones de campos de calibre no abelianos que son capaces de generar las anisotrop�as caracter�sticas de las cosmolog�as de Bianchi sin cizallamiento.  
2. que esta investigaci�n extiende y complementa el an�lisis de Mikel Thorsrud para campos abelianos
3. que aportan nuevos conocimientos sobre modelos cosmol�gicos alternativos que se ajustan a las pruebas observacionales, y que relajan las estrictas restricciones de isotrop�a caracter�sticas de los modelos FLRW convencionales.

A lo largo de casi 100 p�ginas, escritas con un lenguaje verboso y grandilocuente, los autores hacen imperceptibles esas contribuciones. La lectura del texto es dif�cil porque es frecuentemente interrumpida por apartados de "suposiciones" (Assumptions) y "observaciones" (Remarcks), etiquedados de una manera curiosa (philosophycal, Matter, Geometry, Theory).  Los autores deben estar conscientes de lo confusa que es la lectura del texto porque, en vez de incluir un cap�tulo de conclusiones donde contextualicen los resultados y su importancia, incluyen un sumario. Es decir, al final del trabajo, describen (en futuro) que fue lo que se hizo en el trabajo. Ese estilo y el lenguaje dista mucho de lo usual para un reporte cient�fico. 

En el cap�tulo 6 (secci�n 6.3) se presenta la extensi�n a los resultados de Thorsrud para el caso no abeliano. Pero el t�tulo y algunas referencias en esa secci�n, califican de "intentos" (attempts) y no resultados. En ese sentido se debe ser mas claros en la firmeza (positiva o negativa) de los resultados.

Mi recomendaci�n es se�irse en lo posible a la estructura de un reporte cient�fico usual, con un lenguaje sobrio, sin pretenciones filos�ficas. En la introducci�n narrar la pregunta de investigaci�n, su importancia y qu� se ha hecho. Luego en la misma introducci�n anunciar qu� es lo que se va a hacer y donde se encuentra. Finalizar con el listado de resultados, el contexto y su importancia. 

Algunas cosas puntuales que tambi�n dificultan la lectura y comprensi�n del documento
1. Se cita varias veces el teorema 2.1 (pg 25 y pg 27) y ese teorema no aparece. 
2. Hay inconsistencia en entre la formulaciones 3+1 y  1+3 de la relatividad general. Son enfoques distintos y en texto se usan de forma equivalente. Eso tendr�an que aclararlo y decir por qu� se usan una u otra formulaci�n  
3. En la bibliograf�a citas de formas distintas las versiones ArXiv de los art�culos. Unas veces est�n enlazadas al url y otra veces no. Debe hacerse de la misma forma siempre.

NICOL�S HERN�NDEZ BELTR�N
EXTENDED FLRW MODELS, NON-ABELIAN GAUGE FIELDS AND THE WEAK COSMOLOGICAL PRINCIPLE



ARS 197093.75