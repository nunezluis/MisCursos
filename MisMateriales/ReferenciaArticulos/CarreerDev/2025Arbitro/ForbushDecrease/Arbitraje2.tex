Second Referee Report: Space Weather Journal
Title: Measurements of Forbush decrease events at the centre of the South Atlantic Magnetic Anomaly with Muon detectors'

Dear Editor,
I have carefully reviewed this second version of the manuscript submitted for publication in Space Weather. My recommendation is to accept the manuscript after a minor but important stylistic revision. The scientific objections have been satisfactorily resolved; only a structural edit is required before final acceptance.

I found that my suggestion for the first version has been addressed.
+ A detailed, self-contained description of the detector hardware has been added (schematics, SiPM biasing, readout chain).
+ The manuscript now situates its results within the broader literature on muon-flux studies, the South Atlantic Magnetic Anomaly and space-weather forecasting.
+ The revised Section 4.2 motivates and demonstrates the Truncated Time-Shift test with adequate mathematical detail and benchmarking.
+ A clearer discussion now links geomagnetic indices, cosmic-ray transport and the observed muon-rate variations.

My primary concern with this version is the manuscript's poor, fragmented structure.

The authors have divided almost every argument into micro-sections, subsections, and even sub-subsections, each containing only a few sentences. 
Examples include:
1. The Introduction is split into "1.1 The South Atlantic Magnetic Anomaly," "1.2 Detector Overview," etc., fragmenting what should be a continuous lead-in.
2. The Methods section reaches a third level (e.g., "3.3.1 Atmospheric Pressure Correction") for topics that occupy only a paragraph.
3. The Conclusions are broken into "6.1 Utility of Muon Detectors," "6.2 Low-Cost Instrumentation," and "6.3 Future Work," diluting the take-home message.

Such micro-segmentation contradicts the editorial guidelines of most journals?including Space Weather?which discourage sub-headings in the Introduction and Conclusion and advise restraint in hierarchical depth. The current layout interrupts the reading flow and forces the reader to jump between very short fragments.

