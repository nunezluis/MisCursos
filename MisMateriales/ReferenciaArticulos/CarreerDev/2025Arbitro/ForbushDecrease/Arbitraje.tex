Referee Report ? Space Weather Journal
Title: Measurements of Forbush decrease events at the centre of the South Atlantic Magnetic Anomaly with Muon detectors'

Dear Editor,
I have carefully reviewed the manuscript submitted for publication in Space Weather. The topic?investigating cosmic-ray-induced variations through muon detection?falls well within the journal's scope and is particularly timely, given the increasing interest in muon-based proxies for heliospheric modulation. However, in its current form, the manuscript does not meet the standards expected for publication in this journal.

While the subject is relevant and potentially valuable to the Space Weather readership, the work lacks the necessary depth, rigour, and clarity. In particular, the paper would benefit from a more thorough and self-contained presentation of the instrumentation, a stronger connection to the existing body of literature on muon flux variations and Forbush Decreases, and a more detailed justification and evaluation of the proposed TTS method. Moreover, the English language usage throughout the manuscript requires substantial revision for readability and precision.

Additionally, the manuscript would greatly benefit from a deeper and more critical physics-based discussion, especially concerning the underlying mechanisms that link space weather phenomena to muon flux variations. This would help anchor the results within a stronger theoretical framework and enhance the scientific impact of the study.

I recommend that the authors undertake substantial revisions before the manuscript can be evaluated further. Specifically, I encourage the authors to:
+ Please provide a detailed and self-contained description of the detector and its electronic architecture, including the influence of environmental variables such as pressure and temperature on the count rate.
+ Incorporate a comprehensive and critical review of prior work, particularly the numerous studies linking muon flux to solar modulation effects, including Forbush Decreases.
+ Clarify the novelty and implementation of the TTS method and benchmark it against established techniques.
+ Thoroughly revise the manuscript for language, grammar, and structural coherence.

I look forward to reviewing a substantially improved version of the manuscript.

Below, I detail my main concerns:

+ Insufficient description of instrumentation. The manuscript offers only a minimal and incomplete description of the instrumentation. It relies heavily on references to prior works ?in Spanish? without presenting clear, self-contained explanations. Essential components such as schematics, block diagrams, or concise system architecture summaries are missing. A thorough and accessible description of the electronic design and overall setup ?including the acquisition chain, signal processing stages, triggering mechanism, and data storage workflow? would significantly enhance the reader's understanding of the detector's capabilities and limitations. Additionally, the manuscript lacks a rigorous analysis of how environmental variables, such as atmospheric pressure and temperature, influence the muon count rate. This superficiality is critical, as such correlations are fundamental to the reliable interpretation of flux variations in space weather studies.

+ Lack of engagement with prior literature. The manuscript shows little engagement with the extensive body of research accumulated over nearly two decades on the modulation of muon flux by Forbush Decreases. Notably, the authors do not reference or discuss early pioneering studies conducted in the South Atlantic Anomaly region (e.g., https://doi.org/10.1590/S0102-261X2007000600018). To properly situate their work within the broader scientific context, the authors should incorporate a more comprehensive and critical review of the existing literature. A more profound synthesis is essential to highlight the study's relevance and demonstrate how it advances or complements the current understanding in the field.
Unclear Novelty and Insufficient Justification of the TTS Method. The manuscript introduces a Time-Tagging System (TTS) as a novel temporal muon flux analysis approach. However, quantitative evidence does not clearly articulate or support this method's conceptual motivation and practical advantages over well-established techniques, such as conventional time binning or flux smoothing. Key questions remain unanswered: What specific value does TTS add to space weather analysis? How does it perform regarding calibration, timing resolution, or reliability when compared to standard methods? Without addressing these aspects or providing benchmarks and validation data, the claimed methodological innovation lacks clarity and scientific justification.

+ Language and style require substantial editing. Although the manuscript is generally comprehensible, the quality of the English requires significant improvement. Numerous grammatical errors and stylistic issues negatively affect both readability and the scientific tone. Several sections are marked by wordiness, vague phrasing, and ambiguous sentence structures, sometimes obscuring the intended meaning. A thorough language review is necessary to enhance the text's clarity, precision, and professionalism. Examples include:
  + Verb tense inconsistencies (e.g., using present tense to describe results from a previous calculation/analysis/measurements) 
    + Lines 190-191. The present tense ("is") describes results from a previous calculation.
    + Lines 235-240. Present tense verbs "show" and "indicate" are mixed with past events ("was being prepared", "occurred ", "added").
    + Lines 241-244. Present tense ("can be seen", "is", "gives") describing results of past analysis.
  + Ambiguous phrasing and syntax hinder clarity.
    + Lines 235-240. "overlay" is casual terminology; it is unclear whether it is a plot, comparison, or mathematical operation. "with r = 50 and l = 2" needs contextualization (e.g., define r and l).
    + Lines 254-257. The phrase "can be used to detect muon flux changes related to space weather events" is redundant given the prior context and lacks precision regarding what types of events or data features it refers to.
