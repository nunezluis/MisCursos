Report on: "Monte Carlo simulations of nanosecond electromagnetic pulse interaction with field-aligned ionospheric plasma density irregularities" by Kirillin et al.

This work develops a Monte-Carlo, frequency-domain ray-tracing model for the propagation of a 1-ns ultra-wideband electromagnetic pulse through an ionospheric layer containing field-aligned density depletions (AIT-like striations). The authors reconstruct the time-domain pulse after scattering and compare it to propagation in uniform plasma and in an effective-index "no scattering" approximation. They find that scattering is strongly frequency-dependent (dominant at sub-GHz), anisotropic, and that for realistic AIT parameters, its net effect on the main pulse body is weak relative to dispersion. 

The paper is publishable in the Journal of Geophysical Research - Space Physics. It is well written and methodologically engaging. The approach is novel for the ns-UWB + FAIs scenario, the implementation is clearly grounded in prior radiation-transfer MC practice, and the main physical conclusions are plausible. Improving the conclusion section by providing context and a validity-dominating statement will enrich the presentation of this work. 

My primary concerns about this method are :
1. The geometric-optics restriction may remove the most strongly scattered part of the spectrum. The ray model requires striation radius (\gg \lambda), so the first 10 harmonics (10?100 MHz) are excluded. The authors argue that energy loss from this filtering is < 0.2% and negligible for the main pulse body. However, low frequencies are precisely where their results show the most significant scattering and path-length dispersion. Excluding them may underestimate:
a) The full delay spread / late-time coda,
b) Off-axis spectral redistribution,
c) Any low-frequency observational signatures relevant to diagnostics.
Even if total energy is tiny, phase-sensitive metrics can be affected. This limitation is fundamental and should be highlighted more strongly in the conclusions (and perhaps quantified with a sensitivity test).

2. Idealised irregularity and tailor-made ensemble (single-scale cylinders, Poisson spacing, constant Z_{layer} and <l>). All striations have identical radii and depletion depths, with random spacings drawn from a mean-free-path distribution. Real FAIs/AIT exhibit a spectrum of transverse scales and depths, intermittency, and sometimes clustering. A single-scale Poisson model might bias scattering low at higher frequencies if smaller-scale gradients exist.   I suggest adding a short discussion estimating how a multi-scale spectrum (e.g., power-law transverse sizes) could shift the scattering-vs-frequency boundary. Additionally, what are the effects on the result of the variation of the Z_{layer} and <l>? 

3. Regarding the Open Research section, there is a link to the data used in the plots, but no software is available. 

There are also two minor comments and presentation issues.

1. Clarify the robustness of Monte-Carlo convergence, stating explicitly whether key observables (e.g., central-segment delay, width) were checked for stability vs. N for at least one harmonic set.

2. Provide a compact "validity domain"  paragraph. The mentioned limitations are scattered throughout Section 3.1 and later. A summary of frequency and geometry validity would be helpful to readers.

This paper is novel and likely publishable with only modest changes. The main physical conclusion?dispersion dominates ns-EMP distortion, and realistic AIT striations weakly affect the GHz-band pulse?is acceptable within the stated assumptions, which should be stressed. Strengthening the discussion of geometric-optics cutoff, irregularity statistics, and polarisation neglect would substantially improve transparency and the paper's long-term value as a model.