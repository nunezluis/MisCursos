
The researchers of Wigner RCP (Hungary) had previously carried out the identification of the fracture zone using proprietary high-resolution detectors using a series of 2D tomographic reconstructions of subsurface measurements to approximate the geological structure in 3D in greater depth. The muon telescope used was developed specifically for geological research. Our current study presents recent advances in subsurface measurements and their processing, and new results which have been achieved by full 3D inversion of a large dataset of muographic measurements to determine the geometry of the subsurface fracture and cavity system. The development of detector systems and related software is now ongoing. 

The primary objective of the muographic surveys was to detect the lower density fracture zones and possible cavities in the eastern part of the study area (or possibly the detection of slope debris accumulation), while in the western part a smaller-coverage reconnaissance survey was carried out.

The quality of the measurements and the high angular resolution allow for a full 3D tomographic reconstruction of the density distribution of the screened rock mass down to a resolution of 1.5 meters.


Summary of results
? A unique dense network of muographic measurements was used to map fracture networks and cavities in 35 - 100 m thick cherty dolomite layers at Buda Hill.
? The high-resolution measurement data system allowed for a fine tomographic (meter-scale) reconstruction of the rock density distribution.
? The reconstructed image clearly identifies the dominant elements of a near-vertical, predominantly NE-SW crack system.
? The detected fracture system is consistent with the crack zones identified in the excavations and caves.
? The presence of a larger cavity was also detected by muon tomography.


The adequacy of the measurement position system has been verified by simulations, mainly from the point of view of the reconstructibility of the density distribution. % what type of simulations

The widely used Bayesian discrete tomography methods have been applied to reconstruct the density distribution of the object studied. Details of the theory of the mathematical background, the algorithm and its application in muography can be found in the literature. Many variants of the method are known. In the choice of algorithm, we had to take into account that the nature of the muographic scanning is almost one-sided and the individual muograph measurements cover about a 1.9 steradian cone.

The methodological elements are summarized in a previous study.25

Impresive amount of work: recording data in 600 active days, from 43 detector positions which leads to about 15 million muon tracks processed into more than 800 thousand narrow-angle directional muon intensity measurements

Detection of fractured zones, faults and caves by 3D muon tomography based on high resolution muographic measurements in the Buda Hills, Hungary by Laszlo Balazs, Gergo Hamar, Adam Lajos Csicsek, Gergely Suranyi and Dezso Varga.



You are the referee of a paper entitled Detection of fractured zones, faults and caves by 3D muon tomography based on high-resolution muographic measurements in the Buda Hills, Hungary by Laszlo Balazs, Gergo Hamar, Adam Lajos Csicsek, Gergely Suranyi and Dezso Varga.

Write the referee report based on the following information.

1. The verdict:  This unusual and confusing paper, based on an impressive amount of work (recording data in 600 active days from 43 detector positions, which leads to about 15 million muon tracks processed into more than 800,000 narrow-angle directional muon intensity measurements), is not worth publishing in the present form and needs a profound review of its structure and contents. The study presents recent advances in subsurface measurements and their processing, as well as new results that have been achieved by a full 3D inversion of a large dataset of muographic measurements to determine the geometry of the subsurface fracture and cavity system. 

2. The objective of the article: The primary objective of the muographic surveys was to detect the lower-density fracture zones and possible cavities in the eastern part of the study area (or possibly the Detection of slope debris accumulation). In contrast, a smaller-coverage reconnaissance survey was carried out in the western part. The area of applicability is an area of Mesozoic carbonates (Buda Hills, H�rmashat�rhegy Range in Hungary) with variable topography, where the dominant directions and trends of the fault systems and fracture network are well-known, which allows the tomographic results to be controlled. 

3. The results. The study demonstrates the usefulness of tomography in a fractured carbonate (Triassic cherty dolomite) area where the dominant direction of the fault and fracture system is relatively well-known and can be spatially traced in some places using local outcrops and caves. The information provided by muography can be helpful in geological, hydrogeological modelling, cave exploration, or in identifying geotechnical problems such as the risk of landslides. The summary of the paper results is as follows:
   a. A unique dense network of muographic measurements was used to map fracture networks and cavities in 35 - 100 m thick cherty dolomite layers at Buda Hill. 
   b. The high-resolution measurement data system allowed for the rock density distribution's satisfactory tomographic (meter-scale) reconstruction.
   c. The reconstructed image identifies the dominant elements of a near-vertical, predominantly NE-SW crack system.
   d. The detected fracture system is consistent with the crack zones identified in the excavations and caves.
   e. The presence of a larger cavity was also detected by muon tomography.

4. Important drawbacks
    a Unusual confusing structure of the paper. This strategy of presenting the result following the introduction without explaining how they did the measurements and how they validated them is not appropriate to convince the reader of the validity of the obtained results. 
    b. The authors do not present explicitly the following methodology. Instead, they refer the reader to previous publications of the same research team. This makes the readability challenging to follow the reasoning. Additionally, they mentioned that the development of detector systems and related software is now ongoing, making the results displayed in this work more confusing.
    c. The authors mentioned several times that simmulations had verified the measurements (which should be the other round way). However, they did not show the type of simulations used and the assumptions made when doing these simulations. 
    d. The authors pointed out that the Bayesian discrete tomography methods have been applied to reconstruct the density distribution of the object studied. They refer to the readers interested in the theory of the mathematical background, the algorithm, and its application in muography to the existing literature. To improve the readability, they should include a summary of the theory and techniques used to reconstruct the density.
    
 %%%%%%%%%%%%%%%%%
 # **Referee Report on "Detection of Fractured Zones, Faults, and Caves by 3D Muon Tomography in the Buda Hills, Hungary"**

**Referee: [Your Name]**  
**Date: [Insert Date]**  

## **Summary and Verdict**
This paper presents an extensive study using high-resolution muography to investigate the subsurface fracture and cavity system in the Buda Hills, Hungary. The authors report impressive work, with data collection spanning 600 active days across 43 detector positions, yielding over 15 million muon tracks and more than 800,000 directional muon intensity measurements. Their results demonstrate the potential of muography in geological, hydrogeological, and geotechnical applications. However, despite the scientific merit and the importance of the topic, the manuscript in its current form is not suitable for publication and requires substantial restructuring and content clarification.

## **Main Concerns and Recommended Revisions**
### **1. Structure and Readability**
- The paper follows an unconventional and confusing structure, where the results are introduced before a thorough discussion of the methodology. This approach makes it difficult for the reader to understand the basis of the conclusions.
- **Recommendation**: Reorganize the paper to follow a logical progression:  
  1. Introduction  
  2. Methodology (including measurement techniques, data acquisition, and validation)  
  3. Results  
  4. Discussion  
  5. Conclusion  

### **2. Lack of Explicit Methodology**
- The methodology section is insufficient, as it primarily references previous work by the same research group instead of providing a clear description of the procedures used in this study.
- **Recommendation**: The authors should explicitly detail the following aspects:  
  - **Instrumentation**: Specifications of the muon detectors used.  
  - **Data Processing**: Steps taken to filter, calibrate, and analyze the data.  
  - **Validation**: Methods used to verify the accuracy of the results.  

### **3. Incomplete Discussion of Simulations**
- The manuscript states that simulations were used to verify the measurements, but does not provide details on the simulation framework, input parameters, or assumptions.
- **Recommendation**: The authors should clarify:
  - The type of simulations performed (e.g., Monte Carlo methods, GEANT4, ARTI framework).
  - The input parameters and boundary conditions.
  - A discussion on whether the simulations were used to model the expected muon flux or to validate the density reconstruction.

### **4. Limited Explanation of Bayesian Discrete Tomography**
- The paper mentions that Bayesian discrete tomography was used for density reconstruction but does not summarize the mathematical approach, referring readers to other publications instead.
- **Recommendation**: The authors should include:
  - A brief explanation of the Bayesian approach.
  - The criteria for selecting prior distributions and regularization parameters.
  - A discussion of uncertainties and resolution limitations.

### **5. Scientific Contribution and Novelty**
- The study successfully applies muography in a fractured carbonate region, revealing a near-vertical NE-SW fracture system and identifying a large cavity.
- While the results are valuable, the novelty compared to previous muography studies is not well articulated.
- **Recommendation**: Clearly state how this study advances the field beyond existing literature, possibly by comparing findings with other muographic studies.

## **Conclusion**
The work presented in this paper is scientifically significant and demonstrates the power of muography in subsurface imaging. However, the manuscript requires a major revision to improve clarity, structure, and methodological transparency. Once these concerns are addressed, the paper will be a strong contribution to the field of geological muography.  

**Decision: Major Revision Required**

Referee Report on Detection of Fractured Zones, Faults, and Caves by 3D Muon Tomography in the Buda Hills, Hungary

This paper presents an extensive study using high-resolution muography to investigate the subsurface fracture and cavity system in the Buda Hills, Hungary. The authors report impressive work, with data collection spanning 600 active days across 43 detector positions, yielding over 15 million muon tracks and more than 800,000 directional muon intensity measurements. Their results demonstrate the potential of muography in geological, hydrogeological, and geotechnical applications. However, despite the possible scientific merit and the importance of the topic, the manuscript in its current form is not suitable for publication and requires substantial restructuring and content clarification.   

The main concerns and recommended revisions are
1. Structure and Readability. The paper follows an unconventional and confusing structure, introducing the results before thoroughly discussing the methodology. This approach makes it difficult for the reader to understand the basis of the conclusions. My recommendation is to reorganize the paper to follow a logical progression: Introduction, Methodology (including measurement techniques, data acquisition, and validation), Results, Discussion and Conclusion
    
2. Lack of Explicit Methodology. The methodology section is insufficient, as it primarily references previous work by the same research group instead of clearly describing the procedures used in this study. The authors should explicitly detail the following aspects: the muon detectors' specifications and the steps taken to filter, calibrate, and analyze the data. Validation: Methods used to verify the accuracy of the results.

3. Absence Discussion of Simulations
The manuscript states that simulations were used to verify the measurements but does not provide details on the simulation framework, input parameters, or assumptions. The authors should clarify the simulation type (e.g., Monte Carlo methods, GEANT4, or other simulation framework), input parameters, and boundary conditions. Discuss whether the simulations were used to model the expected muon flux or validate the density reconstruction.

4. Limited Explanation of Bayesian Discrete Tomography. The paper mentions that Bayesian discrete tomography was used for density reconstruction but does not summarize the mathematical approach, referring readers to other publications. The authors should include: A brief explanation of the Bayesian approach, The criteria for selecting prior distributions and regularization parameters, and A better discussion of uncertainties and resolution limitations.

5. Scientific Contribution and Novelty. The study claims to successfully applies muography in a fractured carbonate region, revealing a near-vertical NE-SW fracture system and identifying a large cavity. While the results are valuable, the novelty compared to previous muography studies is not well articulated. The author should clearly state how this study advances the field beyond existing literature by comparing findings with other muographic studies.