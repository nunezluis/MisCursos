\documentclass[spanish]{beamer}
\usepackage[ansinew]{inputenc} % Acepta caracteres en castellano
\usepackage[spanish]{babel}    % silabea palabras castellanas
\usepackage{amsmath}
\usepackage{mathtools,cancel} % cancela con una flecha \cancelto{0}{XXXX}
\renewcommand{\CancelColor}{\color{red}} %change cancel color to red
\usepackage{amsfonts}
\usepackage{amssymb}
\usepackage{dsfont}
\usepackage{graphicx}
\usepackage{geometry}
\usetheme{Madrid}
\usecolortheme{beaver}
\usepackage{textpos}
% Logo  en el comienzo 
\addtobeamertemplate{frametitle}{}{%
\begin{textblock*}{100mm}(.85\textwidth,-1cm)
{\includegraphics[height=0.4in, keepaspectratio=true]{/Users/luisnunez/Dropbox/MisDocumentos/UIS/UISImagenInstitucional/UISLOGO.png}}
\end{textblock*}}

\begin{document}

\title{\textbf{Sistemas integrables y ca�ticos} }
\author[L.A. N��ez]{\textbf{Luis A. N��ez}}  
\institute[UIS]{\textit{Escuela de F�sica, Facultad de Ciencias, } \\
\textit{Universidad Industrial de Santander, Santander, Colombia } \\
{\includegraphics[height=0.4in, keepaspectratio=true]{/Users/luisnunez/Dropbox/MisDocumentos/UIS/UISImagenInstitucional/UISLOGO.png}}
}
\date{\today}
\maketitle


\begin{frame}
\frametitle{Agenda}
  \tableofcontents
\end{frame}


%%%%% Diapo 1
\section{Integrales del movimiento}
\frame{
  \frametitle{Primeras integrales del movimiento}
   \begin{itemize}  
  	\item<1-> Las cantidades conservadas, $I_k\left(q_j, \dot{q}_j\right)=C_k$, donde $C_k=$ constante, y $k=1, \ldots, n$
, son primeras integrales del movimiento de un sistema.
	\item<2-> Un sistema con $s$ grados de libertad es integrable si posee $s$ cantidades conservadas; es decir, si $n=s$.
	\item<3-> A partir de esa integraci�n las dem�s coordenadas pueden, en principio, ser integradas. 
	\item<4-> Un sistema para el cual existen m�s cantidades conservadas que grados de libertad $(n>s)$ se llama superintegrable.
	\item<5-> El ejemplo m�s simple de sistema superintegrable es una part�cula libre. Otro ejemplo es el problema de dos cuerpos sujetos a interacci�n gravitacional.
	\item<6-> Si un sistema con $s$ grados de libertad tiene menos de $s$ cantidades conservadas $(n<s)$, se denomina no integrable.
   \end{itemize}
}
%
%%%%% Diapo 2
\section{Secci�n}
\frame{
  \frametitle{T�tulo transparencia}
   \begin{itemize}
   	\item<1-> Ejemplos de sistemas integrables
	\begin{itemize}
		\item Oscilador arm�nico simple: $s=1 ; C_1=E=$ cte, $n=1$; es integrable.
		\item  P�ndulo simple: $s=1 ; C_1=E=\mathrm{cte}, n=1$; es integrable.
		\item  Part�cula sobre un cono: $s=2, C_1=l_z=$ cte, $C_2=E=$ cte, $n=2$; es integrable.
		\item  P�ndulo doble: $s=2 ; C_1=E=$ cte, $n=1$; no es integrable.
		\item  P�ndulo cuyo soporte gira en un c�rculo en plano vertical con velocidad angular constante: $s=1, n=0$; no es integrable.
		\item  P�ndulo de resorte: $s=2 ; C_1=E=$ cte, $n=1$; no es integrable.
		\item  P�ndulo param�trico cuya longitud var�a en el tiempo: $s=1, n=0$; no es integrable.
		\item  Part�cula libre es superintegrable: $s=3 ; n=4$ : $C_1=E=$ cte, $C_2=p_x=$ cte, $C_2=p_y=\mathrm{cte}, C_2=p_z=\mathrm{cte}$.
	\end{itemize}  
	\item<2-> La integrabilidad es un tipo de simetr�a presente en varios sistemas din�micos, y que conduce a una evoluci�n regular (peri�dica o estacionaria) de las variables del sistema en el tiempo
    \end{itemize}
}
%
%%%%% Diapo 2
\section{Secci�n}
\frame{
  \frametitle{T�tulo transparencia}
   	\begin{itemize}  
  \item<1-> 
    \end{itemize}
}
  
\end{document}

%
%%%%% Diapo 2
\section{Secci�n}
\frame{
  \frametitle{T�tulo transparencia}
   	\begin{itemize}  
  \item<1-> 
    \end{itemize}
}
%
%%%%% Diapo 2
\section{Secci�n}
\frame{
  \frametitle{T�tulo transparencia}
   	\begin{itemize}  
  \item<1-> 
    \end{itemize}
}

