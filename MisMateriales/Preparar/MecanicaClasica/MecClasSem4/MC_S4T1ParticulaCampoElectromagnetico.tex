\documentclass[spanish]{beamer}
\usepackage[ansinew]{inputenc} % Acepta caracteres en castellano
\usepackage[spanish]{babel}    % silabea palabras castellanas
\usepackage{amsmath}
\usepackage{mathtools,cancel} % cancela con una flecha \cancelto{0}{XXXX}
\renewcommand{\CancelColor}{\color{red}} %change cancel color to red
\usepackage{amsfonts}
\usepackage{amssymb}
\usepackage{dsfont}
\usepackage{graphicx}
\usepackage{geometry}
\usetheme{Madrid}
\usecolortheme{beaver}
\usepackage{textpos}
% Logo  en el comienzo 
\addtobeamertemplate{frametitle}{}{%
\begin{textblock*}{100mm}(.85\textwidth,-1cm)
{\includegraphics[height=0.4in, keepaspectratio=true]{/Users/luisnunez/Dropbox/MisDocumentos/UIS/UISImagenInstitucional/UISLOGO.png}}
\end{textblock*}}

\begin{document}

\title{\textbf{Part�cula en un Campo electromagn�tico} }
\author[L.A. N��ez]{\textbf{Luis A. N��ez}}  
\institute[UIS]{\textit{Escuela de F�sica, Facultad de Ciencias, } \\
\textit{Universidad Industrial de Santander, Santander, Colombia } \\
{\includegraphics[height=0.4in, keepaspectratio=true]{/Users/luisnunez/Dropbox/MisDocumentos/UIS/UISImagenInstitucional/UISLOGO.png}}
}
\date{\today}
\maketitle


\begin{frame}
\frametitle{Agenda}
  \tableofcontents
\end{frame}


%%%%% Diapo 1
\section{Fuerza de Lorentz una {\it una fuerza generalizada} }
\frame{
  \frametitle{Fuerza de Lorentz}
   \begin{itemize}  
  	\item<1->   Una part�cula de masa $m$ y carga $q$, movi�ndose con velocidad $\mathbf{v}$, en presencia de un campo el�ctrico $\mathbf{E}(\mathbf{r}, t)$ y un campo magn�tico $\mathbf{B}(\mathbf{r}, t)$, est� sujeta a fuerza de Lorentz
	$\mathbf{F}=q\left(\mathbf{E}+\frac{\mathbf{v}}{c} \times \mathbf{B}\right)$
	\item<2-> Consideremos una part�cula con una energ�a potencial funci�n de su posici�n y velocidad en coordenadas cartesianas: $V(\mathbf{r}, \dot{\mathbf{r}})$. 
	\item<3-> El Lagrangiano, en coordenadas cartesianas, es $L=T(\dot{\mathbf{r}})-V(\mathbf{r}, \dot{\mathbf{r}})$, donde $T(\dot{\mathbf{r}})=\frac{1}{2} m\left(\dot{x}^2+\dot{y}^2+\dot{z}^2\right)=\frac{1}{2} m v^2$
	\item<4-> La ecuaci�n de Lagrange es $\frac{d}{d t}\left(\frac{\partial L}{\partial \dot{x}_i}\right)-\frac{\partial L}{\partial x_i}=0$
	\item<5-> Con lo cual $\frac{d}{d t}\left(\frac{\partial T}{\partial \dot{x_i}}\right.\left.-\frac{\partial V}{\partial \dot{x}_i}\right)+\frac{\partial V}{\partial x_i}=0 \Rightarrow \frac{d}{d t}\left(\frac{\partial T}{\partial \dot{x}_i}\right)=-\frac{\partial V}{\partial x_i}+\frac{d}{d t}\left(\frac{\partial V}{\partial \dot{x_i}}\right) $
	\item<6-> Es decir: $m \ddot{x}_i=F_i \Rightarrow F_i \equiv-\frac{\partial V}{\partial x_i}+\frac{d}{d t}\left(\frac{\partial V}{\partial \dot{x_i}}\right)$
	\item<7-> Las fuerzas generalizadas dependen de coordenadas y de velocidades. 
	\item<8-> La fuerza de Lorentz constituye una fuerza generalizada
    \end{itemize}
}
%%%%% Diapo 2
\section{El potencial vector}
\frame{
  \frametitle{El potencial vector}
   \begin{itemize}  
  	\item<1-> Consideremos las ecuaciones de Maxwell
	$\nabla \cdot \mathbf{E}=4 \pi \rho \qquad \nabla \times \mathbf{E}+\frac{1}{c} \frac{\partial \mathbf{B}}{\partial t}=0$ \\
	$\nabla \cdot \mathbf{B}=0 \qquad \quad \nabla \times \mathbf{B}-\frac{1}{c} \frac{\partial \mathbf{E}}{\partial t}=\frac{4 \pi}{c} \mathbf{J}$
	\item<2-> Claramente $\nabla \cdot \mathbf{B}=0 \Rightarrow \mathbf{B}=\nabla \times \mathbf{A}$, con $\mathbf{A}(\mathbf{r}, t)$ un potencial vector
	\item<3-> Con lo cual 
	$\nabla \times \mathbf{E}+\frac{1}{c} \frac{\partial}{\partial t}(\nabla \times \mathbf{A})=0 \Rightarrow \nabla \times\left(\mathbf{E}+\frac{1}{c} \frac{\partial \mathbf{A}}{\partial t}\right)=0$
	\item<4-> Entonces $\mathbf{E}+\frac{1}{c} \frac{\partial \mathbf{A}}{\partial t}=-\nabla \varphi  \Rightarrow \mathbf{E}=-\nabla \varphi-\frac{1}{c} \frac{\partial \mathbf{A}}{\partial t}$
	\item<5-> La fuerza de Lorentz ser� $\mathbf{F}=q\left[-\nabla \varphi-\frac{1}{c} \frac{\partial \mathbf{A}}{\partial t}+\frac{\mathbf{v}}{c} \times(\nabla \times \mathbf{A})\right]$
	\item<6-> y tambi�n $\mathbf{F}=q\left[-\nabla \varphi+\frac{1}{c} \nabla(\mathbf{A} \cdot \mathbf{v})-\frac{1}{c}(\mathbf{v} \cdot \nabla) \mathbf{A}-\frac{1}{c} \frac{\partial \mathbf{A}}{\partial t}\right]$
	\item<7-> Donde $\mathbf{v} \times(\nabla \times \mathbf{A})=\nabla(\mathbf{A} \cdot \mathbf{v})-(\mathbf{v} \cdot \nabla) \mathbf{A}$
	\item<8-> Hemos utilizado $\mathbf{v} \times(\nabla \times \mathbf{A})=\nabla(\mathbf{A} \cdot \mathbf{v})-(\mathbf{v} \cdot \nabla) \mathbf{A}$

    \end{itemize}
}

%%%%% Diapo 2
\section{Secci�n}
\frame{
  \frametitle{T�tulo transparencia}
   \begin{itemize}  
  \item<1-> 
    \end{itemize}
}
  
\end{document}

%%%%% Diapo 2
\section{Secci�n}
\frame{
  \frametitle{T�tulo transparencia}
   \begin{itemize}  
  \item<1-> 
    \end{itemize}
}
%%%%% Diapo 2
\section{Secci�n}
\frame{
  \frametitle{T�tulo transparencia}
   \begin{itemize}  
  \item<1-> 
    \end{itemize}
}
%%%%% Diapo 2
\section{Secci�n}
\frame{
  \frametitle{T�tulo transparencia}
   \begin{itemize}  
  \item<1-> 
    \end{itemize}
}
%%%%% Diapo 2
\section{Secci�n}
\frame{
  \frametitle{T�tulo transparencia}
   \begin{itemize}  
  \item<1-> 
    \end{itemize}
}
