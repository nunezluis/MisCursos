\documentclass[spanish]{beamer}
\usepackage[ansinew]{inputenc} % Acepta caracteres en castellano
\usepackage[spanish]{babel}    % silabea palabras castellanas
\usepackage{amsmath}
\usepackage{amsfonts}
\usepackage{amssymb}
\usepackage{dsfont}
\usepackage{graphicx}
\usepackage{geometry}
\usetheme{Madrid}
\usecolortheme{beaver}
\usepackage{textpos}
% Logo  en el comienzo 
\addtobeamertemplate{frametitle}{}{%
\begin{textblock*}{100mm}(.85\textwidth,-1cm)
{\includegraphics[height=0.4in, keepaspectratio=true]{/Users/luisnunez/Dropbox/MisDocumentos/UIS/UISImagenInstitucional/UISLOGO.png}}
\end{textblock*}}

\begin{document}

\title{\textbf{Principio de M�nima Acci�n} }
\author[L.A. N��ez]{\textbf{Luis A. N��ez}}  
\institute[UIS]{\textit{Escuela de F�sica, Facultad de Ciencias, } \\
\textit{Universidad Industrial de Santander, Santander, Colombia } \\
{\includegraphics[height=0.4in, keepaspectratio=true]{/Users/luisnunez/Dropbox/MisDocumentos/UIS/UISImagenInstitucional/UISLOGO.png}}
}
\date{\today}
\maketitle


\begin{frame}
\frametitle{Agenda}
  \tableofcontents
\end{frame}


%%%%% Diapo 1
\section{Principio de M�nima Acci�n}
\frame{
  \frametitle{Principio de M�nima Acci�n}
   \begin{itemize}  
  	\item<1-> Consideremos un sistema descrito por $s$ coordenadas generalizadas $\left\{q_1, q_2, \ldots, q_s\right\}$ y sus $s$ velocidades generalizadas $\left\{\dot{q}_1, \dot{q}_2, \ldots, \dot{q}_s\right\}$. 
  	\item<2-> Definimos una funci�n $\mathcal{L}$ de $\left\{q_j\right\},\left\{\dot{q}_j\right\}$ y $t$, de la forma $\mathcal{L}\left(q_j, \dot{q}_j, t\right) \equiv T-V, \quad j=1,2, \ldots, s$, donde $T$ y $V$ son la energ�a cin�tica y la energ�a potencial del sistema, respectivamente.
	\item<3-> El estado del sistema, en $t=t_1$ y $t=t_2$, est� descrito por $t_1:\left\{q_j\left(t_1\right)\right\},\left\{\dot{q}_j\left(t_1\right)\right\}$ y  $t_2:\left\{q_j\left(t_2\right)\right\},\left\{\dot{q}_j\left(t_2\right)\right\}$
	\item<4-> El Principio de m�nima acci�n, implica que la evoluci�n del sistema entre el estado en $t_1$ al $t_2$ es tal que el valor de la integral definida
$S=\int_{t_1}^{t_2} \mathcal{L}\left(q_j, \dot{q}_j, t\right) {\rm d} t,$
denominada la acci�n del sistema, sea m�nima; es decir, $\delta S=0$ ( $S$ es un extremo).
    \end{itemize}
}
%%%%% Diapo 2
\section{Secci�n}
\frame{
  \frametitle{T�tulo transparencia}
   \begin{itemize}  
  \item<1-> 
    \end{itemize}
}

%%%%% Diapo Fin
\section{Recapitulando}
\frame{
  \frametitle{Recapitulando}
En presentaci�n consideramos
  \begin{enumerate}
  \item<1->
    \end{enumerate}
}
  
\end{document}
