\documentclass[12pt]{article}
\usepackage[ansinew]{inputenc} % Acepta caracteres en castellano
\usepackage[spanish]{babel}    % silabea palabras castellanas
\usepackage{amsmath,amsfonts,amssymb}
\usepackage{geometry}
\geometry{margin=2.5cm}


\begin{document}
{\Large {\bf Examen Mec�nica Cl�sica}}

\section*{Instrucciones}
\begin{itemize}
    \item Tiempo: 3 horas
    \item Conteste todas las preguntas.
    \item Justifique todas sus respuestas. Se otorgan puntos completos �nicamente a respuestas completas y correctamente razonadas.
    \item Puede usar constantes f�sicas y matem�ticas est�ndar sin demostraci�n.
\end{itemize}

\section*{Parte I: Preguntas Conceptuales (30 puntos)}

Responda de forma concisa pero rigurosa. Cada pregunta vale 10 puntos.

\begin{enumerate}
    \item \textbf{Fuerzas Centrales y Simetr�as} \\
    Discuta el papel de las leyes de conservaci�n en problemas de fuerzas centrales. En particular, derive las cantidades conservadas a partir de las simetr�as del Lagrangiano y explique su significado f�sico o geom�trico.

    \item \textbf{Modos Normales y Acoplamiento} \\
    Explique por qu� los modos normales de un sistema con peque�as oscilaciones corresponden a ecuaciones de movimiento desacopladas. �C�mo se relacionan los autovectores de la matriz din�mica con el movimiento f�sico del sistema?

    \item \textbf{Tensor de Inercia y Ecuaciones de Euler} \\
    Describa el significado f�sico de los ejes principales de inercia. Derive las ecuaciones de Euler para un cuerpo r�gido en rotaci�n libre y discuta en qu� condiciones el movimiento es estable o inestable.
\end{enumerate}

\section*{Parte II: Problemas Pr�cticos (70 puntos)}

Muestre todo su procedimiento. Cada problema indica su valor en puntos.

\subsection*{Problema 1: Dispersi�n en un Potencial Central (25 puntos)}

Una part�cula de masa \( m \) se aproxima a un centro de fuerza con un potencial atractivo de la forma \( V(r) = -\frac{k}{r} \). La part�cula tiene energ�a \( E > 0 \) y par�metro de impacto \( b \).

\begin{enumerate}
    \item[(a)] Derive la ecuaci�n diferencial para la �rbita \( u(\theta) = \frac{1}{r(\theta)} \).
    \item[(b)] Resuelva la ecuaci�n y encuentre el �ngulo de dispersi�n \( \Theta \).
    \item[(c)] Demuestre que el par�metro de impacto se relaciona con el �ngulo de dispersi�n mediante \( \Theta = 2 \arcsin\left(\frac{k}{L\sqrt{2mE}}\right) \).
\end{enumerate}

\subsection*{Problema 2: Osciladores Acoplados (20 puntos)}

Dos masas \( m \) est�n conectadas por tres resortes id�nticos de constante el�stica \( k \), dispuestos linealmente con los extremos exteriores fijos. Sean \( x_1 \) y \( x_2 \) los desplazamientos respecto al equilibrio.

\begin{enumerate}
    \item[(a)] Escriba el Lagrangiano del sistema.
    \item[(b)] Encuentre las frecuencias de los modos normales y describa los movimientos correspondientes.
\end{enumerate}

\subsection*{Problema 3: El Trompo Libre Sim�trico (25 puntos)}

Considere un trompo sim�trico con momentos de inercia \( I_1 = I_2 \ne I_3 \) y sin torques externos. Sea \( \boldsymbol{\omega} \) el vector de velocidad angular y \( \boldsymbol{L} \) el momento angular.

\begin{enumerate}
    \item[(a)] Escriba las ecuaciones de Euler en el sistema del cuerpo.
    \item[(b)] Resuelva \( \omega_1(t), \omega_2(t) \) suponiendo \( \omega_3 \) constante.
    \item[(c)] Discuta la interpretaci�n geom�trica de este movimiento utilizando la construcci�n de Poinsot.
\end{enumerate}

\end{document}

