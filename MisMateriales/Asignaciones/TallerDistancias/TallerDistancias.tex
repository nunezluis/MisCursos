\documentclass[spanish,notitlepage,letterpaper,12pt]{article} % para articulo en castellano
\usepackage[ansinew]{inputenc} % Acepta caracteres en castellano
\usepackage[spanish]{babel}    % silabea palabras castellanas
%
\usepackage{amsmath}
\usepackage{amsfonts}
\usepackage{amssymb}
\usepackage[colorlinks=true,urlcolor=blue,linkcolor=blue]{hyperref} % navega por el doc
\usepackage{graphicx}
\usepackage{geometry}           % See geometry.pdf to learn the layout options.
\geometry{letterpaper}          % ... or a4paper or a5paper or ... 
%\geometry{landscape}           % Activate for for rotated page geometry
%\usepackage[parfill]{parskip}  % Activate to begin paragraphs with an empty line rather than an indent
\usepackage{epstopdf}
\usepackage{fancyhdr} % encabezados y pies de pg
\usepackage{lineno}
%% The lineno packages adds line numbers. Start line numbering with
%% \begin{linenumbers}, end it with \end{linenumbers}. Or switch it on
%% for the whole article with \linenumbers.
%


\pagestyle{fancy} 
\chead{} 
\lhead{\textit{ M�tricas, datos y calibraci�n }} % si se omite coloca el nombre de la seccion
\rhead{\textbf{M�todos Matematicos}} 
\lfoot{Luis A. N\'u\~nez} 
\rfoot{Escuela de F�sica} 
%\rfoot{\thepage} 

\voffset = -0.25in 
\textheight = 8.0in 
\textwidth = 6.5in
\oddsidemargin = 0.in
\headheight = 20pt 
\headwidth = 6.5in
\renewcommand{\headrulewidth}{0.5pt}
\renewcommand{\footrulewidth}{0,5pt}
\DeclareGraphicsRule{.tif}{png}{.png}{`convert #1 `dirname #1`/`basename #1 .tif`.png}


\title{M�tricas, datos y calibraci�n inteligente}
\author{
\textbf{L. A. N��ez} \\ 
\textit{Escuela de F�sica, Facultad de Ciencias, } 
}

\begin{document}
\maketitle
%\begin{abstract}
%Presentamos
%\end{abstract}
%\newpage
%\tableofcontents

\section{El problema}
Estamos viviendo una �poca de desarrollo explosivo de sensores que pueblan y generan datos en todas las facetas de nuestra cotidianidad. Estos sensores de bajo costo forman parte de dispositivos de la llamada revoluci�n de la \textit{Internet de las cosas, IoT}. Muchas veces estos sensores no son lo suficientemente precisos y deben ser calibrados con un patron de referencia (Para un ejemplo de este tipo de calibraciones inteligente pueden consultar \cite{ZimmermanEtal2018}). 
Este ejercicio busca mostrar que esa calibraci�n est� �ntimamente ligada a la idea de m�trica (pueden consultar \cite{Colombo2020}).

El problema est� en cuantificar cu�l es el error de medici�n del sensor de bajo costo y, como calibrarlo para que podamos establecer nuevas lecturas que sean mas precisas.

En el directorio del pie de p�gina\footnote{\url{https://github.com/nunezluis/MisCursos/tree/main/MetMat1S20B/Asignaciones/DatosDistancias/}} podr�n encontrar los datos de referencia y los de las estaciones \textit{IoT}. El archivo \textit{Datos Estaciones AMB} contiene las medidas de referencia de concentraci�n de material particulado PM$_{2.5}$\footnote{\url{https://blissair.com/what-is-pm-2-5.htm}}, vale decir: concentraci�n de part�culas en suspensi�n de dimensiones $\leq 2.5 \mu$m. Los archivos etiquetados por \textit{mediciones..} contiene los registro de las estaciones de bajo costo.

\section{Una posible estrategia para calcular la distancia}
Para comenzar es importante estimar la distancia entre las medidas de las estaciones de referencia y de las de bajo costo. Para ellos utilizamos la distancia euclidea entre las dos mediciones. 
\begin{equation}
\mathcal{D}(\mathbb{D}_{i}, \hat{\mathbb{D}}_{\hat{i}} ) = \sqrt{ \sum_{i,\hat{i}} \left( \mathbb{D}_{i} - \hat{\mathbb{D}}_{\hat{i}} \right)^{2} }
\label{Distancia}
\end{equation}
Donde hemos definido como $\mathbb{D}_{i} = \left\{ (x_{1}, y_{1}), (x_{2}, y_{2}) \cdots (x_{n}, y_{n}) \right\}$ al conjunto de datos de referencia y como $\hat{\mathbb{D}}_{\hat{i}} = \left\{ (\hat{x}_{1}, \hat{y}_{1}), (\hat{x}_{2}, \hat{y}_{2}) \cdots (\hat{x}_{m}, \hat{y}_{m}) \right\} $ al conjunto de datos a calibrar. 

Claramente, estamos identificando por $f(x_{i}) = y_{i}$ como el valor de la variable dependiente (en este caso la concentraci�n de material particulado) medido por la estaci�n patr�n  y las mediciones de la estaci�n a calibrar por $\hat{f}(\hat{x}_{i}) = \hat{y}_{i}$. Adicionalmente, note que las dimensiones de los dos conjuntos de datos ($i = 1, 2, \cdots n$ e $\hat{i} = 1, 2, \cdots m$) son distintas y, que hemos denotamos por $x_{i} , \hat{x}_{i}$ las variables independientes (en este caso el tiempo) ``mas cercanas'' que intervienen en el c�lculo de la distancia (\ref{Distancia}).

Una posible estrategia para identificar los datos ``mas cercanos'' es utilizar el criterio de el promedio m�vil\footnote{\url{https://en.wikipedia.org/wiki/Moving_average}} y comparar los promedios locales de ambos conjuntos $f(\xi_{j})$ y $\hat{f}(\xi_{j})$, calculados para una ventana com�n $a_{j} \leq x_{i} , \hat{x}_{i} \leq b_{j}$. Donde una posible elecci�n de $\xi_{j}$ puede ser $\xi_{j} = a_{j} + (b_{j} - a_{j})/2$, con $j$ indica el n�mero de ventanas a definir en el rango de variaci�n de los datos. 

Definir el ancho de la ventana para calcular los promedios locales es un arte y requiere de experimentaci�n para balancear el tiempo de c�lculo con la precisi�n lograda.  Entonces, \textbf{calcule la distancia euclidea $\mathcal{D}(\mathbb{D}_{i}, \hat{\mathbb{D}}_{\hat{i}} )$ para varios valores de la ventana m�vil} y determine el mejor de los valores para la ventana.  
 
\section{Una posible estrategia para calibrar las mediciones}
Si graficamos los puntos  $\left( \hat{f}(\xi_{j}), f(\xi_{j}) \right)$, y hacemos un ajuste de m�nimos cuadrados podremos determinar un modelo de ajuste lineal, $f(\xi_{j}) = \alpha \hat{f}(\xi_{j})$. Claramente si ambas estaciones midieran lo mismo, tendr�amos $\alpha = 1$. \\ \textbf{Determine el alcance de validez del modelo lineal}. Esto es, defina una tolerancia\footnote{Error que est� dispuesto a aceptar} y encuentre el alcance en $\hat{x}_{i}$ para la validez su modelo lineal.

Otra estrategia posible es dividir los conjuntos de datos por la mitad, \\ $\mathbb{D}_{i} = \left\{ (x_{1}, y_{1}), (x_{2}, y_{2}) \cdots (x_{\approx n/2}, y_{\approx n/2}) \right\}$ y $\hat{\mathbb{D}}_{\hat{i}} = \left\{ (\hat{x}_{1}, \hat{y}_{1}), (\hat{x}_{2}, \hat{y}_{2}) \cdots (\hat{x}_{\approx n/2}, \hat{y}_{\approx n/2}) \right\}$, implementar el modelo lineal y comparar la predicci�n del modelo con las pr�ximas mediciones de referencia $\mathbb{D}_{\approx n/2 \rightarrow n} = \left\{ (x_{\approx n/2 +1}, y_{\approx n/2 +1}), (x_{\approx n/2 +2}, y_{\approx n/2 +2}) \cdots (x_{n}, y_{n}) \right\}$. \\ \textbf{Determine cual el alcance para realizar predicciones dentro de la tolerancia}.

El rango de los valores para generar el modelo (en este caso lineal) y el alcance de su predicci�n es un arte que debe definirse para cada conjunto de datos. Arriba se propuso utilizar la mitad del conjunto de datos para modelar, pero eso en general no es necesario. \textbf{Determine entonces, el m�nimo conjunto de datos para generar el modelo y cu�l ser� su m�ximo alcance para una tolerancia dada.} 
%\subsection{}


\bibliographystyle{alpha}
\bibliography{Taller1}

\end{document}  