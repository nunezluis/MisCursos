\documentclass[spanish,notitlepage,letterpaper,11pt]{article} % para artÌculo en castellano
\usepackage[utf8]{inputenc} % Acepta caracteres en castellano
\usepackage[T1]{fontenc} % font encoding
\usepackage[spanish]{babel} % silabea palabras castellanas
\usepackage[none]{hyphenat} 
\usepackage{amsmath}
\usepackage{color}
\definecolor{labelcolor}{RGB}{100,0,0}
\usepackage{amsfonts}
\usepackage{amssymb}
\usepackage[colorlinks=true,urlcolor=blue,linkcolor=blue]{hyperref} % navega por el doc
\usepackage{graphicx}
\usepackage{geometry} % See geometry.pdf to learn the layout options.
\geometry{letterpaper}  % ... or a4paper or a5paper or ... 
%\geometry{landscape}  % Activate for for rotated page geometry
%\usepackage[parfill]{parskip} % Activate to begin paragraphs with an empty line rather than an indent
\usepackage{epstopdf}
\usepackage{fancyhdr} % encabezados y pies de pg
\usepackage{lineno}
%\linenumbers

\pagestyle{fancy} 
\chead{\bfseries El látigo y las ondas de choque} 
\lhead{\it Núñez  } % si se omite coloca el nombre de la seccion
\rhead{Agosto 2018} 
\lfoot{Escuela de Física} 
\cfoot{Universidad Industrial de Santander} 
\rfoot{\thepage} 

\voffset = -0.25in 
\textheight = 8.0in 
\textwidth = 6.5in
\oddsidemargin = 0.in
\headheight = 20pt 
\headwidth = 6.5in
\renewcommand{\headrulewidth}{0.5pt}
\renewcommand{\footrulewidth}{0,5pt}
\DeclareGraphicsRule{.tif}{png}{.png}{`convert #1 `dirname #1`/`basename #1 .tif`.png}


\begin{document}
\title{El látigo y las ondas de choque}
\author{
\textbf{Luis A. Núñez} \\ 
\textit{Escuela de Física} \\ \textit{Universidad Industrial de Santander} \\ 
\textit{Bucaramanga, Colombia}
}
\date{\today}
\maketitle
%\tableofcontents


\section{Introducción}
Los látigos son objetos famosos y asociados a dos de los mas connotados héroes de mi infancia y adolescencia: el Zorro e Indiana Jones\footnote{\url{https://mappingignorance.org/2014/09/29/whip-electric-jet/}}. 

El chasquido de este instrumento nos recuerda su uso como arma o como elemento de tortura. Pero es impresionante que ese chasquido ocurre cuando el extremo del látigo se mueve más rápido que la velocidad del sonido creando un pequeño estampido sónico\cite{BernsteinHallTrent1958,GorielyMcMillen2002}.  El análisis del movimiento del extremo de  este instrumento ha sido objeto de estudio desde los tiempos de Galileo y constituye un terreno ideal para la aplicación de técnicas de solución de ecuaciones diferenciales con valores de frontera \cite{Preston2011,Telciyan2018}.

Pueden ver un par de videos ilustrativos del problema 
\begin{itemize}
    \item \url{https://www.youtube.com/watch?v=3HABgm-UUi0}
    \item \url{https://www.youtube.com/watch?v=AnaASTBn_K4&feature=youtu.be}
\end{itemize}

\section{El problema}
El problema consiste en calcular la velocidad (y el desplazamiento) de un extremo de una cuerda de longitud $L$, que reacciona un impulso que se le imprime en el otro extremo. Los látigos son un sistema (una cuerda inextensible) de masa variable, donde la masa por unidad de longitud en un extremo $m_{0}$ es mucho mayor que la masa por unidad de longitud en  el otro extremo: $m_{0} >> m_{L}$.

\section{Aproximaciones al problema}
Quizá antes de abordar el problema mas general valga la pena analizar algunos casos mas simplificados:
\begin{itemize}
    \item Considere primero el caso de una cuerda con la misma \textit{masa por unidad de longitud} a la cual se le imprime un pulso en uno de sus extremos. Calcular la velociadad y el desplazamiento del otro extremo.
    \item Seguidamente considere la misma cuerda pero un cambio de la función masa (digamos en $L/2$) de una masa $m_{1}$ antes de $L/2$ y otra $m_{2}$ para despúes de $L/2$\cite{RodriguezzuritaEtal2002modos}.
    \item finalmente, considere el problema de masa variable y calcule la velocidad del extremo mas delgado.
    \item supondremos que el caso del látigo. Vale decir la cuerda de masa variable. Este problema lo debe considerar tanto desde el punto de vista analítico como numérico, resolviéndolo mediante diferencias finitas.
    \item Discuta como cambiarían los resultados si la cuerda fuera extensible.
\end{itemize}

\bibliographystyle{unsrt}
\bibliography{latigo}
\end{document}

\begin{abstract}
    Presentamos una propuesta para realizar un experimento que le permita estimar el valor de la aceleración de gravedad a partir de la medición del período de oscilación de un péndulo.  Para ello, proponemos realizar el experimento con un péndulo oscilando con pequeñas amplitudes (por ejemplo $\theta_{0}\approx 5^{o}$). Luego, procedemos a identificar de qué depende el período de oscilación del péndulo para cualquier tipo de amplitud (por ejemplo $\theta_{0}\approx 30^{o}$, $\theta_{0}\approx 45^{o}$ y $\theta_{0}\approx 60^{o}$).  
    
    En ambos casos deben comparar el experimento con las simulaciones, estimar los errores sistemáticos y la precisión de sus mediciones. Para determinar la precisión de la medición se recomienda realizar cada experimento al menos $10$ veces.
    
    Para reportar los resultados de los experimentos, deberá entregar un reporte técnico de la experiencia, los archivos de datos de las mediciones, los códigos de MAXIMA y una presentación de un maximo 6 láminas donde narre la experiencia.
\end{abstract}