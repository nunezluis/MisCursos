


\documentclass[11pt]{article}
\usepackage[utf8]{inputenc}

\title{Taller Transformadas Discretas de Fourier}
\author{{\bf Luis Núñez} \\ Métodos Matemáticos 2 \\ 
Escuela de Física Universidad Industrial de Santander
}
\date{Fecha de entrega: 24 Septiebre 2021}

\begin{document}
\maketitle

\begin{enumerate}
    \item Considere las siguientes señales
    \[y(t)=3 \cos (\omega t) + 5 \cos (3 \omega t) +8\cos (7 \omega t) 
    \quad {\rm y } \quad y(t)= 3\,{\rm sen}\,(\omega t) +2\,{\rm sen}\,(3 \omega t) +3\,{\rm sen}\, (8 \omega t)
    \] 
    \begin{enumerate}
        \item Implemente en cada caso la transformada analítica y discreta de Fourier. Compárelas.
        \item Muestre para cuáles casos recupera una transformada real o imaginiaria.
        \item Encuentre las distintas componentes para el espectro de potencias y muestre que las frecuencias tienen los valores esperados (no sólo las proporciones)
        \item Experimentar los efectos de elegir diferentes valores del tamaño de paso $h$ y de ampliar el período de medición $T=N h$.
    \end{enumerate}
    %
    \item Considere ahora la siguiente señal
    \[ y(t_i) = \frac{10}{10 - 9\,{\rm sen}\,(t_i)} 
    + \alpha(3 \mathcal{R}_i -1) \quad {\rm con} \quad 0 \leq \mathcal{R}_i \leq -1\]
    donde $\mathcal{R}_i$ es un número aleatorio y $\alpha$ un parámetro de control. Use este parámetro $\alpha$ para simular tres tiepos de ruido: alto, medio y bajo. En el caso del ruido alto, la señal se pierde. En los otros dos casos puede ser identificada
    \begin{enumerate}
        \item Grafique su señal ruidosa, su transformada de Fourier (analítica) y su espectro de potencia
        \item Calcule la función de autocorrelación $A(\tau)$ y su transformada de Fourier $A(\omega)$.
        \item Compare la transformada discreta de fourier de $A(\tau)$ con el verdadero espectro de potencia y discuta la eficacia de la reducción del ruido mediante el uso de la función de autocorrelación.
        \item ¿Para cuales valores de $\alpha$ se pierde toda la información en la entrada?
    \end{enumerate}
    
    
\end{enumerate}


\end{document}



