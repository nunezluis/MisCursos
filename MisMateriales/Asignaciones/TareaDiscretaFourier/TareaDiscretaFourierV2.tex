%\documentclass[11pt]{article}
\documentclass[spanish,notitlepage,letterpaper,11pt]{article} % para articulo en castellano
%\usepackage[utf8]{inputenc}
\usepackage[ansinew]{inputenc} % Acepta caracteres en castellano
\usepackage[spanish]{babel}    % silabea palabras castellanas


\title{Asignaci�n de Transformadas Discretas de Fourier}
\author{{\bf Luis N��ez} \\ Matem�ticas Avanzadas \\ 
Escuela de F��sica Universidad Industrial de Santander
}
 
\begin{document}
\maketitle
Esta asignaci�n tiene dos partes. La primera constituye un calentamiento, un ejercicio de los conceptos que hemos desarrollado en clase y la segunda un ``trabajo de campo'' con datos reales a ver como se comportan los conceptos. 

\section{Calentamiento}
Empecemos con el calentamiento. 
\begin{enumerate}
    \item Considere las siguientes se�ales
    \[y(t)=3 \cos (\omega t) + 5 \cos (3 \omega t) +8\cos (7 \omega t) 
    \quad {\rm y } \quad y(t)= 3\,{\rm sen}\,(\omega t) +2\,{\rm sen}\,(3 \omega t) +3\,{\rm sen}\, (8 \omega t)
    \] 
    \begin{enumerate}
        \item Implemente en cada caso la transformada anal��tica y discreta de Fourier. Comp�relas.
        \item Muestre para cu�les casos recupera una transformada real o imaginiaria.
        \item Encuentre las distintas componentes para el espectro de potencias y muestre que las frecuencias tienen los valores esperados (no s�lo las proporciones)
        \item Experimente los efectos de elegir diferentes valores del tama�o de paso $h$ y de ampliar el per��odo de medici�n $T=N h$.
    \end{enumerate}
    %
    \item Considere ahora la siguiente se�al
    \[ y(t_i) = \frac{10}{10 - 9\,{\rm sen}\,(t_i)} 
    + \alpha(3 \mathcal{R}_i -1) \quad {\rm con} \quad 0 \leq \mathcal{R}_i \leq -1\]
    donde $\mathcal{R}_i$ es un n�mero aleatorio y $\alpha$ un par�metro de control. Use este par�metro $\alpha$ para simular tres tipos de ruido: alto, medio y bajo. En el caso del ruido alto, la se�al se pierde. En los otros dos casos puede ser identificada
    \begin{enumerate}
        \item Grafique su se�al ruidosa, su transformada de Fourier (anal��tica) y su espectro de potencia
        \item Calcule la funci�n de autocorrelaci�n $A(\tau)$ y su transformada de Fourier $A(\omega)$.
        \item Compare la transformada discreta de Fourier de $A(\tau)$ con el verdadero espectro de potencia y discuta la eficacia de la reducci�n del ruido mediante el uso de la funci�n de autocorrelaci�n.
        \item �Para cu�les valores de $\alpha$ se pierde toda la informaci�n en la entrada?
    \end{enumerate}
\end{enumerate}

\end{document}



