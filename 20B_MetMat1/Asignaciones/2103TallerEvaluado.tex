\documentclass[spanish,notitlepage,letterpaper]{article} % para articulo en castellano
\usepackage[ansinew]{inputenc} % Acepta caracteres en castellano
\usepackage[spanish]{babel}    % silabea palabras castellanas
\usepackage{amsmath}
\usepackage{amsfonts}
\usepackage{amssymb}
\usepackage[colorlinks=true,urlcolor=blue,linkcolor=blue]{hyperref} % navega por el doc
\usepackage{graphicx}
\usepackage{geometry}           % See geometry.pdf to learn the layout options.
\geometry{letterpaper}          % ... or a4paper or a5paper or ... 
%\geometry{landscape}           % Activate for for rotated page geometry
%\usepackage[parfill]{parskip}  % Activate to begin paragraphs with an empty line rather than an indent
\usepackage{epstopdf}
\usepackage{fancyhdr} % encabezados y pies de pg

\pagestyle{fancy} 
\chead{} 
\lhead{\textit{ M�todos Matem�ticos 1}} % si se omite coloca el nombre de la seccion
\rhead{\textbf{Taller }} 
\lfoot{Luis A. N��ez} 
\rfoot{Universidad Industrial Santander} 
%\rfoot{\thepage} 

\voffset = -0.25in 
\textheight = 8.0in 
\textwidth = 6.5in
\oddsidemargin = 0.in
\headheight = 20pt 
\headwidth = 6.5in
\renewcommand{\headrulewidth}{0.5pt}
\renewcommand{\footrulewidth}{0,5pt}
\DeclareGraphicsRule{.tif}{png}{.png}{`convert #1 `dirname #1`/`basename #1 .tif`.png}



\begin{document}
\begin{center}
\textbf{Taller evaluado de M�todos Matem�ticos 1}
\end{center}

Dado el siguiente campo vectorial ${\bf A} = 4xy \, {\mathbf{i}} -y^{2}\, {\mathbf{j}} +z\, {\mathbf{k}}$. 

\begin{enumerate}
\item Calculemos el flujo de ${\bf A}$ a trav�s de un cubo unitario: $x = 0, \, x = 1, \, y = 0, \, y = 1, \, z = 0, \, z = 1$.

\item Consideremos ahora un cilindro $x^{2} + y^{2} = 16$  con $-1 \leq z \leq 1$
	\begin{enumerate}
	\item Calculemos el flujo del campo a trav�s de esa superficie. Muestre que el teorema de la divergencia es v�lido. Esto es: calcule los dos lados de la igualdad y muestre que se cumplen.
	\item Encontremos el flujo del rotacional de ese campo para el caso  $z \geq 0$.  
	\end{enumerate}
\item Calculemos las componentes longitudinales y transversales del campo ${\bf A}$.  Como saben, todo campo vectorial ${\bf A}$ se puede separar en una componente longitudinal ${\bf A}^{l}$, y otra transversal ${\bf A}^{t}$. Esto es:
 \[
{\bf A}={\bf A}^{l}+{\bf A}^{t}\quad\text{con}\quad 
\left\{
\begin{array}
[c]{c}
{\boldsymbol \nabla}\times{\bf A}^{l}=0\\
\\
{\boldsymbol \nabla}\cdot{\bf A}^{t}=0
\end{array}
\right\} 
\]
\item Encontremos los potenciales escalar y vectorial asociados a este campo vectorial ${\bf A}$, a trav�s de las componentes longitudinal 
\end{enumerate}	

\end{document}  